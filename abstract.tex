\newpage
\begin{center}
{\Large Statistical Process Control for Multivariate Categorical Processes}\\
\vspace{10mm}
by Jian LI\\
\vspace{15mm}
Department of Industrial Engineering and Logistics Management\\
\vspace{10mm}
The Hong Kong University of Science and Technology
\end{center}
\vspace{12mm}
\begin{center}
\textbf{Abstract}
\end{center}
\par
\noindent

Whatever is the business, either manufacturing or service, quality is the most
critical aspect, which affects the level of success of the business. The importance
of quality cannot be overemphasized. According to its modern definition that quality
is inversely proportional to variability, quality improvement is to reduce
variability. Statistical process control (SPC) is such a methodology that has been
widely used for detecting assignable causes and reducing variability.

Recently, in manufacturing and especially service industries, there have been more
and more processes involving quality characteristics expressed as attribute levels
such as conforming or nonconforming, which are not measured on a continuous scale.
This follows partly because their accurate numerical values are expensive or even
impossible to obtain and partly because such categorical values are sufficient for
monitoring purpose. We consider monitoring processes that involve multiple
categorical quality characteristics, namely multivariate categorical processes.
Usually, these categorical characteristics correlate with each other, indicating a
must of multivariate charting techniques.

Sufficient attention has been paid to monitoring multivariate continuous data, for
example, data collected from a multivariate normal distribution. Such methods are
easily found in the literature. However, there is a scarcity of research on
monitoring multivariate categorical data, and most of the few existing methods lack
robustness in two aspects. First, they apply to multiple characteristics that all
have only two attribute levels. If at least one characteristic has three or more
levels, they will fail. Second, they do not care the cross-classifications among
characteristics. Instead, they focus on only the one-way marginal counts with
respect to the attribute levels of each characteristic, neglecting the dependence
among characteristics.

We employ log-linear models for describing the relationship among categorical
characteristics, which can overcome the above-mentioned two difficulties. This
thesis tries to make three major contributions to statistical process control for
multivariate categorical processes. First, a general Phase II control chart is
proposed for online monitoring, which is robust to detect various shifts
efficiently, especially those in interaction effects representing the dependence
among characteristics. Second, another Phase II control chart that exploits
directional shift information is devised for online detection of some more practical
shifts, namely one-coefficient shifts and high-order interaction shifts. Third, an
off-line Phase I analysis method that considers directional shift information is
proposed for change-point detection in a dataset collected from a multivariate
categorical process.

The three approaches compose a systematic methodology of statistical process control
for multivariate categorical processes. We have also performed numerical simulations
to compare them with corresponding counterparts. In addition, real-data examples
have demonstrated the effectiveness of these techniques. They can be implemented
into practice with reliability.

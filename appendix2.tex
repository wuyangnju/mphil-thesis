\chapter{Proofs for Chapter 3}\label{c3:appendix}



\section{Proof of Corollary \ref{cor:inner}}

\begin{proof}
First we have
\[\mbox{CVaR}_{1-\alpha}(c(y,\xi))-\frac{1}{\alpha}\mbox{L}(y,y)=\mbox{CVaR}_{1-\alpha}
(c(y,\xi))-\frac{1}{\alpha}\E\left[[c(y,\xi)]^+\right]\le0.\]
Therefore $y\in\mathcal {D}_y$. Consider now $x\in \mathcal {D}_y$
and $x\not=y$. From Theorem \ref{thm:CVaRlike} we have
$p(x)\le \alpha$.  %Let $t^*=\mbox{VaR}_{1-\alpha}(c(x,\xi))$. Then we
%have $t^*\le0$.
Now we only need to exclude the case where $p(x)=\alpha$. Suppose
$p(x)=\alpha$. We define $A(y)=\left\{\xi\in\Xi:
c(y,\xi)\geq0\right\}$ and $A(x)=\left\{\xi\in\Xi:
c(x,\xi)>0\right\}$. Then we have $\Pr\{A(y)\}=p(y)<\alpha$ and
$\Pr\{A(x)\}=p(x)=\alpha$. Define
\[A(x)\setminus A(y)=\left\{\xi\in \Xi: \xi\in A(x),\xi\not\in
A(y)\right\}.\] Then we have $\Pr\{A(x)\setminus A(y)\}>0$. For any
$\xi\in A(x)\setminus A(y)$,
\[\left[c(x,\xi)\right]^+>0=\left[c(y,\xi)\right]^+
+\left[\nabla_x\left[c(y,\xi)\right]^+\right]^{\rm
T}\left(x-y\right)\] and for any $\xi\not\in A(x)\setminus A(y)$,
noting the convexity of $c(x,\xi)$, we have
\[\left[c(x,\xi)\right]^+\geq\left[c(y,\xi)\right]^+
+\left[\nabla_x\left[c(y,\xi)\right]^+\right]^{\rm
T}\left(x-y\right).\] It follows that
$\E\left[\left[c(x,\xi)\right]^+\right]>\mbox{L}(x,y)$ and therefore
$x\not\in \mathcal {D}_y$. We obtain a contradiction. This finishes
the proof of the corollary.
\end{proof}



\section{Proof of Theorem \ref{property-nKKT}}

\begin{proof} We prove the four properties one by one.

1. This can be proved by induction. Note that from Assumption
\ref{assu:dominate} we have $c(x,\xi)$ is a continuous random
variable. Since $x_0$ is an optimal solution of Problem
(\ref{cvarprogram}), we have $p(x_0)<\alpha$. Now suppose $x_k$
satisfies $p(x_k)<\alpha$. Since $x_{k+1}$ is an optimal solution of
Problem \mbox{$ ({\rm C_k})$}, we have $x_{k+1}\in \Omega_{x_k}$. It
follows from Corollary \ref{cor:inner} that $p(x_{k+1})<\alpha$.
Then by induction we have $x_k$ is always a strictly feasible
solution of Problem (\ref{eq:ccp2}), i.e., $p(x_k)<\alpha$. Again,
using the continuity of the distribution of $c(x_k,\xi)$ at $0$, we
have
\[\mbox{CVaR}_{1-\alpha}(c(x_k,\xi))-\frac{1}{\alpha}\mbox{L}(x_k,x_k)=
\mbox{CVaR}_{1-\alpha}(c(x_k,\xi))-\frac{1}{\alpha}
\E\left[[c(x_k,\xi)]^+\right]<0.\]

2. Since $x_k$ is always a feasible solution of Problem \mbox{$
({\rm C_k})$} and $x_{k+1}$ is an optimal solution of Problem
\mbox{$ ({\rm C_k})$}, we have $h(x_{k+1})\le h(x_k)$. Therefore
$\{h(x_k),k=1,2,\cdots\}$ is a non-increasing sequence. Note that
$\{h(x_k),k=1,2,\cdots\}$ is also bounded. Then we have
$\{h(x_k),k=1,2,\cdots\}$ is convergent. Moreover, if at some
iteration $k$, $h(x_{k+1})= h(x_k)$, then $x_k$ is also an optimal
solution of Problem \mbox{$ ({\rm C_k})$}. We now argue by a
contradiction. Suppose $x_k$ is not a global optimal solution of
Problem (\ref{eq:ccp2}). Since Problem (\ref{eq:ccp2}) is a convex
optimization problem. we have $x_k$ is not a stationary point of
Problem (\ref{eq:ccp2}). It follows that we can find a sequence
$\left\{x_k^j, j=1,2,\cdots\right\}\subset X$ such that $x_k^j\to
x_k$ as $j\to \infty$ and $h(x_k^j)<h(x_k)$ for all $j=1,2,\cdots$.
From the first property, we have
$\mbox{CVaR}_{1-\alpha}(c(x_k,\xi))-\frac{1}{\alpha}\mbox{L}(x_k,x_k)<0$.
By continuity we can find some $x_k^{j^*}$ such that
$\mbox{CVaR}_{1-\alpha}(c(x_k^{j^*},\xi))-
\frac{1}{\alpha}\mbox{L}(x_k^{j^*},x_k)<0$. This means $x_k^{j^*}$
is also a feasible solution of Problem \mbox{$ ({\rm C_k})$}. This
contradicts with the fact that $x_k$ is an optimal solution of
Problem \mbox{$ ({\rm C_k})$}. Therefore $x_k$ must be a global
optimal solution of Problem (\ref{eq:ccp2}).
%3. All cluster points of $\{x_k,k=1,2,\cdots\}$ satisfy the KKT
%conditions of Problem (\ref{eq:ccp2}). How to prove it?
%4. If Problem (\ref{eq:ccp2}) has finite number of cluster points,
%then the sequence $\{x_k,k=1,2,\cdots\}$ converges to some point
%$x^*$ which satisfies the KKT conditions of Problem (\ref{eq:ccp2}).
%We finished the proof of Theorem \ref{property-nKKT}. How to prove
%it?
\end{proof}

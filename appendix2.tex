\chapter{User Guide of RASE}

To develop your own simulation code, the following steps are suggested:

1.	Install eclipse (\url{http://www.eclipse.org});

2.	Inside eclipse, File $>$ New $>$ Java Project and follow the wizard;

3.	Import rase-sim-1.0.jar, which can be found \url{http://143.89.20.47/rase/guide/rase-sim-1.0.jar}.

4.	Write your own simulation code, which needs to implement the Sim interface, like the example \url{http://143.89.20.47/rase/guide/NormalSim.java}. This example just returns a normally distributed sample value.

To implement the Sim interface, you only need to implement one method:

\begin{lstlisting}[language=Java]
public double[] sim(double[] args, long[] seed)
\end{lstlisting}

Here args is the arguments that distinguish an alternative, while seed determines the result of this simulation. The output should be an array of double.

5.	Package your project into a jar file. Just right click the project $>$ Export $>$ Java $>$ JAR file, and follow the wizard.

After developing the jar file, please move to \url{http://143.89.20.47/rase/}. If the server is available, you will see a form about ranking and selection configuration, namely the figure in \ref{web-ui}. We will give a detailed description here.

1.	Procedure part:

If you only focus on the simulation part, please just choose build-in procedure and fill in the argument like example.

2.	Alternatives:

You need a plain text file specifying the configuration of each alternative, one line for each alternative. When running simulation for some certain alternative, the corresponding line will be passed as the args to the sim function you have just written. \url{http://143.89.20.47/rase/guide/normal.5.alts} is an example describing five alternatives, related to the simulation example mentioned before.

3.	Simulation

Just choose Jar and upload your jar file, and specifying the class(including package name)

4.	Simulation Threads

Currently e we only support single machine version, which means that you can specify the number of threads on Felab-1. We will support cluster version later.

5.	Repeat Times:

It’s the number of replication.

If you want to develop your own ranking and selection code, the steps are quite similar, except that the jar file that needs to import is \url{http://143.89.20.47/rase/guide/rase-ras-1.0.jar}, and we have these examples, they’re Rinott’s Procedure (\url{http://143.89.20.47/rase/guide/RinottRas.java}), Table Filling Procedure (\url{http://143.89.20.47/rase/guide/TableFillingRas.java}) and Parallel Sequential Procedure (\url{http://143.89.20.47/rase/guide/ParallelSequentialRas.java}).

This time you need to implement Ras interface, which also only contains one function:

\begin{lstlisting}[language=Java]
public int ras(double[] args, double[][] alts,
        SimHelper simHelper)
\end{lstlisting}

Here args is the ones that passed from the webpage, while alts is read in from the file uploaded. To understand the usage of SimHelper, please see the examples above and the corresponding Java doc.

\chapter{Proofs for Chapter 4}\label{c4:appendix}



\section{Proof of Theorem \ref{th:properties-of-SA}}

\begin{proof}
We first prove (a). For any $\mu>0$ and any $x\in
\Phi(\varepsilon,\mu)$, we have from (\ref{ul}) that
\begin{equation*}
g_1(x,\varepsilon)-g_2(x,\varepsilon)\le \bar
g_1(x,\varepsilon,\mu)-\bar g_2(x,\varepsilon,\mu)\le 0,
\end{equation*}
which implies $x\in\Omega_{\varepsilon}$. Therefore the inclusion
$\Phi(\varepsilon,\mu)\subset \Omega_\varepsilon$ holds.

Secondly we prove (b). Let
\begin{equation}\label{lambar}
\bar\lambda_j(x,\varepsilon,\xi,\mu)=\displaystyle
\frac{\exp\left(\mu^{-1}\left(c_j(x,\xi)+\varepsilon\right)\right)}{\displaystyle
\sum_{i=0}^m\exp\left(\mu^{-1}\left(c_i(x,\xi)+\varepsilon\right)\right)},\,\,j=0,1,\cdots,m
\end{equation}
and
%\[\widetilde\lambda_0(x,\varepsilon,\xi,\mu)=\displaystyle
%\frac{1}{1+\displaystyle
%\sum_{j'=1}^m\exp\left(\mu^{-1}c_{j'}(x,\xi)\right)}\] and
\begin{equation*}
\widetilde \lambda_j(x,\varepsilon,\xi,\mu)=\displaystyle
\frac{\exp\left(\mu^{-1}c_{j}(x,\xi)\right)}{\displaystyle
\sum_{i=0}^m\exp\left(\mu^{-1}c_{i}(x,\xi)\right)},\,\,j=0,1,\ldots,m
\end{equation*}
where we define both $c_0(x,\xi)\equiv0$ and
$c_0(x,\xi)+\varepsilon\equiv0$ for consistency of notation. Then
simple calculation shows
\[H_2(x,\varepsilon,\xi,\mu)=\displaystyle \sum_{j=0}^m\widetilde
\lambda_j(x,\varepsilon,\xi,\mu)c_{j}(x,\xi)-\mu\displaystyle
\sum_{j=0}^m\widetilde\lambda_j(x,\varepsilon,\xi,\mu)
\log[\widetilde \lambda_j(x,\varepsilon,\xi,\mu)].\]
%\[\bar\lambda_0(x,\varepsilon,\xi,\mu)=
%\displaystyle \frac{1}{1+\displaystyle
%\sum_{j'=1}^m\exp\left(\mu^{-1}\left(c_{j'}(x,\xi)+\varepsilon\right)\right)}\]
%Then
%\[\nabla_x\bar l_1(x,\varepsilon,\xi,\mu)=\displaystyle \sum_{j=0}^m\bar
%\lambda_j(x,\varepsilon,\xi,\mu)\nabla_x
%\left(c_{j}(x,\xi)+\varepsilon\right).\]

By calculating the derivative of $\bar g(x,\varepsilon,\mu)$ with
respect to $\mu$, we have
\[\begin{array}{ll}
\displaystyle\frac{\partial}{\partial\mu}\bar g(x,\varepsilon,\mu)
&=\mu^{-1}\E\left[H_1(x,\varepsilon,\xi,\mu)-\displaystyle\sum_{j=0}^m\bar
\lambda_j(x,\varepsilon,\xi,\mu)\left(c_j(x,\xi)+
\varepsilon\right)\right]\\[6pt]
&\quad \quad
-\mu^{-1}\E\left[H_2(x,\varepsilon,\xi,\mu)-\displaystyle\sum_{j=0}^m\widetilde
\lambda_j(x,\varepsilon,\xi,\mu)c_j(x,\xi)\right]+\log(m+1)\\
&\geq\mu^{-1}\E\left[\left[c(x,\xi)+\varepsilon\right]^+-\displaystyle
\sum_{j=0}^m\bar \lambda_j(x,\varepsilon,\xi,\mu)\left(c_j(x,\xi)+
\varepsilon\right)\right]\\[6pt]
&\quad \quad
-\mu^{-1}\E\left[H_2(x,\varepsilon,\xi,\mu)-\displaystyle\sum_{j=0}^m\widetilde
\lambda_j(x,\varepsilon,\xi,\mu)c_j(x,\xi)\right]+\log(m+1)\\[6pt]
&\geq -\mu^{-1}\E\left[H_2(x,\varepsilon,\xi,\mu)-\displaystyle
\sum_{j=0}^m\widetilde\lambda_j(x,\varepsilon,\xi,\mu)c_j(x,\xi)\right]+\log(m+1)\\[6pt]
&=\E\left[\displaystyle \sum_{j=0}^m\widetilde
\lambda_j(x,\varepsilon,\xi,\mu)\log[\widetilde
\lambda_j(x,\varepsilon,\xi,\mu)]\right]+
\log(m+1)\mbox{ (here }0\log 0\equiv 0)\\[6pt]
&\geq \displaystyle \min \left \{\sum_{j=0}^m \lambda_j\log
\lambda_j: \sum_{j=0}^m \lambda_j=1, \lambda_j \geq 0, j=0,\ldots,
m\right \}+\log (m+1)\\[6pt]
&=0.
\end{array}\]
Therefore the function $\bar g(x,\varepsilon,\mu)$ is nondecreasing
in $\mu$. Thus we have $\bar g(x,\varepsilon,\mu_1)\leq \bar
g(x,\varepsilon,\mu_2)$ for $0<\mu_1\leq \mu_2$ and
$\Phi(\varepsilon,\mu_2)\subset \Phi(\varepsilon,\mu_1)$.

Next we prove (c). As $\Phi(\varepsilon,\mu)$ is monotone in $\mu$,
it follows from Exercise 4.3 of Rockafellar and Wets (1998) that
$\lim_{\mu\searrow 0} \Phi(\varepsilon,\mu)$ exists. From (a), we
can easily get the inclusion $\lim_{\mu\searrow 0}
\Phi(\varepsilon,\mu)\subset\Omega_\varepsilon$. We only need to
prove the opposite inclusion.

For any $x\in \Omega_\varepsilon^I$, let
$\eta=g_2(x)-g_1(x,\varepsilon)$, then $\eta>0$. Let $\bar
\mu=\displaystyle \frac{\eta}{4\log(m+1)}$, then from (\ref{ul}) we
have
\[\bar g_1(x,\varepsilon,\mu)-\bar g_2(x,\varepsilon,\mu)\leq
g_1(x,\varepsilon)-g_2(x)+2\mu\log(m+1)\leq -\eta+\eta/2=-\eta/2<0\]
for any $\mu\in (0,\bar\mu)$, which implies that
$x\in\Phi(\varepsilon,\mu)$ when $\mu\in(0,\bar\mu)$. Therefore we
have $x\in\displaystyle \lim_{\mu\searrow 0} \Phi(\varepsilon,\mu)$.
So we obtain that $\displaystyle \lim_{\mu \searrow
0}\Phi(\varepsilon,\mu)\supset\Omega_{\varepsilon}^I$. Since
$\displaystyle\lim_{\mu\searrow 0}\Phi(\varepsilon,\mu)$ is a closed
set, we have by Assumption \ref{assu:omega},
$\displaystyle\lim_{\mu\searrow
0}\Phi(\varepsilon,\mu)\supset\Omega_\varepsilon$.

Finally we prove (d). Let
$\bar{h}(x)=h(x)+I_{\Omega_\varepsilon}(x)$ and
$\bar{h}_{\mu}(x)=h(x)+I_{\Phi(\varepsilon,\mu)}(x)$, where
$I_{A}(x)=0$ if $x\in A$ and $I_{A}(x)=+\infty$ if $x\not\in A$. By
(c), $\lim_{\mu\searrow 0}\Phi(\varepsilon,\mu)=\Omega_\varepsilon$.
Then by Proposition 7.4(f) of Rockafellar and Wets (1998), we have
that $I_{\Phi(\varepsilon,\mu)}(\cdot)$ epi-converges to
$I_{\Omega_\varepsilon}(\cdot)$ as $\mu\searrow 0$. Since $h(\cdot)$
is continuous, we have that $\bar{h}_{\mu} (\cdot)$ epi-converges to
$\bar{h}(\cdot)$ as $\mu\searrow 0$. As $\Phi(\varepsilon,\mu)$ and
$\Omega_\varepsilon$ are compact, we have that
$\bar{h}_{\mu}(\cdot)$ and $\bar{h} (\cdot)$ are lower
semi-continuous and proper. Then by Theorem 7.33 of Rockafellar and
Wets (1998), we have
$\nu(\varepsilon,\mu)\rightarrow\nu(\varepsilon)$ and
\begin{equation}\label{eq:set-inc}
\displaystyle \limsup_{\mu\searrow 0}S(\varepsilon,\mu)\subset
S(\varepsilon).
\end{equation}
Noting that $S(\varepsilon,\mu)$ and $S(\varepsilon)$ are uniformly
compact (they are subsets of the compact set $X$), we have from the
discussions in Example 4.13 of Rockafellar and Wets (1998) that
Equation (\ref{eq:set-inc}) implies that $\lim_{\mu\searrow 0}
\mathbb{D}(S(\varepsilon,\mu),S(\varepsilon))=0$. This concludes the
proof of (d).
\end{proof}


%\subsection{Proof of Lemma \ref{lem:diffs}}


\section{Proof of Theorem \ref{th:KKT}}

To prove Theorem \ref{th:KKT}, we need the following lemma.

\begin{lemma}\label{lem:diffs}
Let $x\in X$. Then
\begin{description}
\item[(a)] For $i=1$ or $i=2$, the limit
$\lim_{\mu\searrow 0}\nabla_x\bar g_i(x,\varepsilon,\mu)$ exits and
belongs to $\partial g_i (x,\varepsilon)$;
\item[(b)] For $i=1$ or $i=2$, one has
\begin{equation}\label{eq:diff-inclusion}
\displaystyle \limsup_{x' \rightarrow x, \mu \searrow
0}\Big\{\nabla_x\bar g_i(x',\varepsilon,\mu)\Big\}\subset
\partial g_i(x,\varepsilon).
\end{equation}
\end{description}
\end{lemma}

\begin{proof}
Without loss of generality, we only prove the
conclusion for $i=1$. %Let
%\[\bar\lambda_0(x,\varepsilon,\xi,\mu)=
%\displaystyle \frac{1}{1+\displaystyle
%\sum_{j'=1}^m\exp\left(\mu^{-1}\left(c_{j'}(x,\xi)+\varepsilon\right)\right)}\]
%\[\bar \lambda_j(x,\varepsilon,\xi,\mu)=
%\displaystyle
%\frac{\exp\left(\mu^{-1}\left(c_j(x,\xi)+\varepsilon\right)\right)}{\displaystyle
%\sum_{i=0}^m\exp\left(\mu^{-1}\left(c_i(x,\xi)+\varepsilon\right)\right)},\,\,j=0,1,\cdots,
%m\] where we also define $c_0(x,\xi)+\varepsilon\equiv0$ for
%consistency.

(a). Since
\[\nabla_xH_1(x,\varepsilon,\xi,\mu)=\displaystyle\sum_{j=0}^m
\bar\lambda_j(x,\varepsilon,\xi,\mu)\nabla_x\left(c_{j}(x,\xi)+
\varepsilon\right)\] where $\bar\lambda_j(x,\varepsilon,\xi,\mu),
j=0,1,\cdots,m$ are defined by (\ref{lambar}), we have
\begin{equation*}\label{eq:nabla-g1}
\nabla_x\Psi_1(x,\varepsilon,\mu)=\E\left[\nabla_xH_1(x,
\varepsilon,\xi,\mu)\right]=\E\left[\displaystyle\sum_{j=0}^m
\bar\lambda_j(x,\varepsilon,\xi,\mu)\nabla_x
\left(c_{j}(x,\xi)+\varepsilon\right)\right].
\end{equation*}
Note that we can rewrite $\bar\lambda_j(x,\varepsilon,\xi,\mu),
j=0,1,\cdots,m$ as
%\[\bar\lambda_0(x,\varepsilon,\xi,\mu)=\displaystyle\frac{\exp\left(-\mu^{-1}[c(x,\xi)+
%\varepsilon]^+\right)}{\exp\left(-\mu^{-1}[c(x,\xi)+\varepsilon]^+\right)+\displaystyle
%\sum_{j'=1}^m\exp\left(\mu^{-1}\left(c_{j'}(x,\xi)+\varepsilon-[c(x,\xi)+\varepsilon]^+\right)\right)}\]
%and
\[\bar\lambda_j(x,\varepsilon,\xi,\mu)=\displaystyle\frac{\exp\left(\mu^{-1}
\left(c_j(x,\xi)+\varepsilon-[c(x,\xi)+\varepsilon]^+\right)\right)}
{\displaystyle\sum_{i=0}^m\exp\left(\mu^{-1}\left(c_i(x,\xi)+\varepsilon
-[c(x,\xi)+\varepsilon]^+\right)\right)}, j=0,1,\cdots,m.\] Then it
can be verified that
\[\displaystyle\lim_{\mu\searrow 0}\bar\lambda_j(x,\varepsilon,\xi,\mu)=\left\{
\begin{array}{ll}
\displaystyle\frac{1}{|I(x,\xi,\varepsilon)|} & j\in I(x,\xi,\varepsilon),\\[16pt]
0 & j \notin I(x,\xi,\varepsilon),
\end{array}
\right.\] where
$I(x,\xi,\varepsilon)=\{j:c_j(x,\xi)+\varepsilon=[c(x,\xi)+\varepsilon]^+,
0\leq j\leq m\}$. Therefore, we have from Lebesgue Dominated
Convergence Theorem that
\[\begin{array}{ll}
\displaystyle \lim_{\mu\searrow 0}\nabla_x\Psi_1(x,\varepsilon,\mu)
&=\E\left[\displaystyle\sum_{j\in
I(x,\xi,\varepsilon)}\bar\lambda_j(x,\varepsilon,\xi,\mu)
\nabla_x\left(c_j(x,\xi)+\varepsilon\right)\right]\\[20pt]
&=\E\left[\displaystyle\sum_{j\in I(x,\xi,\varepsilon)}
\displaystyle\frac{1}{|I(x,\xi,\varepsilon)|}\nabla_x\left(c_j(x,\xi)+
\varepsilon\right)\right]\in \partial g_1(x,\varepsilon).
\end{array}\]
From the definition of $\bar g_1(x,t,\mu)$, we have (a) holds.

(b). Next, we prove the following inclusion
\begin{equation}\label{eq:linc}
\displaystyle \limsup_{x'\rightarrow x,~ \mu\searrow
0}\Big\{\nabla_xH_1(x',\varepsilon,\xi,\mu)\Big\}\subset
\partial_x[c(x,\xi)+\varepsilon]^+.
\end{equation}
For any $v\in \displaystyle \limsup_{x'\rightarrow x,~ \mu\searrow
0}\Big\{\nabla_xH_1(x',\varepsilon,\xi,\mu)\Big\}$, there is a
sequence $\{(x_k,\mu_k)\}$ satisfying $x_k \rightarrow x$ and $\mu_k
\searrow 0$ such that $v =\displaystyle \lim_{k\rightarrow
\infty}\nabla_xH_1(x_k,\varepsilon,\xi,\mu_k)$. From the definition
of $I(x,\xi,\varepsilon)$, there exists some $\delta>0$ such that
\[c_j(x,\xi)+\varepsilon-[c(x,\xi)+\varepsilon]^+\leq -\delta
\mbox{ for } j \notin I(x,\xi,\varepsilon).\] Since $x'\rightarrow
c_j(x',\xi)+\varepsilon-[c(x',\xi)+\varepsilon]^+$ is continuous,
for large enough $k>0$, we have
\[c_j(x_k,\xi)+\varepsilon-[c(x_k,\xi)+\varepsilon]^+\leq -\delta/2 \mbox{
for } j\notin I(x,\xi,\varepsilon).\] It follows that
\[\begin{array}{ll}
\bar \lambda_j(x_k,\varepsilon,\xi,\mu_k) & \leq
\exp\left(-\mu_k^{-1}\delta/2\right)\rightarrow 0 \mbox{ for }
j\notin I(x,\xi,\varepsilon).
\end{array}\]
On the other hand, there are nonnegative scalars $\bar\mu_j$ for
$j\in I(x,\xi,\varepsilon)$ such that $\displaystyle\sum_{j\in
I(x,\xi,\varepsilon)}\bar\mu_j=1$ and there is a subsequence
$\{k_i\}$ satisfying
\[\bar \lambda_j(x_{k_i},\xi,\varepsilon,\mu_{k_i}) \rightarrow \bar\mu_j
\mbox{ for }j\in I(x,\xi,\varepsilon).\] Therefore,
\[\begin{array}{ll}
v=\displaystyle \lim_{k\rightarrow \infty}
\nabla_xH_1(x_k,\varepsilon,\xi,\mu_k)&=\displaystyle \lim_{i
\rightarrow
\infty}\nabla_xH_1(x_{k_i},\varepsilon,\xi,\mu_{k_i})\\[8pt]
&=\displaystyle\sum_{j\in I(x,\xi,\varepsilon)}\bar\mu_j\nabla_x
\left(c_j(x,\xi)+\varepsilon\right)\in\partial_x[c(x_k,\xi)+\varepsilon]^+
\end{array}\]
and the inclusion (\ref{eq:linc}) is proved. Noting (\ref{eq:linc}),
we have from Dominated Convergence for Selection Expectation
(Theorem 1.38 in Molchanov (2005)) that
\begin{equation}\label{eq:diff-inclusion-proof}
\begin{array}{ll}
\displaystyle\limsup_{x'\rightarrow x, \mu\searrow
0}\nabla_x\Psi_1(x,\varepsilon,\mu)&=\displaystyle
\limsup_{x'\rightarrow x, \mu\searrow 0}\E\Big[\nabla_xH_1(x',
\varepsilon,\xi,\mu)\Big]\\[8pt]
& \subset \E\Big [\displaystyle \limsup_{x'\rightarrow x, \mu
\searrow 0}\nabla_xH_1(x',\varepsilon,\xi,\mu)\Big]\\[12pt]
& \subset\E\Big[\partial_x[c(x,\xi)+\varepsilon]^+\Big] =\partial
g_1(x,\varepsilon).
\end{array}
\end{equation}
From the definition of $\bar g_i(x,t,\mu)$, we obtain
(\ref{eq:diff-inclusion}) from the inclusion
(\ref{eq:diff-inclusion-proof}).
\end{proof}

Now we can prove Theorem \ref{th:KKT}.

\begin{proof}
For any $(x,\lambda)\in \limsup_{\mu\searrow
0}\Lambda(\varepsilon,\mu)$, there exist $\{(x_k,\lambda_k)\}$ and
$\{\mu_k\}$, such that $(x_k,\lambda_k)\in
\Lambda(\varepsilon,\mu_k)$, $\mu_k\searrow 0$ and
$(x_k,\lambda_k)\rightarrow (x,\lambda)$. The inclusion
$(x_k,\lambda_k)\in \Lambda(\varepsilon,\mu_k)$ means
\begin{eqnarray}
&& -\left[\nabla_x h(x_k)+\lambda_k(\nabla_x\bar
g_1(x_k,\varepsilon,\mu_k)-\nabla_x\bar
g_2(x_k,\varepsilon,\mu_k))\right]\in N_{X}(x_k),\label{eq:k-KKT1}\\
&&\lambda_k \left[\bar g_1(x_k,\varepsilon,\mu_k)-\bar
g_2(x_k,\varepsilon,\mu_k)\right]=0,\, \lambda_k\geq 0.
\label{eq:k-KKT2}
\end{eqnarray}
It follows from Lemma \ref{lem:diffs} that, for $i=1,2$
\[\limsup_{k\rightarrow\infty}\{\nabla_x\bar
g_i(x_k,\varepsilon,\mu_k)\}\subset\partial g_i(x,\varepsilon),\]
from which there exists an subsequence $\{k_j\}$, two vectors $v_1$
and $v_2$ such that
\[\displaystyle\lim_{j\rightarrow \infty}\nabla_x\bar
g_i(x_{k_j},\varepsilon,\mu_{k_j})=v_i\in \partial
g_i(x,\varepsilon), i=1,2.\] Noting the outer semi-continuity of
$N_{X}$, letting $j\rightarrow \infty$, we have from
(\ref{eq:k-KKT1}) and (\ref{eq:k-KKT2}) that
\[\begin{array}{lll}
&& -\left[\nabla_x h(x)+\lambda(v_1-v_2)\right]\in N_{X}(x),\\[6pt]
&&v_1 \in \partial g_1(x,\varepsilon), \, v_2\in \partial
g_2(x,\varepsilon),\\[6pt]
&&\lambda\left[g_1(x,\varepsilon)-g_2(x,\varepsilon)\right]=0,\,
\lambda \geq 0,
\end{array}\]
which implies $(x,\lambda)\in \Lambda(\varepsilon)$. The proof is
completed.
\end{proof}


%\subsection{Proof of Lemma \ref{lemma:nset-convergencess}}

%\begin{proof}
%Let $z_k=(y_k,t_k)$. Note that $\displaystyle \liminf_{k
%\rightarrow +\infty} \Phi_{z_k}(\mu_k) \subset \displaystyle
%\limsup_{k \rightarrow +\infty} \Phi_{z_k}(\mu_k)$. Then it suffices
%to prove that
%\[\displaystyle \limsup_{k \rightarrow +\infty} \Phi_{z_k}(\mu_k)\subset
%\widehat \Omega_{\bar{y}}(v)\subset\displaystyle \liminf_{k
%\rightarrow +\infty} \Phi_{z_k}(\mu_k).\] We first prove that
%$\displaystyle \limsup_{k \rightarrow +\infty} \Phi_{z_k}(\mu_k)
%\subset \widehat \Omega_{\bar{y}}(v)$. Let $(y,s) \in \displaystyle
%\limsup_{k \rightarrow +\infty} \Phi_{z_k}(\mu_k)$, then there is a
%sequence $\{k_j\} \subset \textbf{N}$ such that
%$(y,s)=\displaystyle\lim_{j \rightarrow \infty}(y'_j,t'_j)$ for some
%$z'_j:=(y'_j,t'_j) \in \Phi_{z_{k_j}}(\mu_{k_j})$. We have $z'_j \in
%\widehat  X$ and $ \bar g_1(z'_j,\mu_{k_j})-[\bar
%g_2(z_{k_j},\mu_{k_j})+\langle \nabla_z\bar g_2(z_{k_j},\mu_{k_j}),
%(z'_j-z_{k_j})\rangle] \leq 0$. Letting $j \rightarrow + \infty$, we
%have $(y,s) \in \widehat X$ and $\widehat g_1(y,s)-[g_2(\bar{y})+
%\langle v, y-\bar{y} \rangle ] \leq 0$, which implies $(y,s) \in
%\widehat \Omega_{\bar{y}}(v)$. Therefore, $\displaystyle \limsup_{k
%\rightarrow +\infty} \Phi_{z_k} (\mu_k)\subset \widehat
%\Omega_{\bar{y}}(v)$.

%We then prove that $\widehat\Omega_{\bar{y}}(v)\subset \displaystyle
%\liminf_{k \rightarrow +\infty}\Phi_{z_k}(\mu_k)$. Let $(y,s)\in
%\widehat\Omega_{\bar{y}}(v)$, we only need to prove that there
%exists $\{(y'_j,t'_j)\}$ such that $(y'_j,t'_j)\in\Phi_{z_j}
%(\mu_j)$ for $j$ large enough and $(y'_j,t'_j) \to (y,s)$.

%For any $z'=(y',t'),z''=(y'',t'') \in \widehat X$, let
%$z_\mu=z''+\mu (z'-z'')$ where $\mu\in [0,1]$.  Since
%\[\displaystyle\lim_{k\rightarrow \infty}\Big\{\displaystyle
%\sum_{i=0}^m \bar \lambda_j(z_\mu,\xi)\nabla_z \bar
%c_j(z_\mu,\xi)-\displaystyle \frac{1}{|I(z_\mu,\xi)|}\displaystyle
%\sum_{i\in I(z_\mu,\xi)} \nabla_z \bar c_j(z_\mu,\xi)\Big\}=0,\] one
%has from Dominated Convergence Theorem that for $k>0$ large enough
%\begin{equation}\label{eq:bond22}
%\left\|\displaystyle\int_0^1\E\Big\{\displaystyle \sum_{i=0}^m \bar
%\lambda_j(z_\mu,\xi)\nabla_z \bar c_j(z_\mu,\xi)-\displaystyle
%\frac{1}{|I(z_\mu,\xi)|}\displaystyle \sum_{i\in I(z_\mu,\xi)}
%\nabla_z \bar c_j(z_\mu,\xi)\Big\}d\mu\right\|\leq \gamma/8.
%\end{equation}
%As $\nabla_x \bar g_2(y_k,t_k,\mu_k)\rightarrow v$, when $k$ is
%large enough, we have
%\begin{equation}\label{eq:v-ineq}
%\|\nabla_x \bar g_2(y_k,t_k,\mu_k)-v\|\leq \gamma/8.
%\end{equation}
%Since $\widehat g_1(z)$ is convex, we have from the mean-value
%theorem for convex functions, see for example Hiriart-Urruty and
%Lemar\'{e}chal(1993), that
%\begin{equation}\label{eq:mean-value-th}
%\begin{array}{ll}
%\widehat g_1(z')-\widehat g_1(z'') & =\displaystyle \int_0^1 \langle
%\partial g_1(z_\mu), z'-z''\rangle d\mu\\[8pt]
%& =\left\langle \displaystyle \int_0^1 \E\Big\{\displaystyle
%\frac{1}{|I(z_\mu,\xi)|} \displaystyle \sum_{i \in I(z_\mu,\xi)}
%\nabla_z \bar c_j(z_\mu,\xi)\Big\}d\mu,z'-z'' \right\rangle.
%\end{array}
%\end{equation}
%Define $D^k(x,t)=\bar g_{z_k}(x,t,\mu_k)-\widehat g_{\bar
%y,v}(x,t)$. Then for $z'=(y',t'),z''=(y'',t'')\in \widehat X$,
%\[\begin{array}{ll}
%D^k(z')-D^k(z'') &=\bar g_1(z',\mu_k)-\bar g_1(z'',\mu_k)-\langle
%\nabla \bar g_2(z_k,\mu_k),z'-z''
%\rangle\\[6pt]
%&-[\widehat g_1(z')-\widehat g_1(z'')]+\langle v, y'-y''\rangle\\[6pt]
%&=[\bar g_1(z',\mu_k)-\bar g_1(z'',\mu_k)]-[\widehat g_1
%(z')-\widehat g_1 (z'')]-\langle \nabla_x \bar g_2(y_k,t_k\mu_k)-v,
%y'-y''\rangle.
%\end{array}\]
%From the mean-value theorem, we have
%\[\begin{array}{ll}
%\bar g_1(z',\mu_k)-\bar g_1(z'',\mu_k) & =\displaystyle \int_0^1
%\langle \nabla_z g_1(z_\mu,\mu_k), z'-z''\rangle d\mu\\[6pt]
%&=\left \langle \displaystyle \int_0^1\E\Big\{\displaystyle
%\sum_{i=0}^m \bar \lambda_j(z_\mu,\xi)\nabla_z \bar
%c_j(z_\mu,\xi)\Big\}d\mu,z'-z''\right\rangle,
%\end{array}\]
%together with (\ref{eq:mean-value-th}), (\ref{eq:bond22}) and
%(\ref{eq:v-ineq}), one has for large $k$ that
%\[\begin{array}{l}
%|D^k(z')-D^k(z'')|\\
%\quad \quad =\left \langle \displaystyle
%\int_0^1\E\Big\{\displaystyle \sum_{i=0}^m \bar
%\lambda_j(z_\mu,\xi)\nabla_z \bar c_j(z_\mu,\xi)-\displaystyle
%\frac{1}{|I(z_\mu,\xi)|} \displaystyle \sum_{i \in I(z_\mu,\xi)}
%\nabla_z \bar c_j(z_\mu,\xi)\Big\}d\mu, z'-z''\right\rangle\\[8pt]
%\quad \quad \quad \, -\langle \nabla_x \bar
%g_2(y_k,t_k\mu_k)-v, y'-y''\rangle\\[8pt]
%\quad \quad \leq \left \|\displaystyle \int_0^1\E\Big\{\displaystyle
%\sum_{i=0}^m \bar \lambda_j(z_\mu,\xi)\nabla_z \bar
%c_j(z_\mu,\xi)-\displaystyle \frac{1}{|I(z_\mu,\xi)|}\displaystyle
%\sum_{i\in I(z_\mu,\xi)}\nabla_z \bar c_j(z_\mu,\xi)\Big\}
%d\mu\right\|\|z'-z''\|\\[8pt]
%\quad \quad \, +\|g_2(y_k,t_k\mu_k)-v\|\|z'-z''\|\\[8pt]
%\quad \quad \leq \displaystyle \frac{\gamma}{4}\|z'-z''\|.
%\end{array}\]
%Then $D^k$ is Lipschitz continuous  in $\widehat X$ with constant
%smaller than $\gamma/4$ so that $\partial D^k(x,t)\subset
%\displaystyle \frac{\gamma}{4} \textbf{B}_{n+1}$ for any $(x,t) \in
%\widehat X$, where $\textbf{B}_{n+1} \subset {\bf R}^{n+1}$ is the
%unite ball. We can rewrite $\bar g_{z_k}(x,t,\mu_k)=\widehat g_{\bar
%y,v}(x,t)+D^k(x,t)$ which implies that
%\begin{equation}\label{eq:nest-1}
%\partial \bar g_{z_k}(x,t,\mu_k) \subset  \partial
%\widehat g_{\bar y,v}(x,t)+\displaystyle \frac{\gamma}{4}
%\textbf{B}_{n+1}.
%\end{equation}
%Since Condition (\ref{eq:nstrong-CQ}) holds for $\widehat
%\Omega_{\bar{y}}(v)$, we have
%\[
%\|u\| \geq \gamma, \forall u \in \partial \widehat g_{\bar
%y,v}(x,t), \forall (x,t) \in \widehat \Omega_{\bar{y}}(v).
%\]
%It follows from the outer semi-continuity of $\partial \widehat
%g_{\bar y,v}(\cdot)$, for any $(x,t) \in \widehat
%\Omega_{\bar{y}}(v)$, there exists an positive number $\mu_{x,t}
%>0$ such that
%\begin{equation}\label{eq:nlower-1}
%\|u\| \geq \displaystyle \frac{\gamma}{2}, \forall u \in \partial
%\widehat g_{\bar y,v}(x',t'), \forall (x',t')  \in {\rm
%int}\textbf{B}_{\mu_{x,t}}(x,t).
%\end{equation}
%(If $(x',t') \notin \widehat X$, $\partial \widehat g_{\bar
%y,v}(x',t')=\emptyset$ and (\ref{eq:nlower-1}) holds trivially) As
%$\widehat \Omega_{\bar{y}}(v)$ ($\subset \widehat X$) is a compact
%set, $\Big\{{\rm int}\textbf{B}_{\mu_{x,t}/2}(x,t): (x,t) \in
%\widehat \Omega_{\bar{y}}(v)\Big\}$ forms an open covering of
%$\widehat \Omega_{\bar{y}}(v)$, there exists a finite number of
%points in $\widehat \Omega_{\bar{y}}(v)$, say $(x^1,t^1), \ldots,
%(x^K,t^K)$ for some integer $K>0$ such that
%\[\widehat \Omega_{\bar{y}}(v) \subset \bigcup_{i=1}^K{\rm
%int}\textbf{B}_{\mu_{x^i,t^i}/2}(x^i,t^i).\] Let
%\[\mu=\displaystyle \min_{1 \leq i \leq K} \frac{\mu_{x^i,t^i}}{4}.\]
%Then from the outer semi-continuity of $\Phi_{z_k}(\mu_k)$, when $k$
%is sufficiently large,
%\[\Phi_{z_k}(\mu_k) \subset \widehat \Omega_{\bar{y}}(v)+\mu
%\textbf{B}_{n+1}.\] Therefore, when $k$ is large enough,
%\begin{equation}\label{eq:nlower-2}
%\|u\| \geq \displaystyle \frac{\gamma}{2}, \forall u \in \partial
%\widehat g_{\bar y,v}(x',t'), \forall (x',t')  \in
%\Phi_{z_k}(\mu_k).
%\end{equation}
%Combining (\ref{eq:nest-1}) and (\ref{eq:nlower-2}), we have for
%sufficiently large $k$ that
%\begin{equation}\label{eq:nstrong-CQk}
%\displaystyle \inf_{(x,t) \in {\rm bd }[\Phi_{z_k}(\mu_k)]}
%\,\displaystyle \min_{u \in
%\partial \bar g_{z_k}(x,t,\mu_k)} \|u\| \geq \displaystyle \frac{\gamma}{4}.
%\end{equation}
%It comes from (\ref{eq:nmetric-V}) that
%\begin{equation}\label{eq:nmetric-Vk}
%{\rm dist }((x,t), \Phi_{z_k}(\mu_k)) \leq \displaystyle
%\frac{4}{\gamma}[\bar g_{z_k}(x,t,\mu_k)]^+, \forall (x,t)\in \Re^n
%\times \Re
%\end{equation}
%for $k$ large enough.

%Therefore, for large $k$, there exists $(y'_k,t'_k) \in \widehat
%\Omega_{y_k}(v_k)$ such that
%\begin{equation}\label{eq:nest-21s}
%\|(y'_k,t'_k)-(y,s)\| \leq \displaystyle \frac{4}{\gamma}[\bar
%g_{z_k}(y,s,\mu_k)]^+=\displaystyle \frac{4}{\gamma}[\widehat
%g_{\bar y,v}(y,s)+D^k(y,s)]^+.
%\end{equation}
%Let $z=(y,s)$. Then
%\[\begin{array}{ll}
%D^k(z) & =\bar g_{z_k} (z,\mu_k)-\widehat g_{\bar y,v}(z)\\[6pt]
%&=(\bar g_1(z,\mu_k)-\widehat g_1 (z))-(\bar c_2
%(z_k,\mu_k)-\mu_k \log (m+1)-\widehat g_2(z_k))\\[6pt]
%& \quad \quad -\Big(\langle \nabla_x \bar  g_2(z_k,\mu_k),
%y-y_k\rangle-\langle v, y-\bar y \rangle\Big)\\[6pt]
%&=(\bar c_1(z,\mu_k)-g_1 (z))-(\bar c_2
%(z_k,\mu_k)-\mu_k \log (m+1)-\widehat g_2(z_k))\\[6pt]
%& \quad \quad -\Big(\langle \nabla_x \bar g_2(z_k,\mu_k), \bar
%y-y_k\rangle +\langle \nabla_x \bar g_2(z_k,\mu_k)-v, y-\bar y
%\rangle \Big)
%\end{array}\]
%and for sufficiently large $k$ such that $\|\nabla_x \bar
%g_2(z_k,\mu_k)\|\leq 2\|v\|$, we have that
%\[\begin{array}{ll}
%|D^k(z)| & \leq |\bar c_1(z,\mu_k)-g_1 (z)|+|\bar c_2
%(z_k,\mu_k)-\mu_k \log (m+1)-\widehat g_2(z_k)|\\[6pt]
%& \quad \quad +\|\nabla_x \bar g_2(z_k,\mu_k)\|\|\bar
%y-y_k\|+\|\nabla_x \bar
%g_2(z_k,\mu_k)-v\|\|y-\bar y\|\\[6pt]
%&\leq 3 \mu_k \log (m+1)+|\widehat g_2(z_k)-\widehat g_2(z)|+2M_0
%\|\nabla_x \bar g_2(z_k,\mu_k)-v\|+2\|v\|\|y_k-\bar y\|,
%\end{array}\]
%where $\bar M_0=\max \{\|y\|: y\in X\}$. Therefore, we have from
%(\ref{eq:nest-21s}) that
%\[\begin{array}{ll}
%\|(y'_k,t'_k)-(y,s)\|  & \leq \displaystyle \frac{4}{\gamma}\left
%[[\widehat g_{\bar y,v}(y,s)]^+ +|D^k(y,s)| \right ]\\[8pt]
%&\leq \displaystyle \frac{4}{\gamma} \Big[3\mu_k\log
%(m+1)+|\widehat g_2(z_k)-\widehat g_2(z)|\\[6pt]
%&\quad \quad \quad \quad +2M_0\|\nabla_x \bar
%g_2(z_k,\mu_k)-v\|+2\|v\|\|y_k-\bar y\|\Big],
%\end{array}\]
%which implies $(y'_k,t'_k) \rightarrow (y,s)$ as $k \rightarrow
%\infty$. Hence $\widehat \Omega_{\bar{y}}(v)\subset \displaystyle
%\liminf_{k\rightarrow +\infty}\Phi_{z_k}(\mu_k)$. This concludes the
%proof of the lemma.
%\end{proof}


\section{Proof of Theorem \ref{property-nKKTss}}

\begin{proof}
It follows from Theorem 5 of Hong et al. (2011) that the results
holds.
\end{proof}


%\subsection{Convergence of Non-smooth $\varepsilon$-Approximation}

%It is obvious that
%\[\partial\widehat g_1(x,t)=\left\{u+
%\left(
%\begin{array}{l}
%0\\[6pt]
%-\alpha
%\end{array}
%\right): u \in \partial g_1(x,t)\right\}\] and it follows from
%Theorem 7.47 of Shapiro et al. (2009) that
%\begin{equation}\label{eq:nablas}
%\partial g_1(x,t)=\E[\partial_{x,t} \widehat c(x,\xi,t)],\,\,
%\partial \widehat g_2(x,t)=\E[\partial_{x,t}\widehat c_0
%(x,\xi,t)], \partial g_2(x)=\E[\partial_x \widehat c_0(x,\xi,t)]
%\end{equation}
%\begin{definition}\label{def:stat}
%We say $(x^*,t^*) \in \widehat{X}$ is a stationary point of Problem
%\mbox{$({\rm P}_{\mu})$} if
%\begin{equation}\label{eq:stationarity}
%0 \in \nabla h(x^*,t^*)+[\partial \widehat g_1(x^*,t^*)-\partial
%\widehat g_2(x^*,t^*)]^TN_{\Re_-}(g(x^*,t^*))+N_{\widehat
%X}(x^*,t^*).\end{equation}
%\end{definition}
%Note that if $\widehat g_1$ or $\widehat g_2$ is differentiable at
%$(x^*,t^*)$, then Condition \ref{eq:stationarity} is reduced to
%Lagrange Multiplier Rule for optimization of Lipschitz functions in
%Clarke (1983). Problem \mbox{$({\rm P}_{\mu})$} is essentially a
%quasi-differentiable optimization problem in the sense of Demyanov
%and Rubinov (1980), we can use the methodology in Shapiro (1984) to
%investigate its optimality conditions. However, the stationary
%condition \ref{eq:stationarity} is suitable for analyzing
%convergence properties of numerical algorithms.

\chapter{Conclusions and Future Work}

In this thesis, after explaining the mismatch of existing simulation R\&S procedures and current parallel computing technology, we proposed our solution, a general framework for parallel simulation R\&S procedures, which can be deployed on cluster and is extensible to both customized simulations and customized R\&S procedures. On top of that, specific simulations, like $(s, S)$ inventory management or three-stage allocation, and original or revised R\&S procedures, including classic Rinott's procedure, asynchronous Rinott's procedure, vector filling procedure and asymptotic parallel sequential procedure are developed. Numerical experiments showed that the parallel ratio of the whole system is pretty high, which guarantees the scalability.

However, there is still work left to do in the future, except from a thread-safe discrete-event library and more rigorous algorithm analysis with heap adjustment probability already mentioned in thesis, we have at list the followings:

Firstly, we have adopted heap to decrease the complexity of processing single new sample value from $O(n)$ to $O(\log{n})$, however it's still not enough, because the simulations can be much faster by scaling out. Batched processing is a choice, but potentially losing change of eliminating earlier. Scaling out simulation R\&S procedures by divide-and-conquer has been tried in \cite{potwsc05ras}, but it also sacrifices the change of earlier elimination with passing the "best" alternative" around.

Secondly, communicating through two FIFO queues is easy to archive load balancing for the whole system, but too weak in scheduling the sequence of simulation experiments. A certain simulation experiment may bring more probability for eliminating inferior alternatives, thus should be carried out with higher priority. But involving an explicit scheduler will take the risk of load-unbalancing.

Thirdly, as there exist so many different simulation R\&S procedures, it is necessary to compare their performance in parallel computing environment empirically. Although previous work like \cite{ms05ras} has already made an intensive comparative study, they only work on the serial computing environment. We have conducted some experiments in this thesis but still no enough.

Finally, the most exciting vision here is to make it the stand benchmark for all the previous and future R\&S procedures by providing this implement as a service in the form of cloud computing, or more precisely, in the form of platform as a service(PAAS), which could be based on infrastructure as a service(IAAS), like Amazon Elastic Compute Cloud(EC2), where more practical issues like pricing also need to be taken consideration.

%function: OvS support
%Extend to Optimization via Simulation, an interface of sim(SimInput[]) should be enough.

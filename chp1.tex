\chapter{Introduction}\label{chp1}


\section{Background and Motivation}\label{sec1.1}

\subsection{Statistical Process Control}

Quality has become one of the most important factors that decides selection of
products and services. Therefore, improving quality is key to business success,
growth, and competitiveness. Traditionally, quality means fitness for use. This
definition is based on the viewpoint that products and services must meet consumers'
requirements. Extending from it, quality can have eight dimensions Garvin (1987),
including performance, reliability, durability, serviceability, aesthetics,
features, perceived quality and conformance to standards. In fact, everyone has a
conceptual understanding of quality that relates to one or more characteristics a
product or service should possess.

With the rapid development of technology, actually we prefer a modern definition of
quality, which is that quality is inversely proportional to variability. Based on
it, quality improvement is the reduction of variability in processes and products
Montgomery (2009). Since variability can only be characterized in statistical terms,
statistical methods play a critical rule in improving quality. Statistical process
control (SPC) is the application of statistical methods to the monitoring and
control of a process to ensure that it operates at its full potential to produce
conforming products.

In general, there are two types of variability in any production process. The first
comes from chance causes of variation. Basically, it is the cumulative effect of
many small unavoidable causes. Such variability is inherent or natural and always
exists, no matter how well designed or carefully maintained the process is. If a
process involves only chance causes of variation, it is said to be in (statistcal)
control (IC). The second is related to assignable causes of variation, which usually
results from improper operations or defective raw materials, etc. Generally, such
variability is large compared to the first type and usually indicates an
unacceptable level of process performance. If a process is operating with assignable
causes of variation, it is regarded as out-of-control (OC).

Statistical process control mainly aims at detecting the assignable causes of
variation as quickly as possible, so that corrective actions may be undertaken to
avoid many nonconforming products. SPC usually consists of two phases. In Phase I, a
process reference dataset is collected and examined to see if any unusual patterns
exist. After they are identified and adjusted, a clean dataset results. This dataset
is called the IC dataset and is used for estimating the IC model, which
characterizes the process under IC operating conditions. The performance of Phase I
analysis is often measured by the statistical power of detecting these
abnormalities. After this off-line retrospective analysis, quality inspectors will
have a good reference IC model.

In Phase II, the main task is to construct control charts for monitoring the process
and detecting any shifts from the IC state. The performance of control charts in
Phase II is usually measured by the average run length (ARL), which is defined as
the average number of samples needed for the control chart to trigger an OC signal.
The control chart is an online technique, composed of control limits and charting
statistics. In particular, a charing statistic is calculated based on a sample and
plotted against this sample. A charting statistic falling beyond the control limits
indicates a shift to OC states in the process. The IC ARL is usually fixed to be 370
by selecting suitable control limits, which is a convention in SPC. As a result, for
a shift, a control chart with a smaller OC ARL needs fewer samples to trigger an OC
signal and therefore detects this shift more quickly.



\subsection{Multivariate Categorical Processes}

The overall quality of most modern processes is best described in terms of multiple
characteristics. Multivariate statistical process control is applied in practical
situations where several quality characteristics must be monitored simultaneously.
For instance, tensile strength and diameter, which have been assumed to follow a
bivariate normal distribution, are two important quality characteristics of a
textile fiber that must be jointly controlled.

In manufacturing and especially service industries, however, it is increasingly
common to have quality characteristics that cannot be measured numerically. Although
obtaining their precise continuous values may be expensive, unnecessary, or even
impossible, collecting some attribute data related to them may be quite feasible.
Such rough classification levels do not require precise measurements. Examples
include items on a production line whose several quality characteristics are each
evaluated as conforming or nonconforming according to predefined specifications and
multiple indexes in a service flow that may be assessed as excellent, acceptable, or
unacceptable. The characteristics all have two or more attribute levels, so that the
data are multivariate categorical. Note that this is similar to having multiple
factors in design of experiments (DOE), where each factor has several specific
levels. Here, for simplicity, we use ``factor'' to designate a categorical
characteristic.

We illustrate multivariate categorical processes with two motivating examples.
Aluminium electrolytic capacitor (AEC) manufacturing is a multi-stage process,
comprising a series of stages, from cutting, winding, drying, impregnating, and
assembling, to sleeving, washing, aging, and packaging. The production line
manufactures AECs from various raw materials, including anode aluminum foil, cathode
aluminum foil, guiding pin, electrolyte sheet, plastic cover, aluminum shell, and
plastic tube. Right after each stage, the quality of the stagewise AEC products,
namely capacitor elements in terms of appearance condition and functional
performance, will be inspected by sampling in order to meet the specifications. Here
we concentrate on the quality after the aging stage, which is assessed mainly in
terms of leakage current (LC), dissipation factor (DF), and capacity (CAP). Each
characteristic is classified automatically as conforming or nonconforming to the
specifications by an electronic device at a very high speed, and engineers are
reluctant to obtain their precise continuous or numerical values (which is costly
but not impossible). Consequently, this is a multivariate categorical process with
three factors LC, DF and CAP, each with two levels and therefore $2^3=8$ level
combinations. Without loss of generality, for each factor, $-1$ represents
``conforming'' and 1 ``nonconforming''. For example, the combination ($-1$, 1, $-1$)
means an AEC with conforming LC and CAP and nonconforming DF.

Another example is the quality control of welding rods. One of the key aspects of
welding rod inspection is their appearance, which directly reflects the integrated
level of welding rod manufacturing and influences the welding performance. The
welding rod is composed of a cylindrical metallic core wire and a coating
composition (flux) covering the circumference of the metallic core wire. Its
appearance has some important characteristics such as eccentricity of the core wire,
moisture resistance of the coating, strength of the coating, and rod bend. During
testing, each is simply evaluated as either conforming or nonconforming, and their
(latent) continuous values are not considered. With the four factors (eccentricity,
moisture resistance, strength, and bend), each with two attribute levels, this is
also a multivariate categorical process with $2^4=16$ cross-classification level
combinations.

Consider now a general multivariate categorical process. Suppose that there are $p$
factors $C_1,\ldots,C_p$, and that each classification factor $C_i$ has $h_i$
possible levels. The overall cross-classifications among all the level combinations
of these factors form a $p$-way $h_1\times\ldots\times h_p$ contingency table with
$h=\prod_{i=1}^p h_i$ cells. Each cell corresponds to one level combination of the
$p$ factors and stores the count under this level combination.

For a simple $h_1\times h_2\times h_3$ three-way table, denote the observed count by
$n_{ijk}$ in cell$(i,j,k)$ ($i=1,\ldots,h_1$; $j=1,\ldots,h_2$; $k=1,\ldots,h_3$)
and its expectation by $m_{ijk}$. If the observations are made over a period of
time, it is reasonable to assume that each cell count follows an independent Poisson
distribution (see Bishop et al. (2007)). During the Phase II SPC monitoring process,
the total sum of observations is usually fixed. Conditional on the total sum of cell
counts, a series of independent Poisson distributions result in a multinomial
distribution. So in one sample of size $N$, the cell counts in the three-way
contingency table therefore jointly follow the multinomial distribution
$\mbox{MN}(N;p_{ijk})$ ($i=1,\ldots,h_1$; $j=1,\ldots,h_2$; $k=1,\ldots,h_3$). Here
$p_{ijk}=m_{ijk}/N$ is the probability of an observation falling into cell$(i,j,k)$,
and these probabilities must sum to 1.

To generalize the three-way table to a general $p$-way $h_1\times \ldots\times h_p$
contingency table (see Johnson et al. (1997)), let the probability of obtaining the
combination of the factor levels $a_1,\ldots,a_p$ be $p_{a_1\ldots a_p}$
($a_i=1,\ldots,h_i$ and $i=1,\ldots,p$). Furthermore, denote the count of
observations with the level combination $a_1\ldots a_p$ among a sample of size $N$
by $n_{a_1\ldots a_p}$. The marginal counts for factor $C_i$, which are denoted by
$n_{(i)1},\ldots,n_{(i)h_i}$ and calculated as
\[
n_{(i)v}=\sum_{a_1}\ldots\sum_{a_{i-1}}\sum_{a_{i+1}}\ldots \sum_{a_p}n_{a_1\ldots
a_{i-1}va_{i+1}\ldots a_p},\quad v=1,\ldots,h_i,
\]
will follow the multinomial distribution $\mbox{MN}(N; p_{(i)1},\ldots,p_{(i)h_i})$,
where
\[
p_{(i)v}=\sum_{a_1}\ldots\sum_{a_{i-1}}\sum_{a_{i+1}}\ldots \sum_{a_p}p_{a_1\ldots
a_{i-1}va_{i+1}\ldots a_p},\quad v=1,\ldots,h_i.
\]
Here, $n_{(i)v}$ is actually the sum of the cell counts $n_{a_1\ldots a_p}$ over all
levels of all factors other than $C_i$ and for $C_i$ at level $v$. Also,
$p_{(i)1},\ldots,p_{(i)h_i}$ are the marginal probabilities of factor $C_i$, which
can be calculated in a similar way to $n_{(i)1},\ldots,n_{(i)h_i}$. Thus, consider
the joint distribution of the $p$ sets of variables
\[
n_{(i)1},\ldots,n_{(i)h_i},\quad i=1,\ldots,p,
\]
each being a multinomial distribution. This joint distribution is a multivariate
multinomial distribution (Johnson et al. (1997)), $\mbox{MMN}(N;{\bf p})$, where
$\bf p$ is the $h$-variate cell probability vector composed of $p_{a_1\ldots a_p}$
($a_i=1,\ldots,h_i$ and $i=1,\ldots,p$). When each factor has two levels, it reduces
naturally to a multivariate binomial distribution. Therefore, based on the framework
of multivariate binomial and multivariate multinomial distributions, multiple
categorical factors can be studied.



\section{Literature Review}\label{sec1.2}

\subsection{Overview}

Much effort has been devoted to the monitoring problem in settings where all the
observed variables are numerical and continuous. The most famous control chart is
the Hotelling's $T^2$ chart (see Montgomery (2009)), which assume process data come
from a multivariate normal distribution. Refer to Lowry and Montgomery (1995) and
Bersimis et al. (2007) for thorough reviews of monitoring multivariate continuous
processes. Recent developments introduce variable selection into multivariate
continuous SPC, such as forward selection procedures in Wang and Jiang (2009) and
Wang et al. (2012), as well as least absolute shrinkage and selection operator
(LASSO) (see Tibshirani (1996)) in Zou and Qiu (2009) and Zou et al. (2011). Some
data mining approaches are also employed for monitoring multivariate continuous
processes, such as kernel-distance-based methods. Related work include Sun and Tsung
(2003) and Ning and Tsung (2012). In addition, some multivariate nonparametric
control charts are proposed, including the spatial-sign-based multivariate chart
(Zou and Tsung (2011)) and the spatial-rank-based multivariate chart (Zou et al.
(2012)). These charts are affine-invariant, have a small-sample distribution-free
property over a broad class of distributions, and perform very well in comparison
with Hotelling's $T^2$ chart and other multivariate nonparametric charts on
non-normal distributions.

For categorical or attribute data, the $p$-chart and the $np$-chart for binomial
distributed variables, together with the $c$-chart and the $u$-chart for Poisson
processes, are typical SPC tools used with univariate categorical processes. To
monitor multiple factors, we may employ multiple univariate categorical control
charts as a multi-chart (see Woodall and Ncube (1985)). However, this may be an
unattractive approach for two reasons: First, it is difficult to determine the
control limit to achieve a desired ARL. The control limits of the separate
univariate charts must be set such that each chart achieves a specific individual IC
ARL, so that the overall IC ARL of the multi-chart is the desired value. Determining
these control limits is nontrivial even for the low-dimensional case of a small
number of categorical factors, let alone for a general multivariate categorical
distribution. This complexity also increases dramatically when the marginal
distributions of categorical factors are not identical, since the individual charts
have different run length distributions. Second, a multi-chart considers individual
categorical factors in parallel and therefore is unable to account for any
correlations among them. Naturally, it is desirable to introduce multivariate
categorical control charts that can appropriately describe and exploit the
relationships among multiple categorical factors.

Woodall (1997) summarized many aspects of the control charts for attribute data, but
mostly considered univariate categorical charts. In the literature, some effort has
been devoted to monitoring multinomial and multi-attribute processes. See Topalidou
and Psarakis (2009) for an overview. Among others, Patel (1973) suggested a
$\chi^2$-chart for multivariate binomial populations, which is based on the
assumption that the joint distribution of the correlated binomial variables can be
approximated by a multivariate normal distribution given a sufficiently large sample
size. More recent developments include the Shewhart-type $mnp$-chart proposed by Lu
et al. (1998), which uses the weighted sum of the number of nonconforming units for
each quality characteristic. There is also the $mp$-chart designed by Chiu and Kuo
(2008) for multivariate Poisson count data. However, as with Patel's $\chi^2$-chart,
these two charts can only deal with factors having two levels, i.e., multivariate
binomial processes. On the other hand, some methods focus on monitoring multinomial
processes that have only one factor with three or more levels, such as the
generalized $p$-chart developed by Marcucci (1985). This chart extends the
traditional $p$-chart by adopting the Pearson chi-square statistic. The multinomial
cumulative sum (CUSUM) chart proposed by Ryan et al. (2011) can also deal with this
problem and is based on the likelihood ratio statistic equipped with a CUSUM scheme.
Note that the prior approaches for monitoring multinomial processes are actually
univariate charts, since only one factor is involved.

To this end, we see that with several factors, at least one of them having more than
two levels (i.e., multivariate multinomial processes), there are no appropriate
approaches available. In addition to this deficiency in monitoring multivariate
multinomial processes, most of the existing methods focus entirely on the marginal
sums with respect to each categorical factor, neglecting the cross-classifications
between factors. Because of this, if some cross-classification probabilities shift
to OC states, these charts may not detect them quickly. Therefore, a general
monitoring methodology for multivariate categorical processes is required.

Control charts are for monitoring online processes in Phase II SPC. It should be
noted that the model, based on which a control chart is constructed, is assumed to
represent the IC operating condition and have been estimated from an IC reference
dataset. However, in reality this is not always the case, and any unusual patterns
in this dataset could yield an erroneous IC model estimate. Among others, a
change-point is a sustained special cause that remains until some corrective action
is taken, and an outlier is an isolated special cause that affects a single sample
and then disappears (Hawkins and Qiu (2003)). We must ensure that we have an IC
dataset for estimating the parameters for constructing control charts. This leads to
Phase I analysis, in which the main objective is to identify unusual patterns such
as change-points or outliers and bring the reference dataset into a state of
statistical control. As for multivariate categorical processes, Phase I analysis for
them still remains a challenge and has not been investigated to the best of our
knowledge. Methods for Phase I analysis under the umbrella of multivariate binomial
and multivariate multinomial processes are called for.



\subsection{Some Typical Conventional Approaches}

In this section, we review some typical methods for monitoring multivariate binomial
and multivariate multinomial processes. Hereafter we use the superscripts ``(0)''
and ``(1)'' to denote the IC and OC states, respectively.

In a multivariate binomial process, each factor has two levels, giving rise to a
$p$-way contingency table with $2^p$ cells for $p$ factors. Denote the two levels of
each factor by 1 and 2 and the IC probability of factor $C_i$ taking Level 1 as
$p_{(i)}^{(0)}$ $(i=1,\ldots,p)$, which is known and calculated as
\[
p_{(i)}^{(0)}=\sum_{a_1}\ldots\sum_{a_{i-1}}\sum_{a_{i+1}}\ldots
\sum_{a_p}p_{a_1\ldots a_{i-1}1a_{i+1}\ldots a_p}^{(0)}.
\]
For the $k$th sample of size $N$ in Phase II, if the process is IC, the Level 1
count $n_{(i)k}$ of factor $C_i$ ($i=1,\ldots,p$), which equals
\[
n_{(i)k}=\sum_{a_1}\ldots\sum_{a_{i-1}}\sum_{a_{i+1}}\ldots \sum_{a_p}n_{a_1\ldots
a_{i-1}1a_{i+1}\ldots a_p,k},
\]
is binomially distributed with the total size $N$ and its IC Level 1 probability
$p_{(i)}^{(0)}$. Let ${\bf n}_{\mathrm{MB},k}=\big[n_{(1)k},\ldots,n_{(p)k}\big]^T$
and ${\bf p}^{(0)}_{\mathrm{MB}}=\big[p_{(1)}^{(0)},\ldots,p_{(p)}^{(0)}\big]^T$.
Based on these, Patel (1973) constructed the $\chi^2$ charting statistic of the
Hotelling's $T^2$ form
\begin{equation}
G_{\mathrm{MB},k}=\frac{1}{N}\big({\bf n}_{\mathrm{MB},k}-N{\bf
p}^{(0)}_{\mathrm{MB}}\big)^T\bm{\Sigma}^{-1}_{\mathrm{MB}}\big({\bf
n}_{\mathrm{MB},k}-N{\bf p}^{(0)}_{\mathrm{MB}}\big).\label{F1.1}
\end{equation}
Here, the covariance matrix term $\bm{\Sigma}_{\mathrm{MB}}$ has the elements
\begin{align*}
\bm{\Sigma}_{\mathrm{MB}\;ij}&=\left\{\begin{array}{ll}p^{(0)}_{(i)}\big(1-p^{(0)}_{(i)}\big)&\textrm{if}\
i=j\\p^{(0)}_{(ij)}-p^{(0)}_{(i)}p^{(0)}_{(j)}&\textrm{if}\ i\ne
j\end{array}\right..
\end{align*}
Here $p^{(0)}_{(ij)}$ is the IC probability of factors $C_i$ and $C_j$ both taking
Level 1, which is also known and calculated as
\[
p_{(ij)}^{(0)}=\sum_{a_1}\ldots\sum_{a_{i-1}}\sum_{a_{i+1}}
\ldots\sum_{a_{j-1}}\sum_{a_{j+1}}\ldots \sum_{a_p}p_{a_1\ldots
a_{i-1}1a_{i+1}\ldots a_{j-1}1a_{j+1}\ldots a_p}^{(0)}
\]
by assuming $i<j$ without loss of generality.

Factors with three or more levels are also common in production and service
applications, and they can be treated as multivariate multinomial processes.
Consider customer attitudes towards a service for instance. Suppose that there are
four indexes, each of which may take the values of excellent, acceptable, or
unacceptable. This forms a four-way contingency table with $3^4$ cells. No
appropriate monitoring methods exist that incorporate the cross-classifications
among factors when at least one of them has more than two attribute levels.

The only feasible existing approach of monitoring multivariate multinomial processes
with $p$ factors, albeit a naive one, might be to monitor the $p$ groups of marginal
sums of each factor using $p$ individual charts. If we consider only the group of
marginal sums of factor $C_i$ $(i=1,\ldots,p)$, it is a multinomial process, which
could be handled by applying Marcucci's (1985) generalized $p$-chart. Denote the IC
probability of factor $C_i$ taking Level $v$ ($v=1,\ldots,h_i$) by
$p^{(0)}_{(i,v)}$, which is known as
\[
p^{(0)}_{(i,v)}=\sum_{a_1}\sum_{a_2}\ldots\sum_{a_{i-1}}\sum_{a_{i+1}}\ldots
\sum_{a_{p-1}}\sum_{a_p}p^{(0)}_{a_1a_2\ldots a_{i-1}va_{i+1}\ldots
a_{p-1}a_p},\quad v=1,\ldots,h_i.
\]
Let the Level $v$ count of factor $C_i$ in the $k$th sample of size $N$ in Phase II
be $n_{(i,v)k}$, which equals
\[
n_{(i,v)k}=\sum_{a_1}\sum_{a_2}\ldots\sum_{a_{i-1}}\sum_{a_{i+1}}\ldots
\sum_{a_{p-1}}\sum_{a_p}n_{a_1a_2\ldots a_{i-1}va_{i+1}\ldots a_{p-1}a_p,k},\quad
v=1,\ldots,h_i.
\]
In the IC state, $n_{(i,v)k}$ ($v=1,\ldots,h_i$) jointly follow the multinomial
distribution $\mbox{MN}\big(N;$ $p^{(0)}_{(i,1)},\ldots,p^{(0)}_{(i,h_i)}\big)$. Let
${\bf n}_{\mathrm{MM},(i)k}=\big[n_{(i,1)k},\ldots,n_{(i,h_i-1)k}\big]^T$ and ${\bf
p}^{(0)}_{\mathrm{MM},(i)}=\big[p^{(0)}_{(i,1)},\ldots,p^{(0)}_{(i,h_i-1)}\big]^T$.
For the $k$th sample, the Pearson chi-square charting statistic given by Marcucci
(1985) is
\begin{equation}
G_{\mathrm{MM},(i)k}=\frac{1}{N}\big({\bf n}_{\mathrm{MM},(i)k}-N{\bf
p}^{(0)}_{\mathrm{MM},(i)}\big)^T\bm{\Sigma}^{-1}_{\mathrm{MM},(i)} \big({\bf
n}_{\mathrm{MM},(i)k}-N{\bf p}^{(0)}_{\mathrm{MM},(i)}\big),\label{F1.2}
\end{equation}
where $\bm{\Sigma}_{\mathrm{MM},(i)}$ has the elements
\begin{align*}
\bm{\Sigma}_{\mathrm{MM},(i)\;uv}&=\left\{\begin{array}{ll}p^{(0)}_{(i,u)}
\big(1-p^{(0)}_{(i,u)}\big)&\textrm{if
$u=v$}\\-p^{(0)}_{(i,u)}p^{(0)}_{(i,v)}&\textrm{if $u\ne v$}\end{array}\right.\quad
u,v=1,\ldots,h_i-1.
\end{align*}
So each factor has a charting statistic $G_{\mathrm{MM},(i)k}$ ($i=1,\ldots,p$), and
they consist of a multi-chart. Such a multi-chart would signal whenever at least one
of the $p$ individual charts signals.



\section{Research Objectives}\label{sec1.3}

By reviewing these typical monitoring methods, the above-mentioned two shortcomings,
including applying to only multivariate binomial processes and univariate
multinomial processes as well as almost entirely neglecting cross-classification
interactions among factors, can be seen more clearly. In particular, the
$\chi^2$-chart applies to only multivariate binomial processes, and the multi-chart
developed for multivariate multinomial processes is actually problematic. Both
charts focus on the one-way marginal sums of each factor.

This thesis provides a systematic methodology of statistical process control for
multivariate categorical processes. The whole work is based on log-linear models
(Bishop et al. (2007)), which can completely characterize both the marginal
distribution and the association structure between multiple categorical factors
subject to a multivariate binomial/multinomial distribution. Therefore, the existing
two difficulties in the literature on SPC for multivariate categorical processes can
be overcome, either in Phase I or Phase II. This forms the main objective of this
thesis. In particular, the contribution is composed of the following three aspects:
\begin{enumerate}
\item A log-linear multivariate categorical (LMC) control chart is proposed for
online monitoring, which is a general Phase II tool to detect various shifts
efficiently, especially those in interaction effects representing the dependence
among characteristics.
\item A log-linear directional (LLD) control chart is developed for Phase II SPC,
which exploits directional shift information and aims at online detecting more
practical shifts with some patterns, namely one-coefficient shifts and high-order
interaction shifts. The corresponding diagnostic method can also identify the shift
location powerfully.
\item An off-line Phase I analysis method is proposed by change-point detection
in a dataset collected from a multivariate categorical process. This approach also
considers directional shift information and can detect one-coefficient shifts and
high-order interaction shifts powerfully. The corresponding diagnostic method can
also recognize the change-point and the shift location with accuracy.
\end{enumerate}

The rest of the thesis is arranged as follows. We propose the two control charts,
namely the log-linear multivariate categorical (LMC) control chart and the
log-linear directional (LLD) control chart, for online monitoring multivariate
categorical processes in Chapters 2 and 3, respectively. Note that these two charts
are for Phase II SPC. For Phase I analysis, we propose in Chapter 4 the directional
method for detecting change-points in multivariate categorical processes. In Chapter
5, concluding remarks together with some future research opportunities in this area
are provided. For convenience, appendixes relating to one chapter are attached at
the end of it, instead of the end of the thesis.

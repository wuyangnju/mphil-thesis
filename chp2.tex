\chapter{Log-Linear Multivariate Categorical Chart}\label{chp2}

\section{Introduction}\label{sec2.1}

This chapter and the next focus on Phase II SPC for multivariate categorical
processes, assuming that the IC parameters are known or have been estimated based on
an IC reference dataset. The proposed log-linear multivariate categorical (LMC)
control chart is illustrated in this chapter, which is a general and robust one for
monitoring multivariate categorical processes.

Existing methods lack robustness for some deficiencies, in that they only apply to
univariate/multivariate binomial processes or univariate multinomial processes, and
they neglect the interaction structure among categorical factors. Compared to them,
the LMC chart applies to the systematic framework of univariate/multivariate
binomial/multinomial processes, and it can characterize the association patterns
among categorical factors. Simply put, the LMC chart employs a log-linear model and
relates the logarithms of the expected cell counts in a multi-way contingency table
to a linear model that is similar to an analysis of variance (ANOVA) model. The LMC
chart utilizes the log-likelihood function of log-linear models and incorporates the
exponentially weighted moving average (EWMA) control scheme into the log-likelihood
function, leading to an exponentially weighted log-likelihood ratio test statistic.
Numerical results further confirm its effectiveness under various conditions, as
well as its superiority over existing charts in detecting shifts in interaction
effects that reflect the dependence among factors.



\section{Log-Linear Models}\label{sec2.2}

Multivariate categorical data are usually summarized in a multi-way contingency
table, and there is a clear need to model the relationship between each cell count
and factor levels associated with it. This resembles standard multi-way ANOVA, where
responses to all factor level combinations are also placed in a multi-way table.
Standard multi-way ANOVA is based on the assumption that the responses are normally
distributed, and it aims to quantify how the responses are influenced by main factor
effects and factor interaction effects. Consider for illustration a three-way ANOVA
model, where the three factors take $h_1$, $h_2$, and $h_3$ levels. Denote the
expected response with the first factor at its $i$th level, the second factor at its
$j$th level, and the third factor at its $k$th level by $y_{ijk}$ ($i=1,\ldots,h_1$;
$j=1,\ldots,h_2$; $k=1,\ldots,h_3$). The three-way ANOVA model is
\[
y_{ijk}=u^{(0)}+u^{(1)}_i+u^{(2)}_j+u^{(3)}_k
+u^{(1,2)}_{i,j}+u^{(1,3)}_{i,k}+u^{(2,3)}_{j,k}+u^{(1,2,3)}_{i,j,k},
\]
where $u^{(0)}$ is the overall mean, $u^{(1)}, u^{(2)}, u^{(3)}$ are the main
effects, $u^{(1,2)}, u^{(1,3)}, u^{(2,3)}$ are the two-factor interaction effects,
and $u^{(1,2,3)}$ is the three-factor interaction effect. Furthermore,
identifiability requires constraints such as
\[
\sum_i u^{(1)}_i=\sum_i u^{(1,2)}_{i,j}=\sum_i u^{(1,3)}_{i,k}=\sum_i
u^{(1,2,3)}_{i,j,k}=0
\]
for the first factor along its index $i$. Similar equations describe the second and
third factors along with their indexes $j$ and $k$, respectively. Therefore, ANOVA
can be represented as a linear regression model. From the generalized linear model
(GLM) point of view, a linear regression model where the response is normally
distributed has a canonical link function of unity (see McCullagh and Nelder
(1989)).

By analogy with the preceding standard ANOVA model, it is possible to build a
similar regression model relating the cell counts and their corresponding factor
levels in a multi-way contingency table. Note that in our case the response (the
cell count) is not normally distributed. With no restriction on the total sample
size, we have assumed that each cell count is independently Poisson distributed. A
GLM where the response follows a Poisson distribution has a canonical link function
that is a logarithm (see McCullagh and Nelder (1989)). Therefore, we assume a
log-linear model. A detailed discussion of log-linear models and their applications
are given by Bishop et al. (2007). For a three-way contingency table of size
$h_1\times h_2\times h_3$, the log-linear model characterizing the relationship
between the expectation $m_{ijk}$ ($i=1,\ldots,h_1$; $j=1,\ldots,h_2$;
$k=1,\ldots,h_3$) of the count in cell$(i,j,k)$ and the factor levels indexed with
$i,j,k$, is
\[
\ln m_{ijk}=u^{(0)}+u^{(1)}_i+u^{(2)}_j+u^{(3)}_k
+u^{(1,2)}_{i,j}+u^{(1,3)}_{i,k}+u^{(2,3)}_{j,k}+u^{(1,2,3)}_{i,j,k},
\]
where the $u$-terms are the main or factor interaction effects defined as in the
ANOVA model (see Bishop et al. (2007)). They also satisfy the identifiability
constraints. When the total sample size $N$ is fixed, the cell counts jointly follow
a multinomial distribution, and it is more convenient to focus on the probability
$p_{ijk}$ instead of the expectation $m_{ijk}=Np_{ijk}$. In this case, the
log-linear model will be
\begin{align}
&\ln p_{ijk}=u^{(0)}+u^{(1)}_i+u^{(2)}_j+u^{(3)}_k
+u^{(1,2)}_{i,j}+u^{(1,3)}_{i,k}+u^{(2,3)}_{j,k}+u^{(1,2,3)}_{i,j,k},\label{F2.1}
\end{align}
where the probabilities must satisfy $\sum_{i,j,k}p_{ijk}=1$. Clearly, the
interaction terms such as $u^{(1,2)}$ reflect the dependence among the factors, and
for a log-linear model without any interaction effects, the factors are independent.
In addition, we see that the main effect of a factor determines its marginal
distribution.

The identifiability constraints applicable to a log-linear model in the form of
Equation (\ref{F2.1}) are somewhat inconvenient to write out, but they can be
rewritten equivalently in the following form, which we illustrate for a $2\times 3$
contingency table having factor $C_1$ with two levels and factor $C_2$ with three
levels
\begin{eqnarray}
\begin{array}{lll}
u^{(0)}=\beta_0, & u^{(1)}_1=\beta_{(1)}, &
u^{(1)}_2=-\beta_{(1)},\\
u^{(2)}_1=\beta_{(2_1)}, & u^{(2)}_2=\beta_{(2_2)}, &
u^{(2)}_3=-\beta_{(2_1)}-\beta_{(2_2)},\\
u^{(1,2)}_{1,1}=\beta_{(1,2_1)}, & u^{(1,2)}_{1,2}=\beta_{(1,2_2)},
& u^{(1,2)}_{1,3}=-\beta_{(1,2_1)}-\beta_{(1,2_2)}, \\
u^{(1,2)}_{2,1}=-\beta_{(1,2_1)}, & u^{(1,2)}_{2,2}=-\beta_{(1,2_2)}, &
u^{(1,2)}_{2,3}=\beta_{(1,2_1)}+\beta_{(1,2_2)}.
\end{array}
\nonumber
\end{eqnarray}
Therefore, the logarithms of the probabilities $p_{ij}$ ($i=1,2$; $j=1,2,3$) can be
expressed as a linear combination of the coefficients. Here, $\beta_{(1)}$ measures
the main effect $u^{(1)}$ of factor $C_1$, $[\beta_{(2_1)},\beta_{(2_2)}]^T$
measures the main effect $u^{(2)}$ of factor $C_2$, and
$[\beta_{(1,2_1)},\beta_{(1,2_2)}]^T$ measures the interaction effect $u^{(1,2)}$ of
the two factors. Such representation of factor effects in terms of coefficients can
be extended to a general scenario.

The identifiability constraints dictate that the log-linear model for a general
$p$-way contingency table in which $p$ factors are considered in the form of
Equation (\ref{F2.1}) can be expressed in the following regression form (see
Dahinden et al. (2007))
\begin{equation}
\ln {\bf p}=\bm 1\beta_0+\sum_{i=1}^{2^p-1}{\bf X}_i\bm{\beta}_i,\label{F2.2}
\end{equation}
where ${\bf p}$ is the $h\times 1$ probability vector corresponding to the $h$ cells
of the contingency table, $\bm{1}$ is a column vector of appropriate dimension
consisting of 1 as all its entries, ${\bf X}_i$ is an $h\times q_i$ design submatrix
corresponding to the $i$th main or interaction effect and containing 1, 0, or $-1$
as its elements, and $\bm{\beta}_i$ is the coefficient subvector of size $q_i\times
1$. For the preceding illustrative $2\times 3$ contingency table example with $p=2$
factors, ${\bf p}=[p_{11},p_{12},p_{13},p_{21},p_{22},p_{23}]^T$ and $\bm 1=\bm 1_6$
are both of size $6\times 1$, and $\bm\beta_1=\beta_1$,
$\bm\beta_2=[\beta_2,\beta_3]^T$, $\bm\beta_3=[\beta_4,\beta_5]^T$, and the design
submatrixes are
\[
{\bf X}_1=\left[\begin{array}{r}\bm 1_3\\-\bm 1_3\end{array}\right],\ {\bf
X}_2=\left[\begin{array}{r}{\bf J}\\{\bf J}\end{array}\right],\ \mathrm{and}\ {\bf
X}_3=\left[\begin{array}{r}{\bf J}\\-{\bf J}\end{array}\right],
\]
where
\[
{\bf J}=\left[\begin{array}{rr}1&0\\0&1\\-1&-1\end{array}\right].
\]
Note that the column sums in ${\bf J}$ are all zero, which assures identifiability.

Denoting the design matrix by
\[
\widetilde{{\bf X}}=[\bm{1},{\bf X}]\quad\mathrm{with}\quad{\bf X}=[{\bf
X}_1,\ldots,{\bf X}_{2^p-1}]
\]
and the coefficient vector by
\[
\widetilde{\bm{\beta}}=[\beta_0,\bm{\beta}^T]^T\quad\mathrm{with}\quad
\bm\beta=[\bm{\beta}_1^T,\ldots,\bm{\beta}_{2^p-1}^T]^T,
\]
we can rewrite Equation
(\ref{F2.2}) as $\ln {\bf p}=\widetilde{{\bf X}}\widetilde{\bm{\beta}}$. The
log-linear model (\ref{F2.2}) is at the effect level, and there are in total $2^p-1$
effects from the main effects up to the $p$-factor interaction effect. Here,
$\bm1^T{\bf p}=1$, and $\beta_0$ is a scalar representing the intercept. Therefore,
only $h-1$ coefficients are free to vary independently, and $\beta_0$ is included to
ensure the constraint $\bm1^T{\bf p}=1$. Hereafter, attention will be paid to the
coefficient subvectors $\bm\beta=[\bm{\beta}_1^T,\ldots,\bm{\beta}_{2^p-1}^T]^T$. In
the case of factors all with two levels, the derivation of the design matrix
$\widetilde{{\bf X}}$ is identical to that of the design matrix of a $2^k$ full
factorial experiment with 1 and $-1$ representing the high and low levels,
respectively. However, this becomes a little complex if at least one factor has more
than two levels. We defer the design matrix derivation for a general multivariate
categorical process till the next chapter, since the LMC chart depicted in this
chapter actually skips such derivation in computing its charting statistic.

The design submatrixes, together with their corresponding coefficient subvectors,
are arranged in a log-linear model from the overall mean $\beta_0$, the main
effects, up to the effect of the highest order. For example, we consider three
factors $C_1$, $C_2$, and $C_3$ with 2, 3, and 3 levels, respectively. The sequence
is the overall mean, the main effects $C_1$, $C_2$, and $C_3$, the two-factor
interaction effects $C_1C_2$, $C_1C_3$, and $C_2C_3$, and finally the three-factor
interaction effect $C_1C_2C_3$. Hence, the coefficient vector is
\begin{eqnarray}
\begin{array}{rrrlllllll}
\widetilde{\bm{\beta}} & = & [ & \beta_0 & \beta_{(1)} & \beta_{(2_1)} &
\beta_{(2_2)} & \beta_{(3_1)} & \beta_{(3_2)} & \\
& & & \beta_{(1,2_1)} & \beta_{(1,2_2)} & \beta_{(1,3_1)} & \beta_{(1,3_2)} &
\beta_{(2_1,3_1)} & \beta_{(2_1,3_2)} & \\
& & & \beta_{(2_2,3_1)} & \beta_{(2_2,3_2)} & \beta_{(1,2_1,3_1)} &
\beta_{(1,2_1,3_2)} & \beta_{(1,2_2,3_1)} & \beta_{(1,2_2,3_2)} & ]^T.
\end{array}\nonumber
\end{eqnarray}
Note that by the identifiability constraints, the scalar $\beta_{(1)}$ is sufficient
to determine the main effect of factor $C_1$ with only two levels. However, we need
the two coefficients $\beta_{(3_1)}$ and $\beta_{(3_2)}$ consisting of the subvector
$\bm\beta_3$ to jointly represent the main effect of factor $C_3$ with three levels.
In fact, for an effect that contains a factor with three levels or more, its
corresponding coefficient subvector has two elements or more, instead of reducing
into a scalar. Similarly, $\bm{\beta}_4=[\beta_{(1,2_1)},\beta_{(1,2_2)}]^T$
measures the two-factor interaction effect $C_1C_2$, and
$\bm{\beta}_7=[\beta_{(1,2_1,3_1)},
\beta_{(1,2_1,3_2)},\beta_{(1,2_2,3_1)},\beta_{(1,2_2,3_2)}]^T$ measures the
three-factor interaction effect $C_1C_2C_3$. We see that the $i$th main or
interaction effect, the design submatrix ${\bf X}_i$, and the coefficient subvector
$\bm{\beta}_i$ ($i=1,\ldots,2^p-1$) correspond to each other. Therefore, the cell
probability vector is ultimately determined by the magnitudes of these coefficient
subvectors.

A log-linear model for a $p$-way contingency table is saturated if it involves all
the effects of all orders from 1 up to $p$. In other words, from the main effects to
the $p$-factor interaction effect, they are all included in the log-linear model.
See model (\ref{F2.1}) for a saturated log-linear model example. Usually, a reduced
model can be obtained via variable selection. It means that some effects in the
saturated model can be dropped whereas others are retained. Therefore, some of the
coefficient subvectors $\bm\beta_i$ together with their corresponding design
submatrixes ${\bf X}_i$ in model (\ref{F2.2}) may be dropped.

For the reduced models, the hierarchy principle should be followed. It means that
all the lower-order effects have to be included if a higher-order interaction effect
containing them appears in the model. For instance, for a three-way contingency
table, if the effect $u^{(2,3)}$ is in the log-linear model, $u^{(2)}$ and $u^{(3)}$
should also be adopted. The hierarchical models can be denoted in a simple way, such
as [13][23] representing the model without the effects $u^{(1,2)}$ or $u^{(1,2,3)}$
as well as [123] representing the saturated model (\ref{F2.1}). In a log-linear
model, the design matrix such as $\widetilde{\bf X}$ may actually be constructed in
other ways. If the hierarchical log-linear model is reparametrized using a different
design matrix, all the zero terms in the coefficient vector still remain zero.
However, this cannot apply to non-hierarchical ones. In other words, hierarchy is
preserved after reparametrization, and all the zero coefficients can be interpreted
in terms of conditional independence (Dahinden et al. (2007)). This is the main
advantage of hierarchical models over non-hierarchical ones.



\section{Phase II Multivariate Categorical Monitoring}\label{sec2.3}

This chapter focuses on Phase II monitoring only and presumes that the coefficient
vector $\widetilde{\bm\beta}$ in the log-linear model (\ref{F2.2}) has already been
estimated from an IC dataset by variable selection and parameter estimation, which
is denoted by $\widetilde{\bm\beta}^{(0)}$. Refer to Christensen (1997) to see the
stepwise procedures for variable selection and the Newton-Raphson iterative
algorithm for parameter estimation. Correspondingly, let ${\bf p}^{(0)}$ be the
known IC cell probability vector, which can be derived from
$\widetilde{\bm\beta}^{(0)}$.

Based on the IC coefficient vector in a log-linear model, we present in this section
a Phase II monitoring scheme, which is able to detect various shifts among multiple
categorical variables in the systematic framework of univariate/multivariate
binomial/multinomial distributions. For simplicity, let $F({\bf X};{\bm\beta})$
represent the pre-specified log-linear model
\begin{align}
\ln{\bf p}=\widetilde{\bf X}\widetilde{\bm\beta}=\bm 1\beta_0+{\bf
X}\bm\beta\quad\mathrm{and}\quad\bm 1^T{\bf p}=1.\label{F2.3}
\end{align}
Since $\beta_0$ can be determined by $\bm\beta$, we focus on $\bm\beta$. It is
usually reasonable to assume that the $j$th online multivariate sampling observation
vector, ${\bf n}_j$ of size $h\times 1$, is collected over time from the following
change-point model with the Phase II sample size $N$
\begin{align}
{\bf n}_j\ {\mathop{\sim}\limits^{\mbox{i.i.d.}}}\left\{\hspace{-0.1cm}
\begin{array}{ll} F({\bf X};{\bm\beta}^{(0)}), & {\mathrm {for\ }} \quad j=1,\ldots,\tau,\\[0.1cm]
F({\bf X};{\bm\beta}^{(1)}), & {\mathrm {for\ }} \quad j=\tau+1,\ldots,
\end{array}\right.\label{F2.4}
\end{align}
where $\tau$ is the unknown change-point, and $\bm\beta^{(0)} \neq \bm\beta^{(1)}$
are the known IC and unknown OC process coefficient vectors, respectively. To this
end, the monitoring problem is closely related to the goodness-of-fit test in the
context of multinomial analysis (Bishop et al. (2007)).

There is a one-to-one correspondence between factor effects and coefficient
subvectors. Therefore, shifts in the marginal distributions of factors or the
dependence among multiple factors to OC states, which appear in the form of
deviations of their main effects or interaction effects, respectively, are reflected
in the changes of corresponding coefficient subvectors. As a result, the monitoring
task is to test if $\bm\beta=\bm\beta^{(0)}$. Based on the likelihood ratio test
(LRT) (Christensen (1997)), a naive method that comes to mind for online detection
is to use the current sampling observation vector to construct a Shewhart-type
chart. However, this would be very inefficient for moderate and small changes, since
it completely ignores past samples. As an alternative, we may consider an EWMA
scheme. A natural idea is to first obtain the estimate of the coefficient vector
$\bm\beta$ for each sample, and then apply the multivariate EWMA chart (Lowry et al.
(1992)) and some modifications of the LRT to these estimates of $\bm\beta$ at
different time points. However, this naive approach may still be inefficient, since
for each sample the coefficient vector $\bm\beta$ is estimated based on only $N$
random observations.

Alternatively, we propose an EWMA-type control chart by using the idea of weighted
likelihood. The log-likelihood of the observation vector ${\bf n}_j$ in the $j$th
sample of size $N$ in Phase II can be written from the probability mass function
(PMF) of the multinomial distribution and expressed as
\begin{align*}
l_j(\widetilde{\bm\beta})=&\sum_{a_1,a_2,\ldots,a_p}n_{a_1a_2\ldots a_p,j}\ln
p_{a_1a_2\ldots a_p} + \ln(N!)-\sum_{a_1,a_2,\ldots,a_p}\ln(n_{a_1a_2\ldots
a_p,j}!)\nonumber\\
=&{\bf n}^T_j\ln{\bf p}+\ln(N!)-\bm 1^T\ln({\bf n}_j!)\nonumber\\
=&{\bf n}^T_j\widetilde{\bf X}\widetilde{\bm\beta}+\ln(N!)-\bm 1^T\ln({\bf
n}_j!).
%\nonumber\\
%=&N\beta_0+{\bf n}^T_j{\bf X}\bm\beta+\ln(N!)-\bm 1^T\ln({\bf n}_j!)
\end{align*}
For any time point $k$, consider the following exponentially weighted log-likelihood
over samples 1 to $k$,
\begin{equation*}
w_k(\widetilde{\bm\beta})=
a_{0,k,\mu}^{-1}\sum_{j=1}^{k}(1-\mu)^{k-j}l_j(\widetilde{\bm\beta}),
\end{equation*}
where $\mu\in(0,1]$ is the smoothing parameter, and $
a_{t_0,t_1,\mu}=\sum_{j=t_0+1}^{t_1}(1-\mu)^{t_1-j}$ is a sequence of constants to
ensure that all the weights sum up to 1. Here, $w_k(\widetilde{\bm\beta})$ makes
full use of all available samples up to the current time point $k$, and different
samples are weighted as in an EWMA chart (i.e., the more recent sample has more
weight, and the weight changes exponentially over time). Then given $\mu$, the
maximum weighted likelihood estimate of $\widetilde{\bm\beta}$ at the time point
$k$, $\widehat{\widetilde{\bm\beta}}_k$, is defined as the solution to the following
maximization problem,
\begin{align}
\widehat{\widetilde{\bm\beta}}_k=\arg\max_{\widetilde{\bm\beta}}
w_k(\widetilde{\bm\beta}),\ \ &\mbox{subject to} \ \bm 1^T\exp(\widetilde{\bf
X}\widetilde{\bm\beta})=1.\label{F2.5}
\end{align}
Therefore, the weighted likelihoods under the alternative and null hypotheses,
evaluated at $\widehat{\wt{\bm\beta}}_k$ and $\wt{\bm\beta}^{(0)}$, respectively,
are given by
\begin{align*}
{w}_{1,k}&=a_{0,k,\mu}^{-1}\sum_{j=1}^{k}(1-\mu)^{k-j}l_j(\widehat{\wt{\bm\beta}}_k),\\
{w}_{0,k}&=a_{0,k,\mu}^{-1}\sum_{j=1}^{k}(1-\mu)^{k-j}l_j({\wt{\bm\beta}}^{(0)}).
\end{align*}
Consequently, the $-2$LRT statistic will be
\begin{align}
U_k&=2(w_{1,k}-w_{0,k})\nonumber\\
&=2{\bf z}_k^T\big(\wt{\bf X}\widehat{\wt{\bm\beta}}_k-\wt{\bf
X}\wt{\bm\beta}^{(0)}\big), \label{F2.6}
\end{align}
where
\begin{equation}
{\bf z}_k=a_{0,k,\mu}^{-1}\sum_{j=1}^{k}(1-\mu)^{k-j}{\bf n}_j.\label{F2.7}
\end{equation}
Clearly, ${\bf z}_k$ is the exponentially weighted average of the observation
vectors ${\bf n}_j$ ($j=1,\ldots,k$) over time. Before proceeding, we need to
calculate the weighted likelihood $w_{1,k}$ in which ${\bf z}_k^T\wt{\bf
X}\widehat{\wt{\bm\beta}}_k$ is involved. Although computing power has greatly
improved, and it is computationally trivial to perform log-linear model estimation
for individual sampling observation vectors, for online process monitoring which
generally handles a large amount of samples, fast implementation is important, and
some computational issues deserve our careful examination. At first glance, to
obtain $\widehat{\wt{\bm\beta}}_k$ by solving the maximization problem (\ref{F2.5})
directly requires a considerable amount of computing time especially when $k$ is
large. However, notice that the core of the exponentially weighted log-likelihood
$w_k(\wt{\bm\beta})$ over samples 1 to $k$, which contains $\wt{\bm\beta}$, is
$a_{0,k,\mu}^{-1}\sum_{j=1}^{k}(1-\mu)^{k-j}{\bf n}_j\wt{\bf X}\wt{\bm\beta}$, and
that the log-likelihood of ${\bf z}_k$ is ${\bf z}_k\wt{\bf X}\wt{\bm\beta}$ by
ignoring some constants. According to Equation (\ref{F2.7}), these two parts are
equal, which concludes the following proposition.
\begin{pro}
The $\wh{\wt{\bm\beta}}_k$ is the maximum likelihood estimation (MLE) of the
log-linear model (\ref{F2.3}) with a pseudo-observation vector ${\bf z}_k$ in
Equation (\ref{F2.7}).
\end{pro}
We call ${\bf z}_k$ the pseudo-observation vector, because its components are not
integers so that it cannot be observed in reality. Proposition 1 provides an easy
way to evaluate $w_{1,k}$. Furthermore, since $\bm 1^T{\bf z}_k=N$ and the
constraint in Equation (\ref{F2.5}) is identical to that in the log-linear model
(\ref{F2.3}), the maximization in Equation (\ref{F2.5}) is exactly the same as
solving the MLE for model (\ref{F2.3}) with ${\bf z}_k$. Therefore, we can rewrite
$U_k$ in Equation (\ref{F2.6}) as
\begin{equation}
U_k=2{\bf z}_k^T(\ln\widehat{\bf q}_k-\ln{\bf p}^{(0)}),\label{F2.8}
\end{equation}
where $\widehat{\bf q}_k$ is the MLE of the cell probability vector estimated from
${\bf z}_k$. Note that the probability vector $\widehat{\bf q}_k$ in Equation
(\ref{F2.8}) can be obtained by performing the iterative proportional fitting (IPF)
algorithm (Bishop et al. (2007)), which is efficient in calculating the MLE of cell
count expectations under any hierarchical log-linear model and is included in major
statistical softwares such as the subroutine ``PRPFT'' in Fortran with the IMSL
library.

By using Equation (\ref{F2.8}), a large value of $U_k$ rejects the null hypothesis,
and hence our proposed chart triggers an OC signal if
\[
U_k>L,\  \mbox{for}\  k\geq 1,
\]
where $L>0$ is a control limit chosen to achieve a specific IC ARL, denoted by
ARL$_0$. This is the proposed log-linear multivariate categorical (LMC) control
chart. The control limit $L$ can be searched by simulation based on the IC model.
For a given $\mu$, model $F({\bf X};\bm\beta)$, and a desired ARL$_0$, the
computation involved in finding $L$ is not difficult, partly due to the fact that
the IPF algorithm used in the MLE computation is efficient. For the searching
procedure, some numerical searching algorithms, such as the bisection search, can be
applied.

We now discuss the diagnostic issue of multivariate binomial and multivariate
multinomial processes. According to the one-to-one correspondence between factor
effects and coefficient subvectors in a log-linear model, shifts in the marginal
distribution of one factor lead to deviations of the coefficient subvector
corresponding to its main effect, and shifts in the dependence among multiple
factors result in deviations of the coefficient subvector reflecting their
interaction effect. Possible diagnostic procedures should focus on separating the
shifted coefficient subvector once an OC signal is triggered. The LASSO proposed by
Tibshirani (1996) could be employed to identify this. In particular, Zou et al.
(2011) provided a general framework for diagnosis, and log-linear models can be
formulated to adapt this framework.

\noindent {\it Remark 1}. In some applications, only a few cells or even only one
cell of a contingency table have very large probabilities, and therefore the
remaining cells share the residuary very small probability. If some cells have very
small probabilities or the sample size is not large enough, there will be no
observations in these cells, and these cells have zero counts and are
noninformative. Because of the sampling zeros, it may be questioned whether the
sampled data come from the multinomial distribution that it claims to be. So this
sparsity phenomenon deserves special consideration. A direct aftermath of sparsity
in model estimation is the nonexistence of MLEs, which is related to the presence of
zero cells. Indeed, some researchers have informally observed that the only thing
that appeared to matter was the number and locations of sampling zeros in the
contingency table, not the values of the cell counts in the remaining cells
(Fienberg and Rinaldo (2007)). For each observation vector ${\bf n}_j$, it is
probably the case that sampling zeros would occur when the sample size in Phase II
is not large enough, or some cell probabilities are extremely small. However, the
locations of zero cells may differ from sample to sample. By combining the collected
samples in an exponentially weighted way, the probability that the
pseudo-observation vector ${\bf z}_k$ still has exact zero cells is low, especially
when $k$ becomes large as the process goes on. The cell counts of ${\bf z}_k$ still
sum up to the sample size $N$, and most of them may not be integers, i.e., a
pseudo-observation vector. Therefore, the sparsity is mitigated to a large extent or
even eliminated by the EWMA scheme, and much more information about the process
states is collected, as if the sample size is very large. Thus the nonexistence of
MLEs is avoided, and they can be performed with accuracy. We found that the
non-convergence frequency of IPF is negligible in our all simulation studies. Of
course, how to handle the non-convergence situation should be a topic for future
research.

\noindent {\it Remark 2}. The IPF method proportionally adjusts the cell counts of a
sample to fit a set of margins (e.g., the factor association or interaction
structure). Once given the hierarchical structure of log-linear models, this
successive proportional adjustment can obtain directly the MLE of the cell
probability vector $\bf p$ by skipping the estimation of the coefficient vector
$\widetilde{\bm\beta}$. In addition, it has the following properties (Biship et al.
(2007)): (1) converging to the required unique set of MLEs; (2) a stopping rule that
ensures accuracy to any desired degree in the elementary cell estimates; (3)
depending only on the sufficient configurations and no special provision for
sporadic cells with no observations (this relates to the sparsity phenomenon above);
(4) any set of starting values; (5) yielding the exact estimates in one cycle if
direct estimates exist. The IPF algorithm possesses these advantages over the
earlier techniques proposed, such as the Newton-Raphson which fails to have
properties (3) and (5) above. Therefore, we adopt the IPF for online estimation in
Phase II.



\section{Performance Assessment}\label{sec2.4}

In this section, the performance of the proposed LMC chart is investigated through
some numerical simulations. The simulations are made in scenarios of both
multivariate binomial and multivariate multinomial processes. In both cases, the LMC
chart is compared with its counterparts in various cases of shifts, and the
advantages of the LMC chart over others are verified. Throughout the simulation, for
fair comparison, the IC ARL is fixed at 370 for each control chart, and all ARL
values reported are averages of 10,000 replicated simulations. If the process is OC,
a smaller OC ARL of a chart means that it needs fewer samples and gives rise to an
OC signal faster, and that therefore this chart performs better.

\subsection{Monitoring a Multivariate Binomial Process}

For monitoring multivariate binomial processes, a natural competitive method is
Patel's (1973) Hotelling's $T^2$ type $\chi^2$-chart. However, since the
$\chi^2$-chart is the Shewhart-type, and the LMC chart is the EWMA-type, the
comparison may not be fair to the Shewhart-type as we could expect its deficiency in
detecting small and moderate shifts due to the fact that it completely ignores past
information. Hence, the EWMA version of the $\chi^2$-chart, which is a typical
competitor of the LMC chart for monitoring multivariate binomial processes, is
developed accordingly for fair comparison. By replacing ${\bf n}_{\mathrm{MB},k}$ in
Equation (\ref{F1.1}) with
\[
{\bf z}_{\mathrm{MB},k}=a_{0,k,\mu}^{-1}\sum_{j=1}^{k}(1-\mu)^{k-j}{\bf
n}_{\mathrm{MB},j},
\]
we combine the $\chi^2$-chart with the EWMA scheme and refer to this revised chart
as the multivariate binomial EWMA (MBE) control chart. We compare it with the LMC
chart in monitoring multivariate binomial processes.

Suppose that during a production process, five quality characteristics each labeled
as conforming or nonconforming are being monitored, hence we have a five-way
contingency table with $2^5$ cells. After model estimation from an IC dataset, we
obtain the IC model hierarchy structure [14][123][135][234][235][345] and the IC
log-linear model with the coefficient vector
\begin{eqnarray}
\begin{array}{rrrrrrrrrrrl}
\wt{\bm\beta}^{(0)} & = & [ & \beta_0 & 0.72 & 0.93
& 0.49 & 0.25 & 0.47 & -0.57 & 0.22 & \\
& & & 0.11 & -0.14 & 0.15 & -0.16 & 0.41 & 0.16 & -0.19 & 0.33 & \\
& & & 0.39 & 0 & 0 & 0 & 0.21 & 0 & 0.45 & 0.33 & \\
& & & 0 & 0.27 & 0 & 0 & 0 & 0 & 0 & 0 & ]^T.
\end{array}
\nonumber
\end{eqnarray}
Here, $\beta_0$ is the intercept accommodating the constraint that all cell
probabilities sum up to 1 as expressed in Equation (\ref{F2.3}). Note that the zeros
here mean that the corresponding effect terms are excluded.

%Table 1
\begin{table}[htp]
\tabcolsep 4.5pt \vspace{-0.1cm} \centering \caption{OC ARL comparison between the
LMC and MBE charts when only one coefficient changes} \vspace{0.3cm}
\renewcommand{\arraystretch}{1.32}
\begin{tabular}{r|cccc|cccc|cccc}\hline
$\delta$ & \multicolumn{2}{c}{LMC} & \multicolumn{2}{c|}{MBE} &
\multicolumn{2}{c}{LMC} & \multicolumn{2}{c|}{MBE} & \multicolumn{2}{c}{LMC} &
\multicolumn{2}{c}{MBE}
\\\hline
& \multicolumn{4}{c|}{$\beta_{(1)}$} & \multicolumn{4}{c|}{$\beta_{(4)}$} &
\multicolumn{4}{c}{$\beta_{(1,2)}$}
\\\hline
0.01 & 232 & (2.30) & 160 & (1.52) & 246 & (2.40) & 172 & (1.65) & 210 & (2.02) &
156 & (1.51)
\\
0.02 & 87.8 & (0.77) & 49.0 & (0.39) & 100 & (0.92) & 56.7 & (0.47) & 71.0 & (0.61)
& 45.4 & (0.36)
\\
0.05 & 14.8 & (0.07) & 10.2 & (0.05) & 16.6 & (0.08) & 11.3 & (0.05) & 12.5 & (0.06)
& 9.61 & (0.04)
\\
0.20 & 2.99 & (0.01) & 2.38 & (0.01) & 3.24 & (0.01) & 2.56 & (0.01) & 2.67 & (0.01)
& 2.27 & (0.01)
\\
$-0.01$ & 224 & (2.19) & 146 & (1.40) & 240 & (2.37) & 156& (1.50) & 205 & (1.96) &
139 & (1.34)
\\
$-0.02$ & 81.7 & (0.72) & 44.1 & (0.36) & 94.8 & (0.87) & 50.2 & (0.42) & 67.7 &
(0.59) & 41.6 & (0.33)
\\
$-0.05$ & 13.5 & (0.07) & 9.21 & (0.04) & 15.1 & (0.08) & 10.2 & (0.05) & 11.7 &
(0.05) & 8.87 & (0.04)
\\
$-0.20$ & 2.41 & (0.01) & 1.95 & (0.01) & 2.54 & (0.01) & 2.04 & (0.01) & 2.26 &
(0.01) & 1.94 & (0.01)
\\\hline
& \multicolumn{4}{c|}{$\beta_{(2,3)}$} & \multicolumn{4}{c|}{$\beta_{(2,5)}$} &
\multicolumn{4}{c}{$\beta_{(3,4)}$}
\\\hline
0.01 & 259 & (2.55) & 248 & (2.41) & 289 & (2.88) & 322 & (3.28) & 262 & (2.62) &
305 & (3.01)
\\
0.02 & 110 & (1.01) & 106 & (0.97) & 148 & (1.42) & 199 & (1.93) & 114 & (1.04) &
156 & (1.50)
\\
0.05 & 18.3 & (0.10) & 19.2 & (0.11) & 25.9 & (0.16) & 44.4 & (0.35) & 19.0 & (0.10)
& 30.3 & (0.21)
\\
0.20 & 3.42 & (0.01) & 3.54 & (0.01) & 4.21 & (0.01) & 5.86 & (0.02) & 3.49 & (0.01)
& 4.61 & (0.02)
\\
$-0.01$ & 248 & (2.47) & 221 & (2.20) & 279 & (2.79) & 287 & (2.84) & 249 & (2.47) &
259 & (2.56)
\\
$-0.02$ & 101 & (0.91) & 89.3 & (0.81) & 139 & (1.29) & 170 & (1.63) & 106 & (0.99)
& 131 & (1.24)
\\
$-0.05$ & 16.3 & (0.09) & 16.4 & (0.10) & 22.4 & (0.14) & 34.9 & (0.26) & 16.9 &
(0.09) & 24.5 & (0.17)
\\
$-0.20$ & 2.63 & (0.01) & 2.66 & (0.01) & 3.07 & (0.01) & 4.01 & (0.01) & 2.67 &
(0.01) & 3.37 & (0.01)
\\\hline
& \multicolumn{4}{c|}{$\beta_{(1,3,5)}$} & \multicolumn{4}{c|}{$\beta_{(2,3,4)}$} &
\multicolumn{4}{c}{$\beta_{(3,4,5)}$}
\\\hline
0.01 & 232 & (2.33) & 252 & (2.43) & 273 & (2.70) & 339 & (3.38) & 264 & (2.62) &
343 & (3.38)
\\
0.02 & 84.7 & (0.74) & 109 & (1.01) & 132 & (1.23) & 239 & (2.33) & 118 & (1.10) &
233 & (2.32)
\\
0.05 & 14.4 & (0.07) & 20.0 & (0.12) & 21.7 & (0.13) & 64.5 & (0.56) & 19.1 & (0.10)
& 59.7 & (0.50)
\\
0.20 & 2.95 & (0.01) & 3.56 & (0.01) & 3.80 & (0.01) & 7.41 & (0.03) & 3.55 & (0.01)
& 6.91 & (0.03)
\\
$-0.01$ & 227 & (2.24) & 227 & (2.21) & 267 & (2.65) & 314 & (3.14) & 251 & (2.46) &
302 & (2.97)
\\
$-0.02$ & 79.6 & (0.70) & 94.7 & (0.87) & 120 & (1.11) & 204 & (1.99) & 109 & (1.03)
& 194 & (1.88)
\\
$-0.05$ & 13.2 & (0.06) & 17.4 & (0.10) & 19.3 & (0.11) & 51.3 & (0.42) & 17.1 &
(0.09) & 46.5 & (0.37)
\\
$-0.20$ & 2.41 & (0.01) & 2.82 & (0.01) & 2.85 & (0.01) & 5.03 & (0.02) & 2.70 &
(0.01) & 4.81 & (0.02)
\\\hline
\multicolumn{13}{l}{NOTE: Standard errors are in parentheses.}
\end{tabular}
\end{table}


It is believed that possible shifts to OC states arise in the main effect of one
factor reflecting its marginal distribution or in the interaction effect of multiple
factors representing their dependence. According to the one-to-one correspondence
between factor effects and coefficient subvectors, shifts to OC states reflect the
changes of the corresponding coefficient subvectors in the IC log-linear model. For
simplicity, we consider the case where only one coefficient of the log-linear model
changes by adding a magnitude $\delta$ and study the OC ARL performance. Note that
the models before and after the change have the same hierarchy structure, and that
the change only occurs on the coefficient magnitude of one retained term. The OC
ARLs of the LMC and MBE charts for various shift magnitudes are presented in Table
2.1 with the smoothing parameter $\mu=0.1$ and the Phase II sample size $N=1,000$.
To save space, not all coefficients are listed. According to Table 2.1, for main
factor effects, e.g., $\beta_{(1)}$ and $\beta_{(4)}$, the MBE chart outperforms the
LMC chart almost uniformly. The MBE chart exactly and completely summarizes the
one-way marginal sums for each factor, which are mostly determined by the main
effects. Therefore, it is more sensitive than the LMC chart to the change of each
main effect, which reflects the shifts of the one-way marginal sums directly. For
two-factor interaction effects such as $\beta_{(1,2)}$, $\beta_{(2,3)}$,
$\beta_{(2,5)}$, and $\beta_{(3,4)}$ as well as three-factor interaction effects
including $\beta_{(1,3,5)}$, $\beta_{(2,3,4)}$, and $\beta_{(3,4,5)}$, as the effect
order increases, the LMC chart shows more and more significant superiority over the
MBE chart. The change occurring in a high-order interaction effect leads to little
shifts of the one-way marginal sums, which are difficult for the MBE chart to
detect. However, the LMC chart is still capable of capturing the potential change of
a high-order interaction effect powerfully via the log-likelihood ratio.

The OC ARLs of the LMC and MBE charts for the same coefficient shift magnitudes as
Table 2.1 under some other choices of the $\mu$ and $N$ combination are omitted here
for saving space. Generally, they exhibit the same patterns as in Table 2.1 for
various changes of the coefficients, and similar conclusions can be drawn. With a
fixed $\mu$, for the same change, both charts become more powerful when the sample
size $N$ increases. Moreover, for a fixed sample size $N$ and the same coefficient,
the LMC chart with a smaller $\mu$ detects smaller shifts faster, whereas it has a
better performance when detecting larger shifts with a larger $\mu$. This is
consistent with the properties of the conventional EWMA chart (Lucas and Saccucci
(1990)), and it is further confirmed by Figures 2.1-(a) and -(b), which show the OC
ARL curves (in log-scale) of the LMC chart with the $\mu$ values of 0.05, 0.1, 0.2,
and 0.5 when there are shifts in $\beta_{(2,5)}$ and $\beta_{(2,3,4)}$,
respectively. The above property can in fact guide the selection of $\mu$, and our
empirical results show that a reasonable suggestion of $\mu$ may be between 0.05 and
0.2.


\begin{figure}[ht]
\begin{center}
\includegraphics[width=16.0cm,height=8.0cm]{fig2-1.ps}
\vspace{-0.5cm} \caption{\small OC ARL curves of the LMC chart with various values
of $\mu$ when there are shifts in: (a) $\beta_{(2,5)}$; (b)
$\beta_{(2,3,4)}$}\vspace{-0.3cm}
\end{center}
\end{figure}


We assume in the above that the model hierarchy structures in both the IC and OC
states are identical. Recall that the IC model hierarchy structure is
[14][123][135][234][235] [345]. This means the IC log-linear model does not contain
the terms $\beta_{(1,2,4)}$, $\beta_{(1,2,5)}$, $\beta_{(1,3,4)}$,
$\beta_{(1,4,5)}$, $\beta_{(2,4,5)}$, $\beta_{(1,2,3,4)}$, $\beta_{(1,2,3,5)}$,
$\beta_{(1,2,4,5)}$, $\beta_{(1,3,4,5)}$, $\beta_{(2,3,4,5)}$,
$\beta_{(1,2,3,4,5)}$, which leads to the absence of the three-factor interaction
effects $C_1C_2C_4$, $C_1C_2C_5$, $C_1C_3C_4$, $C_1C_4C_5$, and $C_2C_4C_5$, as well
as all the four-factor interaction effects and the five-factor interaction effect.
However, sometimes the model structure itself changes when the process is OC.
Compared to the IC model, the OC model may be either reduced or extended. In either
case, the hierarchy principle should not be violated. After model extension by only
one term, one nonexistent effect term emerges, such as $\beta_{(1,2,4)}$,
$\beta_{(1,3,4)}$, and $\beta_{(1,4,5)}$ with some magnitude $\delta$, and the
original IC model does not encompass the current model any longer. Hence, the LMC
chart may not be necessarily superior over the MBE chart. The simulation results are
illustrated in the upper half of Table 2.2 with $\mu=0.1$ and $N=1,000$. On the
other hand, after model reduction by only one term, one effect becomes zero or
disappears, and the original IC model can still include the reduced model.
Therefore, the LMC chart may still outperform the MBE chart. Some cases of removing
merely one existent effect, for instance, $\beta_{(1,3,5)}$, $\beta_{(2,3,4)}$, and
$\beta_{(3,4,5)}$, are listed in the lower half.

%Table 2
\begin{table}[ht]
\tabcolsep 3.6pt \vspace{-0.1cm} \centering \caption{OC ARL comparison between the
LMC and MBE charts for model misspecification} \vspace{0.3cm}
\renewcommand{\arraystretch}{1.32}
\begin{tabular}{r|cccc|cccc|cccc}\hline
$\delta$ & \multicolumn{2}{c}{LMC} & \multicolumn{2}{c|}{MBE} &
\multicolumn{2}{c}{LMC} & \multicolumn{2}{c|}{MBE} & \multicolumn{2}{c}{LMC} &
\multicolumn{2}{c}{MBE}
\\\hline
$$ & \multicolumn{4}{c|}{$\beta_{(1,2,4)}$} &
\multicolumn{4}{c|}{$\beta_{(1,3,4)}$} & \multicolumn{4}{c}{$\beta_{(1,4,5)}$}
\\\hline
0.01 & 231 & (2.31) & 275 & (2.74) & 233 & (2.25) & 207 & (2.02) & 249 & (2.49) &
267 & (2.65)
\\
0.02 & 87.6 & (0.77) & 132 & (1.27) & 89.4 & (0.79) & 71.8 & (0.64) & 107 & (0.98) &
124 & (1.17)
\\
0.05 & 14.8 & (0.07) & 24.6 & (0.16) & 15.2 & (0.07) & 13.4 & (0.04) & 17.7 & (0.09)
& 22.9 & (0.15)
\\
0.20 & 2.96 & (0.01) & 3.99 & (0.01) & 2.97 & (0.01) & 2.78 & (0.01) & 3.30 & (0.01)
& 3.84 & (0.01)
\\
$-0.01$ & 224 & (2.17) & 239 & (2.29) & 232 & (2.29) & 179 & (1.74) & 248 & (2.45) &
240 & (2.36)
\\
$-0.02$ & 83.9 & (0.75) & 113 & (1.06) & 86.2 & (0.77) & 63.2 & (0.54) & 102 &
(0.92) & 109 & (1.01)
\\
$-0.05$ & 13.7 & (0.07) & 21.3 & (0.14) & 14.1 & (0.07) & 12.2 & (0.06) & 16.1 &
(0.08) & 19.8 & (0.12)
\\
$-0.20$ & 2.47 & (0.01) & 3.18 & (0.01) & 2.52 & (0.01) & 2.32 & (0.01) & 2.69 &
(0.01) & 3.05 & (0.01)
\\\hline
& \multicolumn{4}{c|}{$\beta_{(1,3,5)}$} & \multicolumn{4}{c|}{$\beta_{(2,3,4)}$} &
\multicolumn{4}{c}{$\beta_{(3,4,5)}$}
\\\hline
 & 2.26 & (0.01) & 2.64 & (0.01) & 1.20 & (0.00) & 1.94 &
(0.01) & 1.99 & (0.01) & 3.30 & (0.01)
\\\hline
\multicolumn{13}{l}{NOTE: Standard errors are in parentheses.}
\end{tabular}\vspace{-0.3cm}
\end{table}


\subsection{Monitoring a Multivariate Multinomial Process}

Apart from all factors with two levels, factors at least one with three or more
levels may exist in real production or services. A simple example is the attitude of
a customer towards a service, which may take on the values of excellent, acceptable,
or unacceptable. If four service indexes are taken into account, there will be a
four-way contingency table with $3^4$ cells for the four factors. In such complex
cases, it is challenging to compare the proposed approach with alternative methods.
This is because to the best of our knowledge, there is currently no appropriate
monitoring approach incorporating the cross-classifications among multiple factors,
among which at least one has more than two attribute levels. A naive method that
comes into mind for comparison is to monitor the $p$ groups of marginal sums of a
$p$-way contingency table by introducing $p$ individual charts separately.

If only the group of marginal sums for the $i$th factor $(i=1,\ldots,p)$ is
considered, we face the same situation as the generalized $p$-chart in Marcucci
(1985), which treated the monitoring of univariate multinomial processes with the
Pearson chi-square statistic. Still for fair comparison purpose, by replacing
\[
{\bf z}_{\mathrm{MM},(i)k}=a_{0,k,\mu}^{-1}\sum_{j=1}^{k}(1-\mu)^{k-j} {\bf
n}_{\mathrm{MM},(i)j}
\]
with ${\bf n}_{\mathrm{MM},(i)k}$ in Equation (\ref{F1.2}), we obtain the EWMA
version of the charting statistic of the generalized $p$-chart. Hereafter, this
updated multi-chart is referred to as the multivariate multinomial EWMA (MME) chart,
in the sense that it signals whenever at least one of these $p$ charts constituting
the MME chart does. We compare the LMC and MME charts in monitoring multivariate
multinomial processes.

Assume that a service flow has four quality characteristics under surveillance, with
the first two evaluated as satisfactory or dissatisfactory and the last two assessed
as excellent, acceptable, or unacceptable. This is a case of factors with mixed
levels, and it forms a four-way contingency table of size $2\times2\times3\times3$
with 36 cells. After model estimation from an IC dataset, we obtain the IC model
hierarchy structure [12][134][234] and the IC log-linear model with the coefficient
vector
\begin{eqnarray}
\begin{array}{rrrrrrrrrrrrl}
\wt{\bm\beta}^{(0)} & = & [ & \beta_0 & 0.73 & 0.72 & 0.70 &
0.12 & 0.72 & 0.11 & 0.17 & 0.12 & \\
& & & -0.15 & 0.19 & -0.14 & 0.23 & 0.07 & 0.16 & -0.14 & 0.23 & -0.30 & \\
& & & -0.17 & 0.14 & 0 & 0 & 0 & 0 & 0.19 & -0.15 & 0.11 & \\
& & & 0.22 & 0.24 & 0.24 & -0.08 & -0.16 & 0 & 0 & 0 & 0 & ]^T,
\end{array}
\nonumber
\end{eqnarray}
where $\beta_0$ is the intercept accommodating the sum 1 constraint in Equation
(\ref{F2.3}), and the zeros represent the removed effect terms.

%Table 3
\begin{table}[htp]
\tabcolsep 4.5pt \vspace{-0.1cm} \centering \caption{OC ARL comparison between the
LMC and MME charts when only one coefficient changes} \vspace{0.3cm}
\renewcommand{\arraystretch}{1.32}
\begin{tabular}{r|cccc|cccc|cccc}\hline
$\delta$ & \multicolumn{2}{c}{LMC} & \multicolumn{2}{c|}{MME} &
\multicolumn{2}{c}{LMC} & \multicolumn{2}{c|}{MME} & \multicolumn{2}{c}{LMC} &
\multicolumn{2}{c}{MME}
\\\hline
& \multicolumn{4}{c|}{$\beta_{(2)}$} & \multicolumn{4}{c|}{$\beta_{(4_2)}$} &
\multicolumn{4}{c}{$\beta_{(1,3_1)}$}
\\\hline
0.02 & 172 & (1.66) & 79.7 & (0.68) & 200 & (1.99) & 134 & (1.25) & 132 & (1.24) &
111 & (1.04)
\\
0.05 & 30.4 & (0.20) & 14.7 & (0.08) & 41.6 & (0.31) & 22.4 & (0.14) & 21.8 & (0.13)
& 20.4 & (0.12)
\\
0.20 & 4.57 & (0.01) & 3.02 & (0.01) & 4.73 & (0.01) & 3.36 & (0.01) & 3.66 & (0.01)
& 3.56 & (0.01)
\\
$-0.02$ & 153 & (1.48) & 68.4 & (0.61) & 205 & (1.97) & 141 & (1.34) & 125 & (1.16)
& 93.6 & (0.85)
\\
$-0.05$ & 25.7 & (0.17) & 12.5 & (0.07) & 42.8 & (0.32) & 23.1 & (0.14) & 19.8 &
(0.11) & 17.8 & (0.11)
\\
$-0.20$ & 3.26 & (0.01) & 2.26 & (0.01) & 4.85 & (0.01) & 3.46 & (0.01) & 3.00 &
(0.01) & 2.97 & (0.01)
\\\hline
& \multicolumn{4}{c|}{$\beta_{(1,4_2)}$} & \multicolumn{4}{c|}{$\beta_{(2,3_2)}$} &
\multicolumn{4}{c}{$\beta_{(2,4_1)}$}
\\\hline
0.02 & 201 & (1.96) & 243 & (2.39) & 198 & (1.90) & 230 & (2.26) & 134 & (1.26) &
114 & (1.09)
\\
0.05 & 41.0 & (0.30) & 61.6 & (0.52) & 39.5 & (0.29) & 52.8 & (0.43) & 22.2 & (0.13)
& 20.7 & (0.13)
\\
0.20 & 4.68 & (0.01) & 5.75 & (0.02) & 4.57 & (0.01) & 5.19 & (0.02) & 3.71 & (0.01)
& 3.59 & (0.01)
\\
$-0.02$ & 201 & (2.00) & 242 & (2.40) & 197 & (1.92) & 230 & (2.25) & 123 & (1.17) &
95.1 & (0.87)
\\
$-0.05$ & 41.9 & (0.31) & 64.2 & (0.55) & 39.6 & (0.29) & 55.0 & (0.45) & 20.0 &
(0.12) & 18.1 & (0.11)
\\
$-0.20$ & 4.72 & (0.01) & 6.31 & (0.02) & 4.66 & (0.01) & 5.96 & (0.02) & 3.00 &
(0.01) & 2.97 & (0.01)
\\\hline
& \multicolumn{4}{c|}{$\beta_{(3_2,4_1)}$} &
\multicolumn{4}{c|}{$\beta_{(3_2,4_2)}$} & \multicolumn{4}{c}{$\beta_{(1,3_1,4_2)}$}
\\\hline
0.02 & 234 & (2.30) & 291 & (2.84) & 276 & (2.74) & 343 & (3.46) & 241 & (2.37) &
342 & (3.42)
\\
0.05 & 55.4 & (0.45) & 103 & (0.95) & 90.8 & (0.82) & 228 & (2.22) & 62.7 & (0.51) &
222 & (2.14)
\\
0.20 & 5.47 & (0.02) & 7.88 & (0.03) & 7.32 & (0.03) & 20.8 & (0.13) & 5.90 & (0.02)
& 17.5 & (0.10)
\\
$-0.02$ & 234 & (2.38) & 278 & (2.78) & 277 & (2.70) & 339 & (3.41) & 243 & (2.36) &
335 & (3.34)
\\
$-0.05$ & 56.6 & (0.47) & 106 & (0.99) & 92.4 & (0.84) & 238 & (2.34) & 63.8 &
(0.54) & 217 & (2.12)
\\
$-0.20$ & 5.64 & (0.02) & 9.23 & (0.04) & 7.58 & (0.03) & 27.6 & (0.18) & 5.96 &
(0.02) & 22.0 & (0.14)
\\\hline
& \multicolumn{4}{c|}{$\beta_{(1,3_2,4_2)}$} &
\multicolumn{4}{c|}{$\beta_{(2,3_1,4_1)}$} &
\multicolumn{4}{c}{$\beta_{(2,3_2,4_2)}$}
\\\hline
0.02 & 271 & (2.67) & 359 & (3.58) & 143 & (1.36) & 156 & (1.51) & 274 & (2.69) &
362 & (3.69)
\\
0.05 & 90.2 & (0.80) & 286 & (2.81) & 23.8 & (0.14) & 30.6 & (0.21) & 92.5 & (0.83)
& 300 & (3.04)
\\
0.20 & 7.34 & (0.03) & 39.6 & (0.30) & 3.77 & (0.01) & 4.39 & (0.01) & 7.36 & (0.03)
& 44.5 & (0.35)
\\
$-0.02$ & 271 & (2.70) & 359 & (3.62) & 136 & (1.30) & 131 & (1.24) & 273 & (2.74) &
367 & (3.66)
\\
$-0.05$ & 91.4 & (0.81) & 314 & (3.17) & 21.9 & (0.13) & 26.8 & (0.19) & 91.8 &
(0.82) & 296 & (2.92)
\\
$-0.20$ & 7.60 & (0.03) & 102 & (0.94) & 3.25 & (0.01) & 3.98 & (0.01) & 7.48 &
(0.03) & 59.7 & (0.51)
\\\hline
\multicolumn{13}{l}{NOTE: Standard errors are in parentheses.}
\end{tabular}\vspace{-0.1cm}
\end{table}


The control limits of the MME chart are chosen by simulation so that each individual
chart has an identical IC ARL, jointly yielding the overall $\mbox{ARL}_0=370$.
Similar to the comparison between the LMC and MBE charts, we also present the
results where only one coefficient changes by adding a shift magnitude of $\delta$
and study the OC ARL performance. Some simulation results are listed in Table 2.3 in
the case of $\mu=0.1$ and $N=1,000$. From Table 2.3, we see that the MME chart shows
better performance than the LMC chart when main effects, e.g., $\beta_{(2)}$ and
$\beta_{(4_2)}$, change as we would expect. This is easy to understand, since each
of the individual charts constituting the MME chart collects adequately the one-way
marginal sums, which result directly from main effects. Therefore, the MME chart
stands out with a higher sensitivity to main effects than the LMC chart. The
superiority of the LMC chart over the MME chart becomes remarkable when two-factor
interaction effects such as $\beta_{(1,3_1)}$, $\beta_{(1,4_2)}$, $\beta_{(2,3_2)}$,
$\beta_{(2,4_1)}$, $\beta_{(3_2,4_1)}$, and $\beta_{(3_2,4_2)}$ are focused on, and
especially in the cases of changes in three-factor interaction effects, such as
$\beta_{(1,3_1,4_2)}$, $\beta_{(1,3_2,4_2)}$, $\beta_{(2,3_1,4_1)}$, and
$\beta_{(2,3_2,4_2)}$, where the MME chart is outperformed by the LMC chart by quite
a large margin. Like the comparison between the LMC chart and the MBE chart,
simulations under some other parameter settings also exhibit the same trends as in
Table 2.3. The effects of the Phase II sample size $N$ and the EWMA smoothing
parameter $\mu$ on the LMC chart are similar to those in multivariate binomial
processes.

Combined with the comparison between the LMC and MBE charts, it is affirmed again
that the superiority of the LMC chart lies in shifts in the interaction effects of
multiple factors that represent the dependence among them. In addition, it should be
noted that when there is only one factor, the LMC chart reduces to the EWMA version
of the log-likelihood ratio statistic, and the MME chart simplifies to the EWMA
version of the Pearson chi-square statistic. Their performance should be more or
less the same, since they enjoy the same asymptotic distribution under the null
hypothesis as a central $\chi^2$ with an appropriate degree of freedom. Hence, the
LMC chart can work well within the unified framework of the univariate/multivariate
binomial/multinomial processes.



\section{A Real Application}\label{sec2.5}

In this section, the proposed methodology is implemented in the aluminium
electrolytic capacitor (AEC) manufacturing process introduced in Section 1.1.2 to
demonstrate its real utilization. This may be regarded as a typical example to apply
the LMC chart in practice.

We consider the quality in the aging stage, where the AEC quality monitoring is
concentrated on three most important quality characteristics: leakage current (LC),
dissipation factor (DF), and capacity (CAP). Each characteristic can be evaluated as
conforming or nonconforming to its specification by an electronic device at a very
high speed. Obtaining their precise numerical values is possible, but this will cost
too much. To this end, for a specific sample size in Phase II, this can be regarded
as a multivariate binomial process, which has three factors including LC, DF, and
CAP all with two levels. The cross-classification counts with all factor level
combinations are stored in a three-way contingency table with $2^3=8$ cells. Based
on the above information, the log-linear model is formulated as
\[
\ln p_{ijk}=\beta_0+\beta_{(1)}x_1+\beta_{(2)}x_2+\beta_{(3)}x_3+
\beta_{(1,2)}x_1x_2+\beta_{(1,3)}x_1x_3+\beta_{(2,3)}x_2x_3+\beta_{(1,2,3)}x_1x_2x_3,
\]
where $i=1,2;j=1,2;k=1,2$ and all the cell count probabilities $p_{ijk}$ sum up to
1. In addition, $x_1$, $x_2$, and $x_3$ take 1 or $-1$ representing the two levels
of three factors LC, DF, and CAP, respectively. This log-linear can adequately
characterize the relationship between the cell counts and these level combinations.
Therefore, the proposed LMC chart can be adopted to monitor the three quality
characteristics simultaneously in the aging stage.

It is known that each workbench in the aging stage manufactures at least 6,000 AEC
elements every day, and that the three quality characteristics LC, DF, and CAP for
each AEC element are then inspected automatically as conforming or nonconforming by
an electronic device. We perform model estimation from a historical IC dataset
[2,1,19,12,1,75,732,39447] with about 40,000 observations, which contains the 8 cell
counts and could have been completed by one workbench within only a few days. By
variable selection, the IC model has a hierarchy structure [LC CAP][DF CAP].
Furthermore, the estimated coefficient vector is given by
$\wt{\bm\beta}^{(0)}$=[$\beta_0$, 0.88, 2.11, 1.91, 0.00, 1.11, 1.02, 0.00]$^T$. In
addition, the AECs are usually inspected in a batch of 500, hence we have a Phase II
sample size $N=500$. Based on this, the IC cell probability vector
\[
{\bf p}^{(0)}=[0.4599,0.2847,4.752,2.942,0.3435,18.52,181.6,9791]^T\times 10^{-4}.
\]

In Phase II, the EWMA smoothing parameter $\mu$ is chosen to be 0.1. We obtain the
control limit of the LMC chart, which is 0.83, by simulation, such that the IC ARL
is 370. Now the LMC chart is ready to be constructed to monitor the process. After
obtaining new observations, we calculate the charting statistics $U_k$ for each
sample, then plot them in the control chart, and compare them with the control
limit. We take the 9th point for example to illustrate the calculation of charting
statistics. Based on ${\bf p}^{(0)}$ and the original observation vectors ${\bf
n}_1,\ldots,{\bf n}_9$, for example, ${\bf n}_9=[0, 0, 0, 0, 0, 6, 10, 484]^T$, we
first calculate the pseudo-observation vector
\[
{\bf z}_9=[0.89090, 0.55151, 22.598, 26.403, 0.66537, 133.15, 873.51, 48942]^T\times
10^{-2}
\]
using Equation (\ref{F2.7}). Then we use the IPF algorithm based on the hierarchy
structure [LC CAP][DF CAP] to get the MLE of the cell probability vector over ${\bf
z}_9$, which is
\[
\widehat{\bf q}_9=[0.1343, 0.1542, 4.563, 5.237, 0.4684, 26.29, 174.4, 9789]^T\times
10^{-4}.
\]
Finally, by Equation (\ref{F2.8}), its charting statistic $R_9$ is calculated as
0.25332. Figure 2.2 shows the resulting LMC chart (solid curve connecting the dots),
along with its control limit (solid horizontal line). The LMC chart signals at the
28th observation and remains above the control limit in the remainder of the
sequence.

\begin{figure}[ht]
\begin{center}
\includegraphics[width=14.0cm,height=10.0cm]{fig2-2.ps}
\vspace{-0.7cm} \caption{An LMC control chart for monitoring the AEC
process}\vspace{-0.3cm}
\end{center}
\end{figure}

By comparison, we also adopt the MBE chart to monitor the AEC process with
ARL$_0=370$, $N=1,000$, and $\mu=0.1$, which is shown in Figure 2.3. The MBE chart
also triggers an OC signal at the 28th sample. However, it is difficult to say which
chart performs better based on only this single run. The MBE chart enjoys a charting
statistic of a simpler form, but it cannot provide the practical interpretation of
shifts according to the one-to-one correspondence between factor effects and
coefficient subvectors in a log-linear model. As indicated earlier, this
correspondence may exploit insights into multivariate binomial/multinomial processes
and assist in further diagnosis.

\begin{figure}[ht]
\begin{center}
\includegraphics[width=14.0cm,height=10.0cm]{fig2-3.ps}
\vspace{-0.7cm} \caption{An MBE control chart for monitoring the AEC
process}\vspace{-0.3cm}
\end{center}
\end{figure}



\section{Summary}\label{sec2.6}

This chapter proposes a Phase II control chart, namely the log-linear multivariate
categorical control chart, which can be actualized as a general SPC tool for the
monitoring of multivariate/univariate binomial/multinomial data. The LMC chart
adopts the EWMA control scheme in terms of the exponentially weighted
pseudo-observation vector in Phase II, which exploits the information of past and
current sampling cells adequately and distinguishingly as well as mitigates the
potential tendency of sparsity in a multi-way contingency table. In comparison with
the existing approaches, numerical simulations demonstrate that at certain expenses
of sensitivity to changes in main effects, the LMC chart is much more robust than
traditional ones, which take only marginal cell probability sums into account.
Instead, the LMC chart provides much higher detection ability to possible shifts in
interaction effects of multiple factors which represent their dependence. In
addition, the AEC application shows that the LMC chart can be implemented
effortlessly into real manufacturing and even service industries.

As pointed out previously, we assume in this work that the model hierarchy
structures under the IC and OC conditions are the same. However, this is not always
the case. Our ongoing research focuses on monitoring simultaneously possible changes
of both the model hierarchy structure and the coefficient vector, such that the
potential shifts can be detected and diagnosed more accurately and efficiently.
Moreover, the log-linear model may explode when the numbers of factors or their
levels are large. As mentioned in Section 2.2, a reduced model via variable
selection is desired accordingly. In such cases, our proposed procedure is still
applicable. As we can expect, the performance of the proposed chart is affected by
the amount of data in the reference dataset, especially for a large number of
factors or their levels. Very large Phase I samples must be collected for the LMC
chart to perform as well as those with known parameters. So determining the Phase I
sample size required to remove the effects of estimated parameters is critical and
warrants future research.

\chapter{Log-Linear Directional Chart}\label{chp3}

\section{Introduction}\label{sec3.1}

The LMC chart presented in Chapter 2 is a general and robust tool for detecting
various changes in multivariate categorical processes. However, it fails to consider
some practical information in real applications and therefore may be not so powerful
for detecting changes with certain patterns. This chapter illustrates a systematic
directional monitoring mechanism and a diagnostic scheme for multivariate
categorical processes. We have known that there is a one-to-one correspondence
between factor effects and coefficient subvectors in a log-linear model. Therefore,
potential shifts to OC states in factor effects appear in their corresponding
coefficient subvectors, which provides prior information on some potential shift
directions. To monitor a process as efficiently as possible, such practical
information should be exploited. For these purposes, still based on the
log-likelihood function of a log-linear model and the EWMA scheme, we propose the
log-linear directional (LLD) control chart, which considers directional shift
information. This chart also applies to the unified framework of
univariate/multivariate binomial/multinoial processes and has the superiority in
detecting one-coefficient shifts and high-order interaction shifts. Furthermore, we
present a diagnostic scheme to identify the shift direction following an OC signal.
Some implementation guidelines are also provided and illustrated using the AEC
example. Monte Carlo simulations demonstrate the effectiveness of the LLD chart and
the diagnostic approach.



\section{Log-Linear Models Revisited}\label{sec3.2}

The LLD chart is still based on log-linear models. The log-linear model (\ref{F2.2})
is at the effect level, but it can be rewritten equivalently at the coefficient
level as
\begin{equation}
\ln {\bf p}=\bm{1}\beta_0+{\bf X}\bm\beta=\bm{1}\beta_0+\sum_{i=1}^{h-1}{\bf
x}_i\beta_i,\label{F3.1}
\end{equation}
where ${\bf x}_i$ is the $i$th column vector of the matrix ${\bf X}$, and the scalar
$\beta_i$ is its corresponding coefficient (i.e., the $i$th element of $\bm\beta$).
For instance, in the three-way contingency table of size $2\times 3\times 3$
mentioned in Chapter 2, $\beta_1=\beta_{(1)}$, $\beta_{9}=\beta_{(1,3_2)}$, and
$\beta_{17}=\beta_{(1,2_2,3_2)}$. We see ${\bf X}=[{\bf X}_1,\ldots,{\bf
X}_{2^p-1}]=[{\bf x}_1,{\bf x}_2,\ldots,{\bf x}_{h-1}]$ and
$\bm\beta=[\bm{\beta}_1^T,\ldots,\bm{\beta}_{2^p-1}^T]^T=
[\beta_1,\ldots,\beta_{h-1}]^T$. There is also a correspondence between the $i$th
column ${\bf x}_i$ of ${\bf X}$ and the $i$th coefficient $\beta_i$
($i=1,\ldots,h-1$). By analogy with a linear regression model, the log-linear model
is also essentially a regression model with the probabilities as the responses, the
coefficients as the regressors, and the column vectors composing the design matrix.

The construction of the LLD chart requires the column vectors ${\bf x}_i$
($i=1,\ldots,h-1$), or equivalently we need to derive the design matrix $\bf X$.
This is simple for a multivariate binomial process, where factors all have two
levels. For a general multivariate categorical process, this may be a little
complex. Refer to the additional File 1 in Dahinden et al. (2007) for the general
result of deriving $\widetilde{{\bf X}}=[\bm1,{\bf X}]$. Here, we take four factors
$C_1$, $C_2$, $C_3$, and $C_4$ with 2, 2, 3 and 3 levels, respectively, for
illustration. Let
\[
\mathbf{1}_2=\left[\begin{array}{r}1\\1\end{array}\right],\quad
\mathbf{1}_3=\left[\begin{array}{r}1\\1\\1\end{array}\right],\quad {\bf
J}_2=\left[\begin{array}{r}1\\-1\end{array}\right],\quad {\bf
J}_3=\left[\begin{array}{rr}1&0\\0&1\\-1&-1\end{array}\right]
=\left[\begin{array}{c}{\bf I}_2\\-\mathbf{1}_2^T\end{array}\right].
\]
Note that the column sums of matrixes ${\bf J}_2$ and ${\bf J}_3$ are all zeros,
which assures identifiability. For instance, the design submatrix corresponding to
the main effect of $C_3$ is $\mathbf{1}_2\otimes\mathbf{1}_2\otimes{\bf
J}_3\otimes\mathbf{1}_3$, where $\otimes$ is Kronecker product operator. The design
submatrix corresponding to the two-factor interaction effect of $C_2C_4$ is
$\mathbf{1}_2\otimes{\bf J}_2\otimes\mathbf{1}_3\otimes{\bf J}_3$, and the design
submatrix corresponding to the three-factor interaction effect of $C_1C_3C_4$ is
${\bf J}_2\otimes\mathbf{1}_2\otimes{\bf J}_3\otimes{\bf J}_3$. All the other design
submatrixes can be constructed similarly. In a word, given
$\mathbf{1}_2\otimes\mathbf{1}_2\otimes\bm{1}_3\otimes\mathbf{1}_3$, the design
submatrix corresponding to an effect is obtained by replacing $\bm{1}$ with ${\bf
J}$ at all the positions where the factors are contained in this effect.

%Phase II control charts require that the IC process parameters be known, which needs
%estimating the coefficient vector $\widetilde{\bm{\beta}}$ in the log-linear model
%(\ref{F2.2}) or (\ref{F3.1}) from an IC dataset. But as will be seen next, some
%simple approximations allow using only the probability vector ${\bf p}^{(0)}$ in the
%IC state, skipping the estimation of the coefficient vector
%$\widetilde{\bm{\beta}}^{(0)}$.



\section{Log-Linear Directional Monitoring}\label{sec3.3}

In a log-linear model, the marginal distribution of one factor is mainly determined
by its main effect, whereas the dependence among multiple factors is represented by
their interaction effect. This provides a means of interpreting shifts in
multivariate categorical processes. According to the one-to-one correspondence
between factor effects and coefficient subvectors in a log-linear model, shifts in
the marginal distribution of one factor lead to deviations of the coefficient
subvector corresponding to its main effect, and shifts in the dependence among
multiple factors result in deviations of the coefficient subvector reflecting their
interaction effect.

The pre-specified log-linear model $F\big({\bf X}; \bm\beta\big)$ is summarized in
Equation (\ref{F2.3}) as
\[
\ln{\bf p}=\widetilde{{\bf X}}\widetilde{\bm{\beta}}=\bm 1\beta_0+{\bf
X}\bm\beta\quad \mathrm{and}\quad\bm1^T{\bf p}=1.
\]
We still assume that the $j$th online multivariate sampling observation vector ${\bf
n}_j$, of size $h\times 1$, follows a multinomial distribution with a total size $N$
and behaves over time according to the change-point model (\ref{F2.4})
\begin{align*}
{\bf n}_j\ {\mathop{\sim}\limits^{\mbox{i.i.d.}}}\left\{\hspace{-0.1cm}
\begin{array}{ll} F\big({\bf X};\bm{\beta}^{(0)}\big), &
{\mathrm{for\ }} \quad j=1,\ldots,\tau,\\[0.1cm]
F\big({\bf X};\bm\beta^{(1)}\big), & {\mathrm {for\ }} \quad j=\tau+1,\ldots,
\end{array}\right.
%\label{F3.4}
\end{align*}
where $\tau$ is the unknown change point, and $\bm\beta^{(0)} \neq \bm\beta^{(1)}$
are the known IC and unknown OC process coefficient vectors, respectively.

Since $\beta_0$ can be determined from $\bm{\beta}=[\beta_1,\ldots, \beta_{h-1}]^T$,
the monitoring problem can be formulated as a hypothesis testing problem:
\begin{align}\label{F3.2}
H_0:\;\bm{\beta}=\bm{\beta}^{(0)}\ \ \mbox{versus}\ \
H_1:\;\bm{\beta}\ne\bm{\beta}^{(0)}.
\end{align}
A natural test for hypothesis (\ref{F3.2}) can be constructed by using the idea of a
generalized likelihood ratio test (GLRT; Anderson (2003)), which incorporates all
possible shifts in $\bm{\beta}^{(0)}$, and therefore is quite general. In fact, this
idea leads to the LMC chart proposed in Chapter 2.

There is a correspondence between the $i$th main or interaction effect and the
coefficient subvector $\bm{\beta}_i$. According to the effect sparsity principle (Wu
and Hamada (2000)), which states that the number of relatively important effects is
small in DOE, it is reasonable to assume that in practical applications any changes
involve only a few coefficient subvectors or only a few coefficients in the
appropriate model. Suppose that we have some {\it a priori} knowledge that in the OC
state only one coefficient $\beta_i$ ($1\leq i\leq h-1$) is incremented by an
unknown constant $\delta_i$. The hypothesis then becomes
\begin{align*}
H_0:\;\bm{\beta}=\bm{\beta}^{(0)}\ \ \mbox{versus}\ \
H_1:\;\bm{\beta}=\bm{\beta}^{(0)}+{\bf d}_i\delta_i,
\end{align*}
where ${\bf d}_i$ is the direction vector of size $(h-1)\times 1$ with a 1 as its
$i$th component and 0s elsewhere.

Next consider the more practical case that in the OC state only one coefficient
changes, but its location is unknown. The alternative in hypothesis (\ref{F3.2})
reduces to
\begin{align}
H_1:\;\bm{\beta}=\bm{\beta}^{(0)}+{\bf d}_1\delta_1
\;\mathrm{or}\;\;\bm{\beta}=\bm{\beta}^{(0)}+{\bf d}_2\delta_2\ldots\;
\mathrm{or}\;\;\bm{\beta}=\bm{\beta}^{(0)}+{\bf d}_{h-1}\delta_{h-1},\label{F3.3}
\end{align}
where $\delta_i$ ($i=1,2,\ldots,h-1$) are the unknown shift magnitudes, and the
possible shift directions ${\bf d}_1$, ${\bf d}_2$, \ldots, ${\bf d}_{h-1}$ are
defined in a similar way, applying to $\beta_1$, $\beta_2$, \ldots, $\beta_{h-1}$,
respectively. The GLRT derived from hypothesis (\ref{F3.3}) should be better than
that from hypothesis (\ref{F3.2}) because it makes full use of more constructive
information about potential shift directions. In fact, a GLRT based on hypothesis
(\ref{F3.3}) that incorporates directional knowledge about potential changes is very
much like the GLRTs used in multistage process monitoring and diagnosis which
exploit the information of shift directions from the first stage to the last one
(see Zou and Tsung (2008) and Zou et al. (2008)).

Hypothesis (\ref{F3.3}) may be further generalized. There is also the hierarchical
ordering principle in DOE (see Wu and Hamada (2000)), which states that lower order
effects are more likely to be important than higher order ones and that effects of
the same order are equally likely to be important. So deviations involving fewer
factors may appear more frequently, and it is reasonable to believe that in
applications most shifts involve lower-order effects rather than higher-order ones.
Unlike hypothesis (\ref{F3.3}), which considers all one-coefficient shifts from the
main factor effects up to the highest $p$-factor interaction effect, instead we may
focus on effects involving the first few, say $q$, orders. Denote the set of
coefficient indexes among the effects of the first $q$ orders by $\mathbb{I}_q$
($1\leq q\leq p$). Then, hypothesis (\ref{F3.3}) includes all the coefficients
indexed by $\mathbb{I}_p$. Then hypothesis (\ref{F3.3}) can be further extended as
\begin{align}
H_0:\;\bm{\beta}=\bm{\beta}^{(0)}\ \ \mbox{versus}\ \
H_1:\;\bigcup_{i\in\mathbb{I}_q}\Big(\bm{\beta}=\bm{\beta}^{(0)}+{\bf
d}_i\delta_i\Big),\quad 1\leq q\leq p.\label{F3.4}
\end{align}
With $q=p$, hypothesis (\ref{F3.4}) becomes hypothesis (\ref{F3.3}). Consider the
case that all $p$ factors have two levels to illustrate the advantage of considering
$\mathbb{I}_2$ instead of $\mathbb{I}_p$. With the first two orders, only
$\mathrm{C}_p^1+\mathrm{C}_p^2=p(p+1)/2$ one-coefficient shifts are taken into
account. However, this number becomes $2^p-1$ for $\mathbb{I}_p$, increasing
exponentially with $p$. Since shifts to OC states usually involve low-order effects,
the larger $q$ is, the less powerful the GLRT will be. If the real shift indeed
arises from the effects of the first $q$ orders, the GLRT based on hypothesis
(\ref{F3.4}) will certainly be more powerful. Even if a shift occurs in an effect of
an order higher than $q$, however, this change may be reflected to a large extent in
the charting statistic indexed by $\mathbb{I}_q$. So a GLRT derived from hypothesis
(\ref{F3.4}) may still be powerful, as we will show later.

Now we give the GLRT statistic for testing hypothesis (\ref{F3.4}). With the Phase
II sample size $N$ and the observation vector ${\bf n}$ of size $h\times 1$
satisfying $\bm1^T{\bf n}=N$, the $-2$LRT (log-likelihood ratio test) statistic for
testing hypothesis (\ref{F3.4}) is
\begin{equation}
Q=\max_{i\in\mathbb{I}_q}\Big(\frac{1}{N}\big({\bf n}-N{\bf p}^{(0)}\big)^T{\bf
x}_i\big({\bf x}_i^T\bm{\Sigma}^{(0)}{\bf x}_i\big)^{-1}{\bf x}_i^T\big({\bf
n}-N{\bf p}^{(0)}\big)\Big).\label{F3.5}
\end{equation}
Its derivation is placed in Section 3.8.1, the appendix in this chapter. Here ${\bf
p}^{(0)}$ is the known IC cell probability vector, and
$\bm{\Sigma}^{(0)}=\di\big({\bf p}^{(0)}\big)-{\bf p}^{(0)}({\bf p}^{(0)})^T$ is the
IC covariance matrix (see Agresti (2002)), where $\di({\bf a})$ is the diagonal
square matrix with its diagonal elements equal to the elements of the column vector
${\bf a}$. The test will reject the null hypothesis if $Q$ is larger than some
specified threshold.

The test statistic $Q$ in Equation (\ref{F3.5}) can be used to construct a
Shewhart-type control chart for online monitoring. However, this method merely
utilizes the information in the current sample, and it will be very inefficient for
moderate and small changes. To be more efficient, we combine hypothesis (\ref{F3.4})
with an EWMA scheme to properly exploit the information in past and current samples.
Denote the observation vector of the $j$th sample of size $N$ in Phase II by ${\bf
n}_j$. For any time point $k$, consider the following exponentially weighted sum of
the previous and current observation vectors ${\bf n}_j$ ($j=1,\ldots,k$)
\[
{\bf z}_k=a_{0,k,\mu}^{-1}\sum_{j=1}^{k}(1-\mu)^{k-j}{\bf n}_j,
\]
which is the same as in Equation (\ref{F2.7}). Thus the EWMA counterpart of the GLRT
statistic $Q$ is
\begin{align}
V_k=\max_{i\in\mathbb{I}_q}\Big(\frac{1}{N}\big({\bf z}^{(k)}-N{\bf
p}^{(0)}\big)^T{\bf x}_i\big({\bf x}_i^T\bm{\Sigma}^{(0)}{\bf x}_i\big)^{-1}{\bf
x}_i^T\big({\bf z}^{(k)}-N{\bf p}^{(0)}\big)\Big).\label{F3.6}
\end{align}

Here $V_k$ can be regarded as the maximum of a series of quadratic forms indexed by
$\mathbb{I}_q$, and it takes into account all the combined directions of potential
shifts as well as the exponentially weighted past and current online samples.
Analogous to $Q$, a large value of $V_k$ will reject the null hypothesis in
(\ref{F3.4}). Therefore, we take $V_k$ to be the charting statistic for a control
chart, which will trigger an OC signal at time $k$ if $V_k>L$, where $L>0$ is a
control limit chosen to attain a specific IC ARL. Note that the steady-state
covariance of ${\bf z}_k$ is $\frac{\mu}{2-\mu}\bm\Sigma^{(0)}$, but the constant
$\frac{\mu}{2-\mu}$ is implicitly absorbed into the control limit $L$ (which will be
determined via simulation).

Since such a chart combines the one-coefficient shifts, it represents the
directional changes of multivariate categorical processes, and we call it the
log-linear directional (LLD) control chart. From the derivation of the charting
statistic $V_k$, we see that it can also be modified as a robust one for detecting
other types of shifts, such as general shifts arising in all coefficients in a
log-linear model.



\section{Diagnostic Schemes}\label{sec3.4}

Assume that at most one coefficient $\beta_i^{(0)}$ in $\bm{\beta}^{(0)}$ will
change. This can be detected efficiently by the proposed LLD chart. Once the LLD
chart triggers an OC alarm, the question naturally arises which coefficient shift is
responsible for this signal. This diagnosis is especially important for multivariate
categorical processes, identifying the shift direction quickly and accurately.

Diagnosis is required only when there is an OC indication. Here the null hypothesis
in (\ref{F3.4}) has been rejected, and the diagnosis is based on the alternative
hypothesis. This alternative hypothesis considers the one-coefficient shifts of the
first $q$ orders ($1\leq q\leq p$), and the LLD chart monitors the process by
checking the charting statistic (\ref{F3.6}) indexed by $\mathbb{I}_q$, but it is
still possible that the real shift occurs in an effect of an order higher than $q$.
In other words, a coefficient $\beta_j^{(0)}$ ($j\notin\mathbb{I}_q$) may deviate.
In diagnosis, therefore, it is necessary to choose a candidate set of diagnostic
shift direction indexes larger than what we used in monitoring, lest the real shift
direction is left out. Thus, we consider one-coefficient shifts in effects of the
first $q^{\prime}$ orders ($1\leq q\leq q^{\prime}\leq p$) for identifying the shift
direction. As with $q$, the choice of $q^{\prime}$ will be discussed later.

Note that in OC states, the true probability vector and the corresponding covariance
matrix are no longer ${\bf p}^{(0)}$ and $\bm{\Sigma}^{(0)}$. Suppose that it is
only the deviation of the coefficient $\beta_i^{(0)}$ in $\bm{\beta}^{(0)}$ that
brings the process OC. Since the alternative hypothesis in (\ref{F3.4}) is assumed
to represent the OC state, the OC samples are therefore generated from $F\big({\bf
X};\bm\beta^{(1,i)}\big)$, where $\bm\beta^{(1,i)}$ is the OC coefficient vector
that deviates from the IC coefficient vector $\bm\beta^{(0)}$ only in $\beta_i$.
This OC model is indexed by $i\in\mathbb{I}_{q^{\prime}}$. Denote also the OC
probability vector by ${\bf p}^{(1,i)}$ and the corresponding OC covariance matrix
by $\bm{\Sigma}^{(1,i)}=\di\big({\bf p}^{(1,i)}\big)-{\bf p}^{(1,i)}\big({\bf
p}^{(1,i)}\big)^T$.

To determine the shift direction, one intuition is that some quadratic forms similar
to Equation (\ref{F3.6}) indexed by the set $\mathbb{I}_{q^{\prime}}$ instead of
$\mathbb{I}_q$ may help with identification. Since diagnosis is performed in the OC
state, the covariance matrix term contained in these quadratic forms should be
$\bm{\Sigma}^{(1,i)}$ instead of $\bm{\Sigma}^{(0)}$. If the $i$th quadratic form
has some kind of property that a one-to-one correspondence exists between it and the
current $i$-indexed OC state model $F\big({\bf X};\bm\beta^{(1,i)}\big)$, the
diagnosis can then exploit this relationship. Suppose that ${\bf n}$ is an
observation vector of size $N$ collected from $F\big({\bf X};\bm\beta^{(1,i)}\big)$.
Define
\begin{equation}
S_j=\frac{1}{N}\big({\bf n}-N{\bf p}^{(0)}\big)^T{\bf x}_j\big({\bf
x}_j^T\bm{\Sigma}^{(1,i)}{\bf x}_j\big)^{-1}{\bf x}_j^T\big({\bf n}-N{\bf
p}^{(0)}\big).\label{F3.7}
\end{equation}
It is shown in the appendix in this chapter (Section 3.8.2) that
$\arg\max_{j\in\mathbb{I}_{q^{\prime}}}S_j$ is consistent with the real shift
direction index $i$. So there is indeed a probable correspondence between the
largest quadratic form $S_i$ ($i\in\mathbb{I}_{q^{\prime}}$) and the real shift
direction ${\bf d}_i$, and this provides some guidelines for diagnosis.

In applications, the OC probability vector ${\bf p}^{(1,i)}$ and the covariance
matrix $\bm{\Sigma}^{(1,i)}$ are unknown and need to be estimated. Suppose that the
LLD chart triggers an alarm at time $\eta$. Between the change point $\tau$ (but not
including $\tau$) and the signal point $\eta$, there have already been observation
vectors collected from $F\big({\bf X};\bm\beta^{(1,i)}\big)$. The best estimation of
${\bf p}^{(1,i)}$ seems to be the average of these observation vectors divided by
the sample size $N$. This, however, relates to estimating the change point $\tau$,
which will not be discussed here. Therefore, the exponentially weighted sum of
observation vectors ${\bf z}_{\eta}$ at $\eta$, which efficiently exploits the
information in the OC state, seems to be a reasonable choice for estimating ${\bf
p}^{(1,i)}$. To this end, the diagnostic procedures can be finalized as
\begin{description}
\item[Step 1] Define
$\widehat{{\bf p}}={\bf z}_{\eta}/N$ and
$\widehat{\bm{\Sigma}}=\di\big(\widehat{{\bf p}}\big)-\widehat{{\bf p}}\widehat{{\bf
p}}^T$ as the estimates of ${\bf p}^{(1,i)}$ and $\bm{\Sigma}^{(1,i)}$,
respectively.
\item[Step 2] Replace $\bm{\Sigma}^{(1,i)}$ with $\widehat{\bm{\Sigma}}$ and ${\bf n}$
with ${\bf z}_{\eta}$ in $S_j$ ($j\in\mathbb{I}_{q^{\prime}}$) in Equation
(\ref{F3.7}), calculate their values, and denote them by $s_j$
($j\in\mathbb{I}_{q^{\prime}}$).
\item[Step 3] Find the maximum among these $s_j$, and its index is taken to be the
direction index $\zeta$.
\end{description}



\section{Practical Implementation and Application}\label{sec3.5}

\subsection{Design Parameter Settings}

\noindent{\it On the parameters $q$ and $q^{\prime}$}: The most likely shift
directions are confined to the effects of the first $q$ orders ($1\leq q\leq p$)
when there are $p$ factors. The GLRT (\ref{F3.4}) will be less powerful if more
shift directions are included when in actuality most shifts occur in lower-order
effects. If $q$ is selected as 1, however, only main effect deviations are
considered, ignoring correlations between factors, which is obviously inappropriate.
In most real applications, one cares about only means and variances, i.e., moments
of the first two orders. Furthermore, the monitoring task is to only answer ``yes''
or ``no'' about whether the process is IC, rather than specifying shifted effects.
Here we should pay attention only to main effects and two-factor interactions and
choose $q$ as 2, focusing on coefficients corresponding to effects of the first two
orders. If the shift indeed arises in a main or two-factor interaction effect, the
GLRT (\ref{F3.4}) with $q=2$ should be powerful. Even if a shift arises in a
three-factor or higher-order interaction effect, the effects of the first two orders
will be influenced to a fairly large extent, so the OC condition may still be
detected quickly by the LLD chart.

The choice of the diagnostic parameter $q^{\prime}$ should be considered in
combination with the monitoring parameter $q$. As has been indicated earlier,
$q^{\prime}$ should be at least as large as $q$ with $q^{\prime}\geq q$, to help
prevent important shift directions from being missed. Since $q=2$ has been
recommended, $q^{\prime}$ should be at least 3, confining the candidate subset of
diagnostic shift directions to effects in the first three orders. This will still be
safe in case a shift appears in a three-factor interaction effect, and it is
believed that shifts in four-factor or even higher-order interaction effects are
rare. As in monitoring, as $q^{\prime}$ increases, the diagnostic accuracy will
decrease. Therefore, $q^{\prime}=3$ is recommended.

In fact, going a step further, we can choose these index sets for monitoring and
diagnosis to contain only the factor effects or coefficients in which we are
interested. Generally, any subset of the index set $\{1,\ldots,h\}$ can be
introduced, with its elements corresponding to the coefficients associated with the
most likely shifts. One should keep in mind that the index set for diagnosis must be
at least as large as the one for monitoring.

\noindent{\it On the smoothing parameter $\mu$}: Generally, a smaller $\mu$ assists
in detecting smaller shifts more quickly, while a larger $\mu$ leads to quicker
detection of larger shifts (see Lucas and Saccucci (1990) and Lowry et al. (1992)).
This still applies to LLD charts, as will be confirmed by Figure 3.2 in the next
section. In diagnosis, a smaller $\mu$ results in better diagnostic consistency for
smaller shifts, while a larger $\mu$ helps in identifying the locations of larger
shifts. This will also be illustrated in Figure 3.3 below. Empirical results show
that $\mu$ between 0.05 and 0.2 may be a reasonable choice.

\noindent{\it On the sample size $N$}: With a small sample size $N$, to achieve a
small OC ARL, a very large magnitude $\delta$ is required. On the other hand, a
large sample size contains more information about observation vectors such as ${\bf
n}_k$. To detect a small shift as quickly as possible, a very large sample size is
needed. Therefore, based on permissible costs or other considerations, we should
select a sample size $N$ as large as possible.

In fact, the sample size for multivariate categorical processes cannot be compared
with that of multivariate continuous processes, which is relatively small, say, only
several or tens. This follows because the data in multivariate categorical processes
are actually the proportions in each cell of a multi-way contingency table, which
are acquired by aggregating multivariate categorical observations. Therefore,
multivariate categorical processes are different in nature from multivariate
continuous ones. Since there is no need of accurate measurements for collecting
categorical data, it usually does not cost as much. So a relatively large sample
size in multivariate categorical processes is acceptable.

In this paper, for simplicity we assume that the sample size $N$ is fixed. For
time-varying sample sizes, in short, if the distribution of sample sizes can be
specified or their trend can be predicted, we can calculate the control limit by
simulation before monitoring. Otherwise, it seems that a time-varying control limit
that depends on the online sample size $N_k$ is required. We can determine the
control limit for each sample point $k$ with Monte Carlo simulation by drawing large
samples from the multinomial distribution $\mbox{MN}(N_k;{\bf n}_k/N_k)$, making the
conditional probability (given there is no alarm up to time point $k-1$) attain
$1/{\rm ARL}_0$. However, we do not investigate the time-varying sample size case
here, and we leave it for future work.

\noindent{\it On computation}: Once the IC probability vector ${\bf p}^{(0)}$ has
been given, the charting statistic $V_k$ of the LLD chart is relatively simple to
compute. Finding the control limit $L$ for a given IC ARL (denoted ARL$_0$) is
straightforward. Using a Pentium 3.0GHz CPU the search procedure based on 10,000
replicated simulations takes only a few minutes using bisection search when
ARL$_0=370$, $h=32$, $\mu=0.1$, and $N=1,000$.

\subsection{Implementation --- a Practical Example}

Implementation of the proposed methodology is demonstrated in this subsection by
revisiting the AEC manufacturing process presented in Section 1.1.2. This is a
multivariate binomial process with three factors LC, DF and CAP, each with two
levels. A three-way contingency table can represent the cross-classifications of the
three factors in eight cells. The relationship between the cell counts and the
factor level combinations can be well characterized by a log-linear model. An LLD
chart can be used to monitor the three quality characteristics simultaneously, and
the suggested diagnostic method can identify the fault location whenever there is an
OC alarm.

\noindent{\bf Step 1}: {\it Define the multivariate categorical process and derive
the design matrix $\widetilde{{\bf X}}$}. This AEC manufacturing process has $p=3$
factors each with 2 levels. Denote the three factors LC, DF, and CAP as $C_1$,
$C_2$, and $C_3$, respectively. The matrix $\widetilde{{\bf X}}$ can be formulated
in the manner of performing a $2^3$ full factorial experiment, and then arranged in
the sequence of 1, $C_1$, $C_2$, $C_3$, $C_1C_2$, $\ldots$, $C_1C_2C_3$.

\noindent{\bf Step 2}: {\it Obtain the Phase I IC probability vector ${\bf
p}^{(0)}$}. Every day thousands of AECs pass through each workbench in the aging
stage, and their LC, DF, and CAP are inspected and classified as conforming or
nonconforming. The IC cell probability vector ${\bf p}^{(0)}$ is estimated by
variable selection and parameter estimation from an IC dataset consisting of about
sixty thousand observations, which is
%where the eight level combination
%counts are summarized as [9, 6, 65, 43, 8, 259, 1830, 61038].
\[
{\bf p}^{(0)}=[1.423, 0.9485, 10.28, 6.798, 1.265, 40.94, 289.3, 9649]^T\times
10^{-4}.
\]
The IC covariance matrix is then $\bm{\Sigma}^{(0)}=\di\big({\bf p}^{(0)}\big)-{\bf
p}^{(0)}({\bf p}^{(0)})^T$.

\noindent{\bf Step 3}: {\it Find the control limit $L$ for a given} ARL$_0$.
Usually, an ARL$_0$ of 370 is used in the AEC process. The LLD chart monitors the
process by focusing on shifts in main effects and two-factor interaction effects
with $q=2$. An EWMA smoothing parameter $\mu=0.1$ and a Phase II sample size $N=500$
are considered appropriate. The control limit $L$ is set to be 0.56 through a
bisection search based on 10,000 simulations, achieving ARL$_0=370$.

\noindent{\bf Step 4}: {\it Phase II monitoring}. The cell counts of the eight level
combinations are tabulated for each sample of 500. For the $k$th sample, the eight
cell counts are recorded in the observation vector ${\bf n}_k$, and based on the
observation vectors up to time $k$, the exponentially weighted sum ${\bf z}_k$ is
calculated with the smoothing parameter 0.1 using Equation (\ref{F2.7}). The
charting statistic $V_k$ is then calculated using Equation (\ref{F3.6}). As the
process proceeds, the charting statistics $V_k$ ($k=1,2,\ldots$) are plotted in a
control chart and compared with the control limit $L$. A typical plot is shown in
Figure 3.1. In the example the chart signals OC at the 35th sample and remains above
the control limit for the remainder of the sequence.

\begin{figure}[ht]
\begin{center}
\includegraphics[width=16.0cm,height=9.0cm]{fig3-1.eps}
\vspace{-0.7cm} \caption{An LLD control chart for monitoring the AEC
process}\vspace{-0.3cm}
\end{center}
\end{figure}

\noindent{\bf Step 5}: {\it Identify the shift direction once an OC signal is
triggered}. We follow the three diagnostic steps stated previously. Based on ${\bf
z}_{35}$, the OC probability vector is estimated as
\[
\widehat{{\bf p}}=\frac{{\bf z}_{35}}{N}=[4.970, 0.1833, 12.37, 5.208, 3.011, 41.23,
342.1, 9591]^T\times 10^{-4},
\]
so the OC covariance matrix may be estimated as
$\widehat{\bm{\Sigma}}=\di\big(\widehat{{\bf p}}\big)-\widehat{{\bf p}}\widehat{{\bf
p}}^T$. As recommended before, the candidate subset of the fault directions is
selected as coefficients in the effects of the first three orders with
$q^{\prime}=3$. The direction estimator chooses the maximum among the values of the
seven quadratic forms $s_j$ ($j=1,\ldots,7$), which are 0.02, 0.02, 0.52, 0.01,
0.40, 0.39, and 0.40 corresponding to the seven coefficients $\beta_{(1)}$,
$\beta_{(2)}$, $\beta_{(3)}$, $\beta_{(1,2)}$, $\beta_{(1,3)}$, $\beta_{(2,3)}$, and
$\beta_{(1,2,3)}$, respectively. Thus, there may have been a shift associated with
the coefficient $\beta_{(3)}$, namely in the main effect of factor $C_3$, i.e., the
main effect of CAP.



\section{Performance Comparison and Assessment}\label{sec3.6}

We now test the performance of the proposed LLD chart and the diagnostic approach
through Monte Carlo simulations. Both multivariate binomial processes and
multivariate multinomial processes are simulated and the performance of the LLD
chart is compared with competing techniques. Throughout the simulations, ARL$_0$ is
fixed at 370, and all ARL values are averages over 10,000 replicated simulations.

\subsection{Monitoring a Multivariate Binomial Process}

With a multivariate binomial process, the LLD chart can be compared with the
multivariate binomial EWMA (MBE) control chart as presented in Section 2.4.1, which
is the EWMA version of the $\chi^2$-chart proposed by Patel (1973). Assume that
during a production process five quality characteristics are each assessed as
conforming or nonconforming, yielding a multivariate binomial process, which can be
arranged into a five-way contingency table with two levels for each factor. The IC
log-linear model has the coefficient vector
\begin{eqnarray}
\begin{array}{rrrrrrrrrrrl}
\widetilde{\bm{\beta}}^{(0)} & = & [ & \beta_0 & 0.72 & 0.93
& 0.49 & 0.25 & 0.47 & -0.57 & 0.22 & \\
& & & 0.11 & -0.14 & 0.15 & -0.16 & 0.41 & 0.16 & -0.19 & 0.33 & \\
& & & 0.39 & 0.10 & 0.07 & -0.05 & 0.21 & -0.02 & 0.45 & 0.33 & \\
& & & 0.08 & 0.27 & 0.04 & -0.13 & 0.07 & -0.07 & 0.03 & 0 & ]^T,
\end{array}
\nonumber
\end{eqnarray}
where $\beta_0$ is the intercept accommodating the constraint $\bm1^T{\bf p}=1$.
Based on $\widetilde{\bm{\beta}}^{(0)}$, the IC probability vector ${\bf p}^{(0)}$
can be further calculated. It is believed that possible OC shifts appear in factor
effects, and correspondingly in coefficients of the log-linear model. According to
our design recommendation, we construct the LLD chart with $q=2$ in its charting
statistic (\ref{F3.6}), again assuming that most shifts will arise in main effects
or two-factor interaction effects.

%Table 1
\begin{table}[!ht]
\tabcolsep 4.5pt \vspace{-0.1cm} \centering \caption{OC ARL comparison between the
LLD and MBE charts for one-coefficient shifts of the first two orders}
\vspace{0.3cm}
\renewcommand{\arraystretch}{1.25}
\begin{tabular}{r|cccc|cccc|cccc}\hline
$\delta$ & \multicolumn{2}{c}{LLD} & \multicolumn{2}{c|}{MBE} &
\multicolumn{2}{c}{LLD} & \multicolumn{2}{c|}{MBE} & \multicolumn{2}{c}{LLD} &
\multicolumn{2}{c}{MBE}
\\\hline
& \multicolumn{4}{c|}{$\beta_{(3)}$} & \multicolumn{4}{c|}{$\beta_{(5)}$} &
\multicolumn{4}{c}{$\beta_{(1,4)}$}
\\\hline
0.01 & 201 & (1.99) & 199 & (1.90) & 241 & (2.37) & 222 & (2.16) & 176 & (1.67) &
249 & (2.48)
\\
0.02 & 71.6 & (0.63) & 70.5 & (0.61) & 95.9 & (0.86) & 86.1 & (0.77) & 53.0 & (0.43)
& 117 & (1.07)
\\
0.05 & 13.2 & (0.07) & 13.6 & (0.07) & 16.3 & (0.09) & 16.5 & (0.09) & 10.3 & (0.05)
& 21.6 & (0.13)
\\
0.20 & 2.82 & (0.01) & 2.90 & (0.01) & 3.22 & (0.01) & 3.25 & (0.01) & 2.39 & (0.01)
& 3.79 & (0.01)
\\
$-0.01$ & 172 & (1.69) & 176 & (1.69) & 200 & (1.94) & 193 & (1.88) & 155 & (1.50) &
230 & (2.24)
\\
$-0.02$ & 60.5 & (0.51) & 61.5 & (0.52) & 79.0 & (0.70) & 74.3 & (0.66) & 46.8 &
(0.38) & 101 & (0.93)
\\
$-0.05$ & 11.6 & (0.06) & 12.0 & (0.06) & 14.1 & (0.07) & 14.2 & (0.08) & 9.35 &
(0.04) & 18.9 & (0.11)
\\
$-0.20$ & 2.18 & (0.01) & 2.23 & (0.01) & 2.40 & (0.01) & 2.42 & (0.01) & 1.96 &
(0.01) & 2.96 & (0.01)
\\\hline
& \multicolumn{4}{c|}{$\beta_{(2,3)}$} & \multicolumn{4}{c|}{$\beta_{(2,5)}$} &
\multicolumn{4}{c}{$\beta_{(3,4)}$}
\\\hline
0.01 & 193 & (1.87) & 247 & (2.44) & 262 & (2.55) & 318 & (3.15) & 216 & (2.07) &
290 & (2.85)
\\
0.02 & 66.0 & (0.56) & 108 & (1.00) & 110 & (1.00) & 201 & (1.90) & 77.0 & (0.67) &
158 & (1.50)
\\
0.05 & 12.7 & (0.06) & 20.2 & (0.12) & 18.0 & (0.10) & 47.3 & (0.38) & 13.8 & (0.07)
& 32.2 & (0.23)
\\
0.20 & 2.76 & (0.01) & 3.69 & (0.01) & 3.39 & (0.01) & 6.14 & (0.02) & 2.90 & (0.01)
& 4.79 & (0.02)
\\
$-0.01$ & 167 & (1.60) & 210 & (2.03) & 219 & (2.15) & 279 & (2.77) & 183 & (1.77) &
249 & (2.44)
\\
$-0.02$ & 58.0 & (0.50) & 93.3 & (0.86) & 90.1 & (0.82) & 165 & (1.59) & 66.0 &
(0.57) & 129 & (1.22)
\\
$-0.05$ & 11.3 & (0.06) & 17.2 & (0.10) & 15.2 & (0.08) & 36.4 & (0.28) & 12.0 &
(0.06) & 25.9 & (0.18)
\\
$-0.20$ & 2.15 & (0.01) & 2.75 & (0.01) & 2.50 & (0.01) & 4.16 & (0.01) & 2.22 &
(0.01) & 3.48 & (0.01)
\\\hline
\multicolumn{13}{l}{NOTE: Standard errors are in parentheses. $\mu=0.1$. $N=1,000$.}
\end{tabular}
\end{table}

The OC ARLs of the LLD and MBE charts for various shifts of magnitude $\delta$ are
summarized in Table 3.1 when the EWMA smoothing parameter $\mu=0.1$ and the Phase II
sample size $N=1,000$. Due to space limitations, only some representative results
are tabulated. Table 3.1 shows that the MBE chart outperforms the LLD chart only for
small shifts in the main effects such as $\beta_{(3)}$ and $\beta_{(5)}$, which
represent the main effects of factors $C_3$ and $C_5$, respectively. This is partly
because the MBE chart is based on collecting one-way marginal sums for each factor,
whose changes are directly echoed by shifts in main effects or coefficients of the
first order. The LLD chart is developed especially for detecting one-coefficient
shifts, so this superiority of the MBE chart over the LLD chart is not very
significant. For shifts in the main effects, as the magnitudes increase, the
situation is reversed, and the LLD chart triggers an OC indication faster than the
MBE chart for larger shifts. If we select $q=1$, the LLD chart is indeed uniformly
superior over the MBE chart in detecting one-coefficient shifts in main effects.

When it comes to shifts in the two-factor interaction effects, including
$\beta_{(1,4)}$ representing the interaction effect of factors $C_1$ and $C_4$,
$\beta_{(2,3)}$, $\beta_{(2,5)}$, and $\beta_{(3,4)}$, the LLD chart performs
consistently better than the MBE chart. Additionally, in most situations this
advantage is quite substantial. The changes in high-order interaction effects
reflect smaller shifts of the one-way marginal sums, which are usually neglected by
the MBE chart. However, the LLD chart is still able to detect possible changes
attributable to higher-order interactions effectively.

With $\mu$ fixed, for the same shifts, both charts have greater power when the
sample size $N$ increases. This could be expected. For a fixed sample size $N$ and
the same coefficients, LLD charts with a smaller $\mu$ signal faster for smaller
shifts, and those with a larger $\mu$ signal faster for larger shifts. This mimics
the properties of the conventional EWMA chart (see Lucas and Saccucci (1990) and
Lowry et al. (1992)). Figures 2-(a) and -(b) confirm this further. They show the OC
ARL curves of the LLD chart for $\mu$ values of 0.05, 0.1, 0.2, and 0.5 with
$N=1,000$ when there are shifts in $\beta_{(5)}$ and $\beta_{(3,4)}$, respectively.

\begin{figure}[ht]
\begin{center}
\includegraphics[width=16.0cm,height=8.0cm]{fig3-2.eps}
\vspace{-0.5cm} \caption{\small OC ARL curves for the LLD chart for various values
of $\mu$ when there are shifts in: (a) $\beta_{(5)}$; (b)
$\beta_{(3,4)}$}\vspace{-0.3cm}
\end{center}
\end{figure}

The LLD chart is constructed by checking the GLRT statistic in Equation (\ref{F3.6})
indexed by $q=2$. However, as indicated earlier, shifts may indeed occur in effects
of the third or even higher order. Although such cases should be infrequent, they
are considered in Table 3.2 to test robustness. The OC ARLs of LLD and MBE charts
are compared for one-coefficient shifts in the three-factor interactions
$\beta_{(1,2,4)}$, $\beta_{(2,3,4)}$, and $\beta_{(3,4,5)}$ when $\mu=0.1$ and
$N=1,000$. Apparently, the advantage of the LLD chart over the MBE chart is
maintained, implying that the LLD chart can still deal with higher-order
interactions well.

As implied earlier, besides considering cross-classifications among factors, another
noteworthy characteristic of the LLD chart is power in discovering one-coefficient
shifts, which may cause one to suspect that it is not robust when dealing with
general changes. Therefore, some representative two-coefficient shift cases are
summarized in Table 3.3, which show that the LLD chart is still powerful in this
situation. Table 3.3 with $\mu=0.1$ and $N=1,000$ contains shifts in two
coefficients of the first or second orders, adding magnitudes $\delta_1$ and
$\delta_2$. According to Table 3.3, if the two shifts both appear in main effects of
the first order, such as $(\beta_{(1)}, \beta_{(5)})$, the MBE chart performs better
than the LLD chart. However, as the two shifts begin to include more effects of the
second order, the LLD chart outperforms the MBE chart.

%Table 2
\begin{table}[!ht]
\tabcolsep 4.5pt \vspace{-0.1cm} \centering \caption{OC ARL comparison between the
LLD and MBE charts for one-coefficient shifts of the third order} \vspace{0.3cm}
\renewcommand{\arraystretch}{1.25}
\begin{tabular}{r|cccc|cccc|cccc}\hline
$\delta$ & \multicolumn{2}{c}{LLD} & \multicolumn{2}{c|}{MBE} &
\multicolumn{2}{c}{LLD} & \multicolumn{2}{c|}{MBE} & \multicolumn{2}{c}{LLD} &
\multicolumn{2}{c}{MBE}
\\\hline
& \multicolumn{4}{c|}{$\beta_{(1,2,4)}$} & \multicolumn{4}{c|}{$\beta_{(2,3,4)}$} &
\multicolumn{4}{c}{$\beta_{(3,4,5)}$}
\\\hline
0.01 & 189 & (1.84) & 262 & (2.55) & 289 & (2.85) & 334 & (3.32) & 275 & (2.72) &
328 & (3.30)
\\
0.02 & 63.2 & (0.54) & 125 & (1.17) & 162 & (1.59) & 245 & (2.44) & 144 & (1.35) &
233 & (2.26)
\\
0.05 & 12.5 & (0.06) & 24.1 & (0.15) & 32.9 & (0.24) & 71.8 & (0.61) & 29.4 & (0.20)
& 62.0 & (0.52)
\\
0.20 & 2.72 & (0.01) & 3.99 & (0.01) & 4.89 & (0.02) & 8.04 & (0.03) & 4.77 & (0.02)
& 7.27 & (0.03)
\\
$-0.01$ & 160 & (1.53) & 232 & (2.26) & 251 & (2.49) & 298 & (2.99) & 236 & (2.30) &
290 & (2.88)
\\
$-0.02$ & 55.5 & (0.47) & 110 & (1.03) & 132 & (1.25) & 200 & (1.95) & 115 & (1.05)
& 187 & (1.81)
\\
$-0.05$ & 11.3 & (0.05) & 20.6 & (0.13) & 26.5 & (0.18) & 55.8 & (0.47) & 24.1 &
(0.16) & 47.8 & (0.38)
\\
$-0.20$ & 2.22 & (0.01) & 3.14 & (0.01) & 3.50 & (0.01) & 5.34 & (0.02) & 3.41 &
(0.01) & 4.91 & (0.02)
\\\hline
\multicolumn{13}{l}{NOTE: Standard errors are in parentheses, $\mu=0.1$ and
$N=1,000$.}
\end{tabular}
\end{table}

%Table 3
\begin{table}[!ht]
\tabcolsep 7.7pt  \centering \caption{OC ARL comparison between the LLD and MBE
charts for two-coefficient shifts of the first two orders} \vspace{0.3cm}
\renewcommand{\arraystretch}{1.25}
\begin{tabular}{rr|cccc|cccc}\hline
$\delta_1$ & $\delta_2$ & \multicolumn{2}{c}{LLD} & \multicolumn{2}{c|}{MBE} &
\multicolumn{2}{c}{LLD} & \multicolumn{2}{c}{MBE}
\\\hline
& & \multicolumn{2}{c}{$\beta_{(1)}+\delta_1$} &
\multicolumn{2}{c|}{$\beta_{(5)}+\delta_2$} &
\multicolumn{2}{c}{$\beta_{(2)}+\delta_1$} &
\multicolumn{2}{c}{$\beta_{(1,3)}+\delta_2$}
\\\hline
0.02 & 0.02 & 31.9 & (0.23) & 31.8 & (0.22) & 35.1 & (0.25) & 45.9 & (0.36)
\\
0.02 & $-0.02$ & 43.4 & (0.33) & 31.3 & (0.22) & 66.9 & (0.57) & 81.3 & (0.74)
\\
$-0.02$ & 0.02 & 43.1 & (0.33) & 31.1 & (0.22) & 70.1 & (0.60) & 80.2 & (0.72)
\\
$-0.02$ & $-0.02$ & 29.1 & (0.21) & 28.4 & (0.20) & 30.6 & (0.22) & 39.2 & (0.30)
\\
0.20 & 0.20 & 2.04 & (0.01) & 2.01 & (0.01) & 2.19 & (0.01) & 2.46 & (0.01)
\\
0.20 & $-0.20$ & 2.10 & (0.01) & 1.82 & (0.01) & 2.29 & (0.01) & 2.67 & (0.01)
\\
$-0.20$ & 0.20 & 1.95 & (0.01) & 1.79 & (0.01) & 2.74 & (0.01) & 2.90 & (0.01)
\\
$-0.20$ & $-0.20$ & 1.71 & (0.01) & 1.62 & (0.01) & 1.63 & (0.01) & 1.76 & (0.01)
\\\hline
& & \multicolumn{2}{c}{$\beta_{(1,4)}+\delta_1$} &
\multicolumn{2}{c|}{$\beta_{(4,5)}+\delta_2$} &
\multicolumn{2}{c}{$\beta_{(1,5)}+\delta_1$} &
\multicolumn{2}{c}{$\beta_{(2,3)}+\delta_2$}
\\\hline
0.02 & 0.02 & 26.9 & (0.18) & 45.6 & (0.36) & 18.7 & (0.11) & 25.4 & (0.17)
\\
0.02 & $-0.02$ & 47.6 & (0.37) & 101 & (0.93) & 76.6 & (0.67) & 96.8 & (0.87)
\\
$-0.02$ & 0.02 & 46.8 & (0.37) & 100 & (0.91) & 71.8 & (0.64) & 88.6 & (0.80)
\\
$-0.02$ & $-0.02$ & 24.4 & (0.16) & 40.9 & (0.32) & 17.0 & (0.10) & 23.0 & (0.15)
\\
0.20 & 0.20 & 1.93 & (0.01) & 2.32 & (0.01) & 1.69 & (0.00) & 1.88 & (0.01)
\\
0.20 & $-0.20$ & 2.13 & (0.01) & 3.23 & (0.01) & 2.88 & (0.01) & 3.40 & (0.01)
\\
$-0.20$ & 0.20 & 2.07 & (0.01) & 2.91 & (0.01) & 2.38 & (0.01) & 2.73 & (0.01)
\\
$-0.20$ & $-0.20$ & 1.55 & (0.01) & 1.91 & (0.01) & 1.22 & (0.00) & 1.43 & (0.00)
\\\hline
\multicolumn{10}{l}{NOTE: Standard errors are in parentheses, $\mu=0.1$ and
$N=1,000$.}
\end{tabular}
\end{table}


\subsection{Monitoring a Multivariate Multinomial Process}

With a multivariate multinomial process, the LLD chart can be compared with only the
somewhat naive multi-chart described earlier. Similar to the $\chi^2$-chart, the
multi-chart should be equipped with the EWMA scheme by replacing the observation
vectors in the charting statistic with their exponentially weighted sums. This is
actually the multivariate multinomial EWMA (MME) chart introduced in Section 2.4.2.

We consider the following case involving factors with mixed levels for illustration.
A service flow has four quality characteristics being monitored, with the first two
judged as satisfactory or unsatisfactory and the other two assessed as excellent,
acceptable, or unacceptable. This yields a four-way $2\times2\times3\times3$
contingency table. The IC log-linear model is described by the coefficient vector
\begin{eqnarray}
\begin{array}{rrrrrrrrrrrrl}
\widetilde{\bm{\beta}}^{(0)} & = & [ & \beta_0 & 0.73 & 0.72 & 0.70 &
0.12 & 0.71 & 0.10 & 0.17 & 0.12 & \\
& & & -0.15 & 0.19 & -0.14 & 0.23 & 0.07 & 0.16 & -0.14 & 0.23 & -0.30 & \\
& & & -0.17 & 0.14 & 0.10 & 0.06 & 0.09 & -0.12 & 0.19 & -0.15 & 0.11 & \\
& & & 0.22 & 0.24 & 0.24 & -0.08 & -0.16 & 0.07 & -0.11 & 0.05 & 0.03 & ]^T,
\end{array}
\nonumber
\end{eqnarray}
where $\beta_0$ is the intercept accommodating the constraint $\bm1^T{\bf p}=1$.
Since the service flow process has four factors, the MME chart here is composed of
four individual charts. Their IC ARLs are selected by simulation in order to be
identical and jointly achieve an overall ARL$_0$ of 370.

Similar to the comparison between the LLD and MBE charts, the OC ARLs when only one
coefficient changes by $\delta$ are reported in Table 3.4 with $\mu=0.1$ and
$N=1,000$. When the one-coefficient shift comes from one of the main effects, for
instance, $\beta_{(1)}$, $\beta_{(3_2)}$, or $\beta_{(4_1)}$, the MME chart shows
better performance than the LLD chart in terms of giving rise to an OC alert faster.
This is not surprising, as each of the four individual charts summarizes efficiently
the one-way marginal sums for one of the four factors, and these are decided by the
coefficients of the first order. When two-factor interaction effects such as
$\beta_{(1,2)}$, $\beta_{(1,3_2)}$, $\beta_{(2,3_1)}$, $\beta_{(2,4_2)}$,
$\beta_{(3_1,4_1)}$, and $\beta_{(3_2,4_2)}$ are the focus, the superiority of the
LLD chart over the MME chart becomes clear. The advantage of the LLD chart lies
partly in signalling shifts in higher-order interaction effects and partly in
one-coefficient shifts.

%Table 4
\begin{table}[ht]
\tabcolsep 4.5pt \vspace{-0.3cm} \centering \caption{OC ARL comparison between the
LLD and MME charts for one-coefficient shifts of the first two orders}
\vspace{0.3cm}
\renewcommand{\arraystretch}{1.20}
\begin{tabular}{r|cccc|cccc|cccc}\hline
$\delta$ & \multicolumn{2}{c}{LLD} & \multicolumn{2}{c|}{MME} &
\multicolumn{2}{c}{LLD} & \multicolumn{2}{c|}{MME} & \multicolumn{2}{c}{LLD} &
\multicolumn{2}{c}{MME}
\\\hline
& \multicolumn{4}{c|}{$\beta_{(1)}$} & \multicolumn{4}{c|}{$\beta_{(3_2)}$} &
\multicolumn{4}{c}{$\beta_{(4_1)}$}
\\\hline
0.02 & 128 & (1.20) & 88.2 & (0.80) & 145 & (1.36) & 148 & (1.39) & 110 & (0.99) &
104 & (0.95)
\\
0.05 & 19.8 & (0.11) & 15.8 & (0.09) & 24.5 & (0.16) & 24.4 & (0.16) & 17.6 & (0.10)
& 17.2 & (0.09)
\\
0.20 & 3.55 & (0.01) & 3.17 & (0.01) & 3.54 & (0.01) & 3.50 & (0.01) & 3.27 & (0.01)
& 3.20 & (0.01)
\\
$-0.02$ & 103 & (0.94) & 72.7 & (0.66) & 153 & (1.45) & 150 & (1.43) & 92.4 & (0.85)
& 86.5 & (0.77)
\\
$-0.05$ & 16.5 & (0.09) & 13.6 & (0.07) & 25.5 & (0.17) & 24.6 & (0.16) & 15.8 &
(0.09) & 15.3 & (0.08)
\\
$-0.20$ & 2.61 & (0.01) & 2.35 & (0.01) & 3.69 & (0.01) & 3.63 & (0.01) & 2.64 &
(0.01) & 2.58 & (0.01)
\\\hline
& \multicolumn{4}{c|}{$\beta_{(1,2)}$} & \multicolumn{4}{c|}{$\beta_{(1,3_2)}$} &
\multicolumn{4}{c}{$\beta_{(2,3_1)}$}
\\\hline
0.02 & 86.3 & (0.78) & 105 & (0.96) & 149 & (1.41) & 224 & (2.18) & 102 & (0.92) &
140 & (1.25)
\\
0.05 & 14.3 & (0.07) & 21.1 & (0.13) & 24.5 & (0.16) & 51.7 & (0.42) & 16.6 & (0.09)
& 26.4 & (0.18)
\\
0.20 & 2.95 & (0.01) & 3.84 & (0.01) & 3.51 & (0.01) & 5.15 & (0.02) & 3.10 & (0.01)
& 4.11 & (0.01)
\\
$-0.02$ & 71.7 & (0.63) & 85.7 & (0.77) & 148 & (1.39) & 217 & (2.13) & 86.7 &
(0.79) & 115 & (1.10)
\\
$-0.05$ & 12.6 & (0.06) & 17.1 & (0.10) & 24.9 & (0.16) & 53.1 & (0.43) & 14.9 &
(0.08) & 22.8 & (0.15)
\\
$-0.20$ & 2.26 & (0.01) & 2.80 & (0.01) & 3.62 & (0.01) & 5.75 & (0.02) & 2.58 &
(0.01) & 3.42 & (0.01)
\\\hline
& \multicolumn{4}{c|}{$\beta_{(2,4_2)}$} & \multicolumn{4}{c|}{$\beta_{(3_1,4_1)}$}
& \multicolumn{4}{c}{$\beta_{(3_2,4_2)}$}
\\\hline
0.02 & 165 & (1.58) & 245 & (2.43) & 94.6 & (0.85) & 136 & (1.25) & 259 & (2.54) &
338 & (3.33)
\\
0.05 & 27.2 & (0.18) & 63.3 & (0.53) & 15.7 & (0.08) & 25.0 & (0.17) & 64.7 & (0.54)
& 235 & (2.25)
\\
0.20 & 3.71 & (0.01) & 5.89 & (0.02) & 2.98 & (0.01) & 3.94 & (0.01) & 5.79 & (0.02)
& 23.7 & (0.15)
\\
$-0.02$ & 165 & (1.61) & 235 & (2.32) & 82.2 & (0.75) & 114 & (1.05) & 261 & (2.55)
& 332 & (3.31)
\\
$-0.05$ & 27.8 & (0.19) & 62.9 & (0.54) & 14.5 & (0.08) & 22.1 & (0.14) & 68.3 &
(0.57) & 244 & (2.45)
\\
$-0.20$ & 3.81 & (0.01) & 6.28 & (0.02) & 2.59 & (0.01) & 3.44 & (0.01) & 5.96 &
(0.02) & 31.1 & (0.22)
\\\hline
\multicolumn{13}{l}{NOTE: Standard errors are in parentheses, $\mu=0.1$ and
$N=1,000$.}
\end{tabular}\vspace{-0.3cm}
\end{table}

In practice, we should note that in a log-linear model, all the elements of a
coefficient subvector as a whole represent the corresponding main or interaction
effect, and that they should change simultaneously if there is a shift in this
effect. In other words, a shift should occur at the effect or coefficient subvector
level, instead of the coefficient level as shown in Table 3.4. There is a further
discussion about monitoring such changes at the effect level in Section 3.8.3, the
appendix in this chapter.

Suppose that there is a shift at the effect level, by which all the elements of its
corresponding coefficient subvector, say $\bm\beta_i$, shift simultaneously. This
change can still be detected powerfully by the LLD chart, which checks individually
whether each element in $\bm\beta_i$ shifts or not. Table 3.5 lists the comparison
results. For a shift in the main effect of factor $C_3$, a change occurs in
$\bm\beta_3=[\beta_{(3_1)},\beta_{(3_2)}]^T$, where $\beta_{(3_1)}$ and
$\beta_{(3_2)}$ add by $\delta_1$ and $\delta_2$, respectively, the LLD chart
detects it slower than the MME chart. This could be understood. However, if a shift
arises in an interaction effect such as
$\bm\beta_7=[\beta_{(1,4_1)},\beta_{(1,4_2)}]^T$ that represents the interaction
effect of factors $C_1$ and $C_4$, $\bm\beta_8=[\beta_{(2,3_1)},\beta_{(2,3_2)}]^T$,
and $\bm\beta_9=[\beta_{(2,4_1)},\beta_{(2,4_2)}]^T$, the situation is completely
reversed, in that the LLD chart has uniformly smaller OC ARLs than the MME chart.
This is still because the LLD chart takes advantage of cross-classifications among
multiple factors.

%Table 5
\begin{table}[!ht]
\tabcolsep 7.7pt  \centering \caption{OC ARL comparison between the LLD and MME
charts for one-effect shifts of the first two orders} \vspace{0.3cm}
\renewcommand{\arraystretch}{1.25}
\begin{tabular}{rr|cccc|cccc}\hline
$\delta_1$ & $\delta_2$ & \multicolumn{2}{c}{LLD} & \multicolumn{2}{c|}{MME} &
\multicolumn{2}{c}{LLD} & \multicolumn{2}{c}{MME}
\\\hline
& &\multicolumn{2}{c}{$\beta_{(3_1)}+\delta_1$} &
\multicolumn{2}{c|}{$\beta_{(3_2)}+\delta_2$} &
\multicolumn{2}{c}{$\beta_{(1,4_1)}+\delta_1$} &
\multicolumn{2}{c}{$\beta_{(1,4_2)}+\delta_2$}
\\\hline
0.02 & 0.02 & 129 & (1.20) & 87.9 & (0.78) & 62.3 & (0.51) & 95.8 & (0.87) \\
0.02 & $-0.02$ & 61.8 & (0.52) & 57.9 & (0.47) & 89.0 & (0.80) & 107 & (0.97) \\
$-0.02$ & 0.02 & 55.1 & (0.45) & 51.4 & (0.41) & 77.1 & (0.68) & 89.7 & (0.80) \\
$-0.02$ & $-0.02$ & 105 & (0.96) & 68.2 & (0.59) & 51.4 & (0.42) & 73.7 & (0.64)
\\\hline
$\delta_1$ & $\delta_2$ & \multicolumn{2}{c}{$\beta_{(2,3_1)}+\delta_1$} &
\multicolumn{2}{c|}{$\beta_{(2,3_2)}+\delta_2$} &
\multicolumn{2}{c}{$\beta_{(2,4_1)}+\delta_1$} &
\multicolumn{2}{c}{$\beta_{(2,4_2)}+\delta_2$}
\\\hline
0.02 & 0.02 & 103 & (0.94) & 148 & (1.39) & 63.1 & (0.52) & 89.9 & (0.80) \\
0.02 & $-0.02$ & 62.6 & (0.52) & 83.5 & (0.74) & 85.8 & (0.77) & 100 & (0.90) \\
$-0.02$ & 0.02 & 55.1 & (0.46) & 73.1 & (0.64) & 73.5 & (0.64) & 87.1 & (0.77) \\
$-0.02$ & $-0.02$ & 83.1 & (0.74) & 112 & (1.04) & 52.7 & (0.43) & 70.8 & (0.62)
\\\hline
\multicolumn{10}{l}{NOTE: Standard errors are in parentheses, $\mu=0.1$ and
$N=1,000$.}
\end{tabular}
\end{table}



\subsection{Diagnostic Performance Analysis}

The performance of our suggested diagnostic method is reported in this subsection.
In the simulations, the change-point $\tau$ is set at 50, and 10,000 independent
series are generated. Any series for which an OC indication is triggered before time
point $\tau+1$ will be discarded. An IC log-linear model is used with the same
coefficient vector as in the subsection on monitoring multivariate binomial
processes, for the situation with five factors, each with two levels.

According to our design recommendation for the diagnostic method, the candidate
subset for potential shift directions is selected as coefficients in the effects of
the first three orders with $q^{\prime}=3$, with the assumption that no effect of an
order larger than three would shift. Therefore, in the simulations the real
one-coefficient shifts considered are confined to the first three orders, in
particular the main effects $\beta_{(2)}$ and $\beta_{(4)}$, the two-factor
interaction effects $\beta_{(1,3)}$, $\beta_{(1,5)}$, $\beta_{(2,3)}$, and
$\beta_{(4,5)}$, as well as the three-factor interaction effects $\beta_{(1,4,5)}$
and $\beta_{(2,3,5)}$. The diagnostic performance is investigated for various shift
magnitudes $\delta$ and different combinations of the EWMA smoothing parameter $\mu$
and the sample size $N$. Table 3.6 lists the results for the observed matching
probability $P(\hat{\zeta}=\zeta)$, where $\hat{\zeta}\in\mathbb{I}_{q^{\prime}}$ is
the estimated shift direction index and $\zeta\in\mathbb{I}_{q^{\prime}}$ is the
real one-coefficient shift direction index. For instance, in the case of $\mu=0.1$
and $N=1,000$, when only the coefficient $\beta_{(2)}$ adds by $\delta=0.02$, the
diagnostic scheme provides a correct prescription in $48\%$ of the simulations, and
this percentage rises to $84\%$ if the magnitude $\delta$ is 0.05.

At least three conclusions can be drawn with respect to the effects of the shift
magnitudes $\delta$, the EWMA smoothing parameter $\mu$, and the sample size $N$.
First, for one coefficient of the first three orders with fixed $\mu$ and $N$, as
the magnitude $\delta$ increases, the diagnostic consistency improves progressively
with the observed matching probability approaching 1, as might be expected. Second,
while keeping $\mu$ and $\delta$ unchanged, a larger sample size $N$ will improve
diagnostic performance, which is also to be expected. Third, a larger $\mu$ will be
beneficial for identifying larger shifts while a smaller one can assist in
recognizing smaller deviations. Figure 3.3 further demonstrates this pattern with
various values of $\mu$ and $N=1,000$.

%Table 6
\begin{table}[ht]
\tabcolsep 6.0pt \vspace{-0.3cm} \centering \caption{Observed matching probability
in cases of one-coefficient shifts of the first three orders} \vspace{0.3cm}
\renewcommand{\arraystretch}{1.25}
\begin{tabular}{r|r|c|c|c|c|c|c|c|c}\hline
Case & $\delta$ & $\;\;\beta_{(2)}\;\;$ & $\;\;\beta_{(4)}\;\;$ &
$\;\beta_{(1,3)}\;$ & $\;\beta_{(1,5)}\;$ & $\;\beta_{(2,3)\;}$ &
$\;\beta_{(4,5)}\;$ & $\beta_{(1,4,5)}$ & $\beta_{(2,3,5)}$
\\\hline
& 0.02 & 0.48 & 0.46 & 0.52 & 0.43 & 0.46 & 0.50 & 0.23 & 0.18 \\
$\mu=0.1$ & 0.05 & 0.84 & 0.71 & 0.73 & 0.62 & 0.73 & 0.73 & 0.71 & 0.64 \\
& 0.20 & 0.95 & 0.84 & 0.84 & 0.77 & 0.86 & 0.86 & 0.87 & 0.85 \\
& $-$0.02 & 0.51 & 0.46 & 0.53 & 0.43 & 0.46 & 0.50 & 0.24 & 0.20 \\
$N=1,000$ & $-$0.05 & 0.82 & 0.69 & 0.71 & 0.61 & 0.71 & 0.71 & 0.71 & 0.63 \\
& $-$0.20 & 0.94 & 0.85 & 0.84 & 0.79 & 0.86 & 0.86 & 0.87 & 0.85
\\\hline
\multicolumn{10}{l}{NOTE: $\mu=0.1$ and $N=1,000$.}
\end{tabular}\vspace{-0.3cm}
\end{table}

\begin{figure}[ht]
\begin{center}
\includegraphics[width=15.0cm,height=8.0cm]{fig3-3.eps}
\vspace{-0.3cm} \caption{\small Matching probability curves for the diagnostic
scheme with various values of $\mu$ for identifying the real shifts in: (a)
$\beta_{(2,3)}$; (b) $\beta_{(4,5)}$.}\vspace{-0.3cm}
\end{center}
\end{figure}



\section{Summary}\label{sec3.7}

We have developed a new log-linear directional control chart for monitoring process
shifts with high efficiency, which is based on combining the log-likelihood of
log-linear models and the EWMA scheme. In addition, a post-signal diagnostic scheme
for recognizing the shift direction is formulated. The LLD chart has been shown to
work well with multivariate categorical processes, including multivariate binomial
and multivariate multinomial distributions, and to successfully incorporate factor
interactions among multiple categorical factors. Furthermore, the diagnostic method
shows good performance in estimating fault directions. Practical guidelines for
parameter settings have been provided, along with an illustrative example of
implementing the monitoring and diagnostic methodology.



\section{Appendix}\label{sec3.8}

\subsection{Derivation of the GLRT Statistic in Equation
(\ref{F3.5})}

Denote the IC coefficient vector by
$\widetilde{\bm{\beta}}^{(0)}=\big[\beta^{(0)}_0,\big(\bm{\beta}^{(0)}\big)^T\big]^T$,
and the IC probability vector ${\bf p}^{(0)}$ satisfies
\[
{\bf p}^{(0)}=\exp\Big(\big[\bm{1},{\bf
X}\big]\big[\beta^{(0)}_0,\big(\bm{\beta}^{(0)}\big)^T\big]^T\Big).
\]
Consider the log-likelihood function for the log-linear model (\ref{F3.1}), which
can be written from the PMF of the multinomial distribution as
\[
l\big(\widetilde{\bm{\beta}}\big)={\bf n}^T\ln{\bf p}+\ln N!-(\ln {\bf
n}!)^T\bm{1}={\bf n}^T\widetilde{{\bf X}}\widetilde{\bm{\beta}}+\ln N!-(\ln {\bf
n}!)^T\bm{1}.
\]
Both the logarithm and the factorial operators on the column vectors operate on each
of their entries.

Without loss of generality, assume that in the OC state only the $i$th
($i\in\mathbb{I}_q$) coefficient $\beta_i^{(0)}$ in $\bm{\beta}^{(0)}$ is
incremented by an amount $\delta_i$, and that all the other $\beta_j^{(0)}$
($j\in\{1,\ldots,h-1\}$ and $j\ne i$) remain unchanged. The log-likelihood of the
observation vector ${\bf n}$ in Phase II can be expressed as
\[
l(\delta_i)={\bf n}^T\left[\bm{1},{\bf
X}\right]\left[\beta^{(0)}_0+\alpha_i,\big(\bm{\beta}^{(0)}+{\bf
d}_i\delta_i\big)^T\right]^T+\ln N!-(\ln{\bf n}!)^T\bm{1},
\]
where $\alpha_i$ is the variation of $\beta^{(0)}_0$ induced by the constraint
$\bm{1}^T{\bf p}=1$. The MLE $\hat{\delta}_i$ of $\delta_i$ is actually the solution
to $l^{\prime}(\delta_i)=0$. Solving this equation needs a numerical iteration such
as that of Newton-Raphson. However, by some approximations, the MLE $\hat{\delta}_i$
of $\delta_i$ can be expressed as a simple form, which is shown below.

If only $\beta_i^{(0)}$ in $\bm{\beta}^{(0)}$ adds by a magnitude $\delta_i$, the
variation $\alpha_i$ of $\beta^{(0)}_0$ must satisfy the constraint
\begin{equation}
\bm{1}^T\exp\big(\bm{1}\beta^{(0)}_0 +{\bf X}\bm\beta^{(0)}+\bm{1}\alpha_i+{\bf
x}_i\delta_i\big)=1,\label{F3.8}
\end{equation}
and $\alpha_i$ is calculated as
\begin{equation}
\alpha_i=-\ln\big(\bm{1}^T\exp\big(\bm{1}\beta^{(0)}_0 +{\bf X}\bm\beta^{(0)}+{\bf
x}_i\delta_i\big)\big).\label{F3.9}
\end{equation}
Therefore, $l(\delta_i)$ can be further rewritten as
\begin{align*}
l(\delta_i)=&\;{\bf n}^T\big[\bm{1}\beta^{(0)}_0 +{\bf
X}\bm\beta^{(0)}-\bm{1}\ln\big(\bm{1}^T\exp\big(\bm{1}\beta^{(0)}_0 +{\bf
X}\bm\beta^{(0)}+{\bf x}_i\delta_i\big)\big)+{\bf
x}_i\delta_i\big]+\\
&\;\ln N!-(\ln{\bf n}!)^T\bm{1}.
\end{align*}
The first-order derivative of $l(\delta_i)$ with respect to $\delta_i$ is
\begin{align*}
s(\delta_i)=&\;\frac{\ud
l(\delta_i)}{\ud\delta_i}\\
=&\;{\bf x}_i^T{\bf n}-N{\bf x}_i^T
\exp\big[\bm{1}\beta^{(0)}_0+{\bf X}\bm\beta^{(0)}+{\bf x}_i\delta_i-\\
&\;\bm{1}\ln\big(\bm{1}^T\exp\big(\bm{1}\beta^{(0)}_0 +{\bf X}\bm\beta^{(0)}+{\bf
x}_i\delta_i\big)\big)\big].
\end{align*}
Let
\[
{\bf k}(\delta_i)=\exp\big[\bm{1}\beta^{(0)}_0+{\bf X}\bm\beta^{(0)}+{\bf
x}_i\delta_i-\bm{1}\ln\big(\bm{1}^T\exp\big(\bm{1}\beta^{(0)}_0 +{\bf
X}\bm\beta^{(0)}+{\bf x}_i\delta_i\big)\big)\big].
\]
We can further formulate the second-order derivative $l(\delta_i)$ with respect to
$\delta_i$ as
\[
s^{\prime}(\delta_i)=\;\frac{\ud^2 l(\delta_i)}{\ud\delta^2_i}=\;-N{\bf
x}_i^T\di\big({\bf k}(\delta_i)\big){\bf x}_i + N{\bf x}_i^T{\bf k}(\delta_i){\bf
k}^T(\delta_i){\bf x}_i.
\]
Clearly,
\begin{align*}
&s(0)={\bf x}_i^T{\bf n}-N{\bf x}_i^T \exp\big(\bm{1}\beta^{(0)}_0+{\bf
X}\bm\beta^{(0)}\big)={\bf x}_i^T({\bf
n}-N{\bf p}^{(0)}),\\
&s^{\prime}(0)=-N{\bf x}_i^T\di\big({\bf p}^{(0)}\big){\bf x}_i+ N{\bf x}_i^T{\bf
p}^{(0)}({\bf p}^{(0)})^T{\bf x}_i=-N{\bf x}_i^T\bm{\Sigma}^{(0)}{\bf x}_i.
\end{align*}
By performing the first-order Taylor expansion of $s(\delta_i)$ at $\delta_i=0$, we
have
\[
s(\delta_i)\approx s(0)+s^{\prime}(0)\delta_i.
\]
Note that the MLE $\hat{\delta}_i$ of $\delta_i$ should be the solution to
$s(\delta_i)=0$, and therefore, the MLE $\hat{\delta}_i$ can be approximated as
\begin{equation}
\hat{\delta}_i\approx-\frac{s(0)}{s^{\prime}(0)}=\frac{1}{N}\big({\bf n}-N{\bf
p}^{(0)}\big)^T{\bf x}_i\big({\bf x}_i^T\bm{\Sigma}^{(0)}{\bf
x}_i\big)^{-1}.\label{F3.10}
\end{equation}

In the IC state, the log-likelihood function of the observation vector ${\bf n}$ in
Phase II is
\[
l_0={\bf n}^T\big[\bm{1},{\bf
X}\big]\big[\beta^{(0)}_0,\big(\bm{\beta}^{(0)}\big)^T\big]^T+\ln N!-(\ln{\bf
n}!)^T\bm{1}.
\]
The $-2$LRT statistic will then be $Q_i(\hat{\delta_i})=2(l(\hat{\delta_i})-l_0)$.
Here $Q_i(\hat{\delta_i})$ also has a simple form by some approximations, which is
shown below.

From (\ref{F3.9}), $Q_i(\delta_i)$ can be further rewritten as
\begin{align*}
Q_i(\delta_i)=&\;2{\bf n}^T(\bm{1}\alpha_i+{\bf
x}_i\delta_i)\\
=&\;-2N\ln\big(\bm{1}^T\exp\big(\bm{1}\beta^{(0)}_0 +{\bf X}\bm\beta^{(0)}+{\bf
x}_i\delta_i\big)\big) + 2{\bf n}^T{\bf x}_i\delta_i.
\end{align*}
By the second-order Taylor expansion of $Q_i(\delta_i)$ at $\delta_i=0$, we have
\begin{equation}
Q_i(\delta_i)\approx
Q_i(0)+Q_i^{\prime}(0)\delta_i+\frac{1}{2}Q_i^{\prime\prime}(0)\delta_i^2.\label{F3.11}
\end{equation}
In a similar way to the formulation of ${s(0)}$ and ${s^{\prime}(0)}$, the following
results hold:
\begin{align*}
&Q_i(0)=0,\\
&Q_i^{\prime}(0)=2{\bf x}_i^T\big({\bf n}-N{\bf p}^{(0)}\big),\\
&Q_i^{\prime\prime}(0)=-2N{\bf x}_i^T\bm{\Sigma}^{(0)}{\bf x}_i.
\end{align*}
By substituting these terms as well as $\hat{\delta}_i$ in (\ref{F3.10}) into
(\ref{F3.11}) and some algebra, we obtain
\[
Q_i(\hat{\delta}_i)\approx\frac{1}{N}\big({\bf n}-N{\bf p}^{(0)}\big)^T{\bf
x}_i\big({\bf x}_i^T\bm{\Sigma}^{(0)}{\bf x}_i\big)^{-1}{\bf x}_i^T\big({\bf
n}-N{\bf p}^{(0)}\big).
\]

Up to now, it has been assumed that at most one coefficient $\beta_i^{(0)}$ in
$\bm{\beta}^{(0)}$ will deviate, but its location is unknown. Therefore, for each
$\beta_i^{(0)}$ ($i\in\mathbb{I}_q$), calculate its corresponding $-2$LRT statistic
$Q_i(\hat{\delta}_i)$ and then combine them into the following GLRT statistic
\begin{align*}
Q=\max_{i\in\mathbb{I}_q}\Big(\frac{1}{N}\big({\bf n}-N{\bf p}^{(0)}\big)^T{\bf
x}_i\big({\bf x}_i^T\bm{\Sigma}^{(0)}{\bf x}_i\big)^{-1}{\bf x}_i^T\big({\bf
n}-N{\bf p}^{(0)}\big)\Big).
\end{align*}

\subsection{Proof of the Consistency of $\arg\max_{j\in\mathbb{I}_{q^{\prime}}}S_j$ in Equation (\ref{F3.7})}

Without loss of generality, we follow the assumption used in Section 3.8.1 that
$\beta_i^{(0)}$ in $\bm{\beta}^{(0)}$ adds by a magnitude $\delta_i$. In
$\widetilde{\bm{\beta}}^{(1,i)}$, $\bm{\beta}^{(1,i)}=\bm{\beta}^{(0)}+{\bf
d}_i\delta_i$ and $\beta_0^{(1,i)}=\beta_0^{(0)}+\alpha_i$, and the OC probability
vector satisfies
\[
{\bf p}^{(1,i)}=\exp\Big(\big[\bm{1},{\bf X}\big]\big[\beta^{(1,i)}_0,
\big(\bm{\beta}^{(1,i)}\big)^T\big]^T\Big).
\]

To prove the consistency of this estimator is equivalent to showing that
$$\Pr\left\{\bigcup\limits_{j\neq i}\left[S_j> S_i\right]\right\}\rightarrow 0
\quad {\mathrm{as}} \quad N\rightarrow\infty.$$ By Bonferroni inequality, it further
suffices to show that
$$\sum\limits_{j\neq i}\Pr\left\{S_j> S_i \right\} \rightarrow 0  \quad {\mathrm{as}}
\quad N\rightarrow\infty.$$ Denote $z_j=S_j-S_i$. For any $\varepsilon>0$,
\begin{align}
\sum\limits_{j\neq i}&\Pr\left\{z_j>\varepsilon\right\}\nonumber\\
&\leq \sum\limits_{j\neq
i}\Pr\left\{z_j-E(z_j)>\frac{\varepsilon}{2}\right\}+\sum\limits_{j\neq
i}\Pr\left\{E(z_j)>\frac{\varepsilon}{2}\right\} \label{F3.12}
\end{align}

Firstly, we handle the second term in (\ref{F3.12}). Note that the constraint
(\ref{F3.8}) can also be rewritten as
\[
\bm{1}^T\exp\big(\bm{1}\beta^{(1,i)}_0 +{\bf X}\bm\beta^{(1,i)}-\bm{1}\alpha_i-{\bf
x}_i\delta_i\big)=1,
\]
and correspondingly we have
\[
\alpha_i=\ln\big(\bm{1}^T\exp\big(\bm{1}\beta^{(1,i)}_0 +{\bf
X}\bm\beta^{(1,i)}-{\bf x}_i\delta_i\big)\big).
\]
Moreover, the IC probability vector
\begin{align*}
{\bf p}^{(0)}=&\;\exp\big(\bm{1}\beta^{(0)}_0
+{\bf X}\bm\beta^{(0)}\big)\\
=&\;\exp\big(\bm{1}\beta^{(1,i)}_0+{\bf X}\bm\beta^{(1,i)}-\bm{1}\alpha_i-{\bf
x}_i\delta_i\big)\\
=&\;\exp\big(\bm{1}\beta^{(1,i)}_0+{\bf X}\bm\beta^{(1,i)}-
\bm{1}\ln\big(\bm{1}^T\exp\big(\bm{1}\beta^{(1,i)}_0 +{\bf X}\bm\beta^{(1,i)}-{\bf
x}_i\delta_i\big)\big)-{\bf x}_i\delta_i\big).
\end{align*}
Since diagnosis is considered in the OC state, we perform the first-order Taylor
expansion of ${\bf p}^{(0)}$ at ${\bf p}^{(1,i)}$ (i.e., $\delta_i=0$ in the above
equation), and obtain
\begin{align*}
{\bf p}^{(0)}\approx&\;\exp\big(\bm{1}\beta^{(1,i)}_0+{\bf
X}\bm\beta^{(1,i)}\big)-\big(\di\big({\bf p}^{(1,i)}\big){\bf
x}_i-{\bf p}^{(1,i)}({\bf p}^{(1,i)})^T{\bf x}_i\big)\delta_i\\
=&\;{\bf p}^{(1,i)}-\bm{\Sigma}^{(1,i)}{\bf x}_i\delta_i.
\end{align*}
Since ${\bf n}$ is an observation vector of size $N$ collected from
$F\big(\widetilde{{\bf X}};\widetilde{\bm{\beta}}^{(1,i)}\big)$, its expectation and
covariance matrix satisfy
\[
\ext\big({\bf n}-N{\bf p}^{(0)}\big)=N{\bf p}^{(1,i)}-N{\bf
p}^{(0)}=N\bm{\Sigma}^{(1,i)}{\bf x}_i\delta_i,
\]
\[
\cov\big({\bf n}-N{\bf p}^{(0)}\big)=\cov\big({\bf n}\big)=N^2\bm{\Sigma}^{(1,i)}.
\]
Let
\begin{align*}
&{\bf A}_i={\bf x}_i\big({\bf x}_i^T\bm{\Sigma}^{(1,i)}{\bf
x}_i\big)^{-1}{\bf x}_i^T,\;\;i\in\mathbb{I}_{q^{\prime}}\\
&{\bf A}_j={\bf x}_j\big({\bf x}_j^T\bm{\Sigma}^{(1,i)}{\bf x}_j\big)^{-1}{\bf
x}_j^T,\;\;j\in\mathbb{I}_{q^{\prime}}\;\;\textrm{and}\;\;j\ne i.
\end{align*}
Taking the expectation of $S_i$ and $S_j$, we have
\begin{align*}
\ext S_i=&\;\frac{1}{N}\tr\big({\bf A}_i\cov\big({\bf n}-N{\bf
p}^{(0)}\big)\big)+\frac{1}{N}\ext^T\big({\bf n}-N{\bf
p}^{(0)}\big){\bf A}_i\ext\big({\bf n}-N{\bf p}^{(0)}\big)\\
=&\;N\tr\big(\big({\bf x}_i^T\bm{\Sigma}^{(1,i)}{\bf x}_i\big)^{-1}\big({\bf
x}_i^T\bm{\Sigma}^{(1,i)}{\bf x}_i\big)\big)+N{\bf x}_i^T\bm{\Sigma}^{(1,i)}{\bf
A}_i\bm{\Sigma}^{(1,i)}{\bf x}_i\delta_i^2\\
=&\;N+N{\bf x}_i^T\bm{\Sigma}^{(1,i)}{\bf x}_i\delta_i^2,\\
\ext S_j=&\;\frac{1}{N}\tr\big({\bf A}_j\cov\big({\bf n}-N{\bf
p}^{(0)}\big)\big)+\frac{1}{N}\ext^T\big({\bf n}-N{\bf
p}^{(0)}\big){\bf A}_j\ext\big({\bf n}-N{\bf p}^{(0)}\big)\\
=&\;N\tr\big(\big({\bf x}_j^T\bm{\Sigma}^{(1,i)}{\bf x}_j\big)^{-1}\big({\bf
x}_j^T\bm{\Sigma}^{(1,i)}{\bf x}_j\big)\big)+N{\bf x}_i^T\bm{\Sigma}^{(1,i)}{\bf
A}_j\bm{\Sigma}^{(1,i)}{\bf x}_i\delta_i^2\\
=&\;N+N\big({\bf x}_i^T\bm{\Sigma}^{(1,i)}{\bf x}_j\big)\big({\bf
x}_j^T\bm{\Sigma}^{(1,i)}{\bf x}_j\big)^{-1}\big({\bf x}_j^T\bm{\Sigma}^{(1,i)}{\bf
x}_i\big)\delta_i^2.
\end{align*}
From the Cauchy-Schwarz inequality, it follows clearly that
\[
\ext S_i>\ext S_j,
\]
say, $E(z_j)<0$ and thus it immediately follows that the second term of
(\ref{F3.12}) equals to zero.

On the other hand, it is not hard to verify that $z_j\rightarrow E(z_j)$ as
$N\rightarrow \infty$ and ${\mathrm {Var}}(z_j)= O(N^{-1})$. Thus, by the Chebychev
inequality, the first term of (\ref{F3.12}) tends to zero as $N\rightarrow \infty$.
Taking all these results together, (\ref{F3.12}) tends to zero as $N\rightarrow
\infty$, which establishes the consistency of diagnostic statistic
$\arg\max_{j\in\mathbb{I}_{q^{\prime}}}S_j$.



\subsection{Discussion on Monitoring Shifts at the Effect Level}

The marginal distribution of one factor is mainly determined by its main effect,
whereas the dependence among multiple factors is represented by their interaction
effect. So shifts in the marginal distribution of one factor appear in the form of
deviations of the coefficient subvector corresponding to its main effect, and shifts
in the dependence among multiple factors arise in terms of deviations of the
coefficient subvector reflecting their interaction effect. This is the practical
explanation of shifts in multivariate categorical processes.

In a $2\times 3$ contingency table, $\bm\beta_1=\beta_{(1)}=\beta_1$ measures the
main effect of factor $C_1$ with two levels, and
$\bm\beta_2=[\beta_{(2_1)},\beta_{(2_2)}]^T=[\beta_2,\beta_3]^T$ measures the main
effect of factor $C_2$ with three levels. Here the coefficient subvector
$\bm\beta_1$ is actually a scalar, whereas $\bm\beta_2$ is indeed a vector. Without
any practical considerations, a shift in $\bm\beta_2$ may have multiple patterns,
including only $\beta_2$ shifting, only $\beta_3$ shifting, and both $\beta_2$ and
$\beta_3$ shifting simultaneously. However, for a coefficient subvector such as
$\bm\beta_2$ above, a shift in only one element of it, say $\beta_2$ in
$\bm\beta_2$, may not be possible in practice. This follows because all the elements
of this coefficient subvector as a group represent the corresponding main or
interaction effect, and they should change simultaneously if there is a shift in
this effect. In other words, a shift should occur at the effect or subvector level
instead of the coefficient level.

It seems that the LLD chart should be modified to accommodate this fact. For
simplicity, we ignore the fact that shifts tend to occur in lower-order effects,
leading to testing the hypothesis with its alternative in (\ref{F3.3})
\begin{align*}
&H_0:\;\bm{\beta}\neq\bm\beta^{(0)}\\
&H_1:\;\bm{\beta}=\bm{\beta}^{(0)}+{\bf d}_1\delta_1
\;\mathrm{or}\;\;\bm{\beta}=\bm{\beta}^{(0)}+{\bf d}_2\delta_2\ldots\;
\mathrm{or}\;\;\bm{\beta}=\bm{\beta}^{(0)}+{\bf d}_{h-1}\delta_{h-1}.
\end{align*}
Consequently, the GLRT statistic is
\begin{align}
R_k=\max_{i\in\{1,\ldots,h-1\}}R_{i,k},\label{F3.13}
\end{align}
where
\[
R_{i,k}=\frac{1}{N}\big({\bf z}_k-N{\bf p}^{(0)}\big)^T{\bf x}_i\big({\bf
x}_i^T\bm{\Sigma}^{(0)}{\bf x}_i\big)^{-1}{\bf x}_i^T\big({\bf z}_k-N{\bf
p}^{(0)}\big)
\]
with the column vector ${\bf x}_i$ corresponding to the coefficient $\beta_i$.

To account for practical shifts that are in the effect level, the  hypothesis should
be
\begin{align}\label{F3.14}
&H_0:\;\bm{\beta}=\bm{\beta}^{(0)}\\\nonumber
&H_1:\;\bm{\beta}=\bm{\beta}^{(0)}+{\bf D}_1\bm\delta_1
\;\mathrm{or}\;\;\bm{\beta}=\bm{\beta}^{(0)}+{\bf D}_2\bm\delta_2\ldots\;
\mathrm{or}\;\;\bm{\beta}=\bm{\beta}^{(0)}+{\bf D}_{2^p-1}\bm\delta_{2^p-1}.
\end{align}
Here ${\bf D}_i$ of size $(h-1)\times q_i$ is the direction matrixes, with only one
1 and elsewhere 0 in each column corresponding to an element in the increment
subvector $\bm\delta_i$ of size $q_i\times 1$, the same as that of $\bm\beta_i$. We
can derive the GLRT statistic for testing hypothesis (\ref{F3.14}) is
\begin{align}
T_k=\max_{i\in\{1,\ldots,2^p-1\}}T_{i,k},\label{F3.15}
\end{align}
where
\[
T_{i,k}=\frac{1}{N}\big({\bf z}_k-N{\bf p}^{(0)}\big)^T{\bf X}_i\big({\bf
X}_i^T\bm{\Sigma}^{(0)}{\bf X}_i\big)^{-1}{\bf X}_i^T\big({\bf z}_k-N{\bf
p}^{(0)}\big)
\]
with the design submatrix ${\bf X}_i$ corresponding to the subvector $\bm\beta_i$.

However, there is also another problem of using $T_k$. If at least one factor has
more than two levels, for instance, in the $2\times 3$ contingency table, the
coefficient subvectors are $\bm\beta_1=\beta_1$, $\bm\beta_2=[\beta_2,\beta_3]^T$,
and $\bm\beta_3=[\beta_4,\beta_5]^T$. The alternative hypothesis in (\ref{F3.14}) is
actually
\[
H_1:\;\bm\delta_1=\delta_1\neq
0\;\;\mathrm{or}\;\;\bm\delta_2=[\delta_2,\delta_3]^T\neq \bm
0_2\;\;\mathrm{or}\;\;\bm\delta_3=[\delta_4,\delta_5]^T\neq\bm 0_2,
\]
which includes one 1-D real-line and two 2-D planes except their origins. On the
other hand, the alternative hypothesis in (\ref{F3.3}) is
\[
H_1:\;\delta_1\neq 0\;\;\mathrm{or}\;\;\delta_2\neq 0\;\;\mathrm{or}\;\;\delta_3\neq
0\;\;\mathrm{or}\;\;\delta_4\neq 0\;\;\mathrm{or}\;\;\delta_5\neq 0,
\]
which includes five 1-D real-lines except their origins. We see that the alternative
hypothesis in (\ref{F3.14}) considers a much larger difference space than that in
(\ref{F3.3}), which is
\[
S=\{(\delta_2,\delta_3)|\delta_2\neq 0\ \mathrm{and}\ \delta_3\neq
0\}\cup\{(\delta_4,\delta_5)|\delta_4\neq 0\ \mathrm{and}\ \delta_5\neq 0\}.
\]
Therefore, if a shift occurs in $\bm\beta_2$ and its elements $\beta_2$ and
$\beta_3$ change simultaneously, the charting statistic $T_k$ in (\ref{F3.15})
should be more powerful to detect this shift in terms of a smaller ARL than $R_k$ in
(\ref{F3.13}). This follows because this shift in $\bm\beta_2$ leading to
$\delta_2\neq 0\ \mathrm{and}\ \delta_3\neq 0$ is in the difference space $S$.
However, for a shift arising in $\bm\beta_1=\beta_{(1)}$, $T_k$ should have a larger
ARL than $R_k$, since the shift in $\bm\beta_1$ resulting in $\delta_1\neq 0$ is not
in $S$.

By simulation, we have found that if the only shift does not occur in such a
difference space like $S$, $T_k$ leads to a much larger ARL in detecting it than
$R_k$. On the other hand, if the only shift occurs in the difference space, $T_k$
leads to an ARL that is only sightly smaller than $R_k$. This is because a shift in
the difference space is reflected in its edges to a fairly large extent. For
instance, in the $2\times 3$ contingency table, a shift $(\delta_2,\delta_3)$
$(\delta_2\neq 0\ \mathrm{and}\ \delta_3\neq 0)$ in $\bm\beta_2$ is reflected mostly
by $(\delta_2,0)$ and $(0,\delta_3)$. In addition, the alternative hypothesis in
(\ref{F3.3}) is actually a collection of all the edges of the space considered in
the alternative hypothesis in (\ref{F3.14}). So $R_k$ can still detect it faster.
Therefore, in spite of practical shifts at the effect level, we still use the
original charting statistic $R_k$, which only takes a relatively small expense in
detecting shifts in the case of at least one factor with more than two levels.

\chapter{Phase I Analysis by Change-Point Detection}\label{chp4}


\section{Introduction}\label{sec4.1}

It is known that statistical process control involves two phases. In Phase I, a set
of process data are collected and examined. If there are any unusual patterns such
as change-points or outliers, they are identified and adjusted, resulting in a IC
dataset used for estimating the IC model. Phase I analysis primarily assists in
bringing the data into a state of statistical control. After this retrospective
analysis, quality inspectors will have a good reference IC model. This provides the
basis for constructing control charts in Phase II, in which detecting shifts in the
online process is the main objective.

It should be noted that the proposed control charts such as the LMC and the LLD
charts in previous chapters as well as those in the literature used in Phase II SPC
all assume that the process model is representative of the IC operating conditions,
and that its parameters have been estimated from an IC reference dataset. However,
in reality this is not always the case, and any unusual patterns such as
change-points or outliers in the reference dataset could yield an erroneous IC model
estimate. Employing control charts built on such models for online monitoring may be
misleading. Therefore, Phase I analysis is crucial.

This chapter presents an approach of detecting change-points for Phase I analysis in
multivariate categorical processes. Based on log-linear models, our suggested method
applies to the unified framework of multivariate binomial and multivariate
multinomial processes and considers factor interactions besides main effects.
Furthermore, to detect change-points as powerfully as possible, the method also
integrates directional shift information as with the LLD chart. This follows that at
the change-point, shifts that arise in factor effects should result in deviations of
their corresponding coefficient subvectors. A concomitant diagnostic scheme is also
developed to identify the change-point location and the shift direction upon
change-point detection. Monte Carlo simulations are performed to demonstrate the
effectiveness of the change-point detection and diagnostic approach.



\section{Two-Sample Multivariate Categorical Test Problem}\label{sec4.2}

Before introducing the proposed change-point detection method, we need first to
state the two-sample multivariate categorical test in the framework of multivariate
binomial and multivariate multinomial distributions. This is the basis of the
change-point problem. Given a $p$-way contingency table with $h$ cells, consider the
two multinomial distributions $\mbox{MN}(N_{\mathrm{a}};{\bf p}_{\mathrm{a}})$ and
$\mbox{MN}(N_{\mathrm{b}};{\bf p}_{\mathrm{b}})$, where ${\bf p}_{\mathrm{a}}$ and
${\bf p}_{\mathrm{b}}$ are both of size $h\times 1$ and
\[
\ln{\bf p}_{\mathrm{a}}=\bm 1\beta^{(\mathrm{a})}_0+{\bf
X}\bm\beta^{(\mathrm{a})},\quad \ln{\bf p}_{\mathrm{b}}=\bm
1\beta^{(\mathrm{b})}_0+{\bf X}\bm\beta^{(\mathrm{b})}.
\]
We want to test the following hypothesis
\begin{align}\label{F4.1}
H_0:\;{\bf p}_{\mathrm{a}}={\bf p}_{\mathrm{b}}\ \ \mbox{versus}\ \ H_1:\;{\bf
p}_{\mathrm{a}}\ne{\bf p}_{\mathrm{b}}.
\end{align}
According to the employed log-linear model (\ref{F3.1})
\[
\ln {\bf
p}=\bm{1}\beta_0+\sum_{i=1}^{h-1}{\bf x}_i\beta_i,
\]
$\beta^{(t)}_0$ ($t=\mathrm{a},\mathrm{b}$) can be determined by $\bm\beta^{(t)}$
(as $\bm 1^T{\bf p}=1$), hypothesis (\ref{F4.1}) is equivalent to
\begin{align}\label{F4.2}
H_0:\;\bm{\beta}^{(\mathrm{b})}=\bm{\beta}^{(\mathrm{a})}\ \ \mbox{versus}\ \
H_1:\;\bm{\beta}^{(\mathrm{b})}\ne\bm{\beta}^{(\mathrm{a})}.
\end{align}
Assume that ${\bf n}_{\mathrm{a}}$ and ${\bf n}_{\mathrm{b}}$ are two independent
random vectors subject to $\mbox{MN}(N_{\mathrm{a}};{\bf p}_{\mathrm{a}})$ and
$\mbox{MN}(N_{\mathrm{b}};{\bf p}_{\mathrm{b}})$, respectively. We have $\bm 1^T{\bf
n}_{\mathrm{a}}=N_{\mathrm{a}}$ and $\bm 1^T{\bf n}_{\mathrm{b}}=N_{\mathrm{b}}$.
The joint log-likelihood function of ${\bf p}_{\mathrm{a}}$ and ${\bf
p}_{\mathrm{b}}$ can be written from the PMF of the multinomial distribution as
\begin{align*}
l&={\bf n}_{\mathrm{a}}^T\ln{\bf p}_{\mathrm{a}}+\ln N_{\mathrm{a}}!-\bm 1^T\ln{\bf
n}_{\mathrm{a}}!+{\bf n}_{\mathrm{b}}^T\ln{\bf p}_{\mathrm{b}}+\ln
N_{\mathrm{b}}!-\bm 1^T\ln{\bf n}_{\mathrm{b}}!.
\end{align*}
A natural test for hypothesis (\ref{F4.1}) or (\ref{F4.2}) may be constructed using
the generalized likelihood ratio test (GLRT; Anderson (2003)). By some derivations,
it is easy to see that the MLEs of ${\bf p}_{\mathrm{a}}$ and ${\bf p}_{\mathrm{b}}$
under $H_0$ in (\ref{F4.1}) are both equal to $({\bf n}_{\mathrm{a}}+{\bf
n}_{\mathrm{b}})/(N_{\mathrm{a}}+N_{\mathrm{b}})$, whereas under $H_1$ they are
${\bf n}_{\mathrm{a}}/N_{\mathrm{a}}$ and ${\bf n}_{\mathrm{b}}/N_{\mathrm{b}}$,
respectively. So the $-2$LRT statistic is
\begin{align}\label{F4.3}
\theta&=2{\bf n}_{\mathrm{a}}^T\ln\Big(\frac{{\bf
n}_{\mathrm{a}}}{N_{\mathrm{a}}}\Big)+2{\bf n}_{\mathrm{b}}^T\ln\Big(\frac{{\bf
n}_{\mathrm{b}}}{N_{\mathrm{b}}}\Big)-2({\bf n}_{\mathrm{a}}+{\bf
n}_{\mathrm{b}})^T\ln\Big(\frac{{\bf n}_{\mathrm{a}}+{\bf
n}_{\mathrm{b}}}{N_{\mathrm{a}}+N_{\mathrm{b}}}\Big).
\end{align}

Hypotheses (\ref{F4.1}) and (\ref{F4.2}) state that $\bm{\beta}^{(\mathrm{b})}$
differs from $\bm{\beta}^{(\mathrm{a})}$ in all coefficients. With the practical
definition of shifts, however, it is usually reasonable to assume that in real
processes shifts occur in only a few coefficient subvectors or only a few
coefficients, reflecting the changes in their corresponding main effects or
interaction effects. Suppose that there is some prior knowledge that a shift arises
in only one coefficient in $\bm{\beta}^{(\mathrm{a})}$, say the $i$th ($1\leq i\leq
h-1$) coefficient, by adding an unknown constant $\delta_i$ to it. Then
$\widetilde{\bm\beta}^{(\mathrm{a})}$ and $\widetilde{\bm\beta}^{(\mathrm{b})}$ have
a relationship such that
\begin{align}
\left\{
\begin{array}{rll}
\beta^{(\mathrm{b})}_i&\!\!\!\!=\beta^{(\mathrm{a})}_i+\delta_i&\mathrm{for\ some}\ i,\mathrm{and}\ 1\leq i\leq h-1,\\
\beta^{(\mathrm{b})}_j&\!\!\!\!=\beta^{(\mathrm{a})}_j&\mathrm{for\ all}\ j\neq i, \mathrm{and}\ 1\leq j\leq h-1,\\
\beta^{(\mathrm{b})}_0&\!\!\!\!=\beta^{(\mathrm{a})}_0+\alpha_i,&
\end{array}
\right.\label{F4.4}
\end{align}
where $\beta^{(t)}_i$ is the $i$th element of $\bm\beta^{(t)}$
($t=\mathrm{a},\mathrm{b}$), and $\alpha_i$ is also an unknown constant. Here
$\alpha_i$ is actually the increment of $\beta^{(\mathrm{a})}_0$ induced by the
constraint $\bm{1}^T{\bf p}=1$. Now the hypothesis can be expressed as
\begin{align}\label{F4.5}
H_0:\;\bm{\beta}^{(\mathrm{b})}=\bm{\beta}^{(\mathrm{a})}\ \ \mbox{versus}\ \
H_1:\;\bm{\beta}^{(\mathrm{b})}=\bm{\beta}^{(\mathrm{a})}+{\bf d}_i\delta_i,
\end{align}
where ${\bf d}_i$ is the direction vector of size $(h-1)\times 1$ with 1 at its
$i$th element and 0s elsewhere.

Hypothesis (\ref{F4.5}) assumes that the location of the only dissimilar coefficient
between $\bm\beta^{(\mathrm{a})}$ and $\bm\beta^{(\mathrm{b})}$ is known. Here we
consider a more practical and general case where compared to
$\bm\beta^{(\mathrm{a})}$, $\bm\beta^{(\mathrm{b})}$ deviates in only one
coefficient, but its location is unknown. So the original alternative hypothesis in
(\ref{F4.2}) becomes
\begin{align}
H_1:\;\bm{\beta}^{(\mathrm{b})}=\bm{\beta}^{(\mathrm{a})}+{\bf d}_1\delta_1
\;\mathrm{or}\;\;\bm{\beta}^{(\mathrm{b})}=\bm{\beta}^{(\mathrm{a})}+{\bf
d}_2\delta_2\ldots\;
\mathrm{or}\;\;\bm{\beta}^{(\mathrm{b})}=\bm{\beta}^{(\mathrm{a})}+{\bf
d}_{h-1}\delta_{h-1},\label{F4.6}
\end{align}
where $\delta_i$ ($i=1,2,\ldots,h-1$) are the unknown shift magnitudes, and the
finite possible shift direction vectors ${\bf d}_1$, ${\bf d}_2$, \ldots, ${\bf
d}_{h-1}$, which apply to $\beta^{(\mathrm{a})}_1$, $\beta^{(\mathrm{a})}_2$,
\ldots, $\beta^{(\mathrm{a})}_{h-1}$, respectively, are defined similarly. In other
words, hypothesis (\ref{F4.6}) states that $\bm\beta^{(\mathrm{b})}$ may differ from
$\bm\beta^{(\mathrm{a})}$ in only one unknown coefficient, which may be any of the
coefficients from the main factor effects up to the highest $p$-factor interaction
effect. Since the GLRT derived from (\ref{F4.6}) exploits more constructive
information about potential shift directions, it should be more powerful than that
from (\ref{F4.2}).

Let us now generalize hypothesis (\ref{F4.6}). Hypothesis (\ref{F4.6}) considers
shifts in effects of all orders, but it is believed that usually deviations
containing fewer factors occur more frequently. In other words, most shifts appear
in lower-order effects such as main effects and two-factor interaction effects,
instead of higher-order ones. So far we have seen this reasoning is very similar to
that from hypothesis (\ref{F3.2}) to hypothesis (\ref{F3.4}) for deriving the
charting statistic of the LLD chart in Chapter 3. Thus, we may pay sufficient
attention to effects of the first few, say $q$, orders. This allows focusing
detection power on a limited subspace with improved sensitivity. For the log-linear
model (\ref{F3.1}), still denote the set of coefficient indexes corresponding to the
effects of the first $q$ orders by $\mathbb{I}_q$ ($1\leq q\leq p$). Then hypothesis
(\ref{F4.6}) has a coefficient set $\mathbb{I}_p$, and it can further be generalized
as
\begin{align}
H_0:\;\bm{\beta}^{(\mathrm{b})}=\bm{\beta}^{(\mathrm{a})}\ \ \mbox{versus}\ \
H_1:\;\bigcup_{i\in\mathbb{I}_q}\Big(\bm{\beta}^{(\mathrm{b})}=\bm{\beta}^{(\mathrm{a})}+{\bf
d}_i\delta_i\Big),\quad 1\leq q\leq p.\label{F4.7}
\end{align}
If $q=p$, the alternative hypothesis in (\ref{F4.7}) is equivalent to that in
(\ref{F4.6}). Since shifts usually occur in low-order effects, it is clear that the
larger $q$ is, the less powerful the GLRT based on (\ref{F4.7}) will be. This
follows because the alternative hypothesis considers redundant high-order shift
directions. If the real shift indeed appears in an effect of the first $q$ orders,
the GLRT will certainly be powerful. Even if a shift occurs in an effect of an order
higher than $q$, this change will be reflected to a large extent by the derived
$-2$LRT statistic, and the GLRT will still be effective.

To obtain the $-2$LRT statistic $\lambda$ for testing hypothesis (\ref{F4.7}), we
need first to derive the $-2$LRT statistic $\lambda_i$ for testing hypothesis
(\ref{F4.5}), where it is assumed that only the coefficient $\beta_i$ shifts. The
derivation of $\lambda_i$ is a little complex, so it is left in the appendix in this
chapter (Section 4.7) and finalized in Equation (\ref{F4.20}). Given $\lambda_i$,
the $-2$LRT statistic for testing hypothesis (\ref{F4.7}) is therefore
\begin{align}
\lambda=\max_{i\in\mathbb{I}_q}\lambda_i.\label{F4.8}
\end{align}



\section{Directional Change-Point Detection}\label{sec4.3}

Based on the two-sample test with directional shift information, we can build the
directional change-point detection method for Phase I analysis of multivariate
categorical processes. The pre-specified log-linear model is summarized by
$F\big({\bf X}; \bm\beta\big)$ in Equation (\ref{F2.3}). In Phase I analysis, it is
usually reasonable to assume that the given dataset is composed of multiple samples,
say $M$, of size $N$, and each sample forms an observation vector of dimension
$h\times 1$ subject to the multinomial distribution $\mbox{MN}(N;{\bf p})$.
Furthermore, it is also reasonably assumed that the $j$th multivariate sampling
observation vector ${\bf n}_j$ is collected over time from the following
change-point model
\begin{align*}
{\bf n}_j\ {\mathop{\sim}\limits^{\mbox{i.i.d.}}}\left\{\hspace{-0.1cm}
\begin{array}{ll} F\big({\bf X};\bm\beta^{(\mathrm{a})}\big), &
{\mathrm{for\ }} \quad j=1,\ldots,\tau,\\[0.1cm]
F\big({\bf X};\bm\beta^{(\mathrm{b})}\big), & {\mathrm {for\ }} \quad
j=\tau+1,\ldots,M,
\end{array}\right.
\end{align*}
where $\tau$ is the unknown change-point, and $\bm\beta^{(\mathrm{a})} \neq
\bm\beta^{(\mathrm{b})}$ are the unknown before-change and after-change process
coefficient vectors, respectively. Note that hereafter we use the superscripts
``$(\mathrm{a})$'' and ``$(\mathrm{b})$'' to denote the before-change and
after-change cases, respectively.

To detect the change-point, like Srivastava and Worsley (1986) which aimed at
detecting a change-point in the mean vector of a multivariate normal distribution,
one natural idea is to integrate the GLRT with the binary segmentation procedure.
Given the observation vectors ${\bf n}_j$ ($j=1,\ldots,M$), the change-point $\tau$
is unknown and may take any of the values $1,\ldots,M-1$. Therefore, for each
possible $\tau$, we pool both the before-$\tau$ and after-$\tau$ samples together to
form two large samples, respectively, and based on them compare the process
coefficient vectors $\widetilde{\bm{\beta}}^{(\mathrm{a})}$ and
$\widetilde{\bm{\beta}}^{(\mathrm{b})}$ by performing the two-sample test for
hypothesis (\ref{F4.2}) in multivariate categorical distributions. If the maximum of
the $M-1$ statistics is large enough, there will be a change-point during the
process. To be more specific, let
\begin{align}\label{F4.9}
{\bf n}^{\mathrm{A}}_k=\sum_{j=1}^k{\bf n}_j,\quad\ &
{\bf n}^{\mathrm{B}}_k=\sum_{j=k+1}^M{\bf n}_j,\nonumber\\
N^{\mathrm{A}}_k=kN,\quad\quad\quad & N^{\mathrm{B}}_k=(M-k)N.
\end{align}
Actually, ${\bf n}^{\mathrm{A}}_k$ and ${\bf n}^{\mathrm{B}}_k$ can be regarded as
being collected from $\mbox{MN}(N^{\mathrm{A}}_k;{\bf p}_{\mathrm{a}})$ and
$\mbox{MN}(N^{\mathrm{B}}_k;{\bf p}_{\mathrm{b}})$, respectively, where the
probability vectors ${\bf p}_{\mathrm{a}}$ and ${\bf p}_{\mathrm{b}}$ are determined
by $\widetilde{\bm{\beta}}^{(\mathrm{a})}$ and
$\widetilde{\bm{\beta}}^{(\mathrm{b})}$, respectively. Clearly, this results in a
two-sample test problem. By replacing ${\bf n}_{\mathrm{a}}$, ${\bf
n}_{\mathrm{b}}$, $N_{\mathrm{a}}$, and $N_{\mathrm{b}}$ in (\ref{F4.3}) by ${\bf
n}^{\mathrm{A}}_k$, ${\bf n}^{\mathrm{B}}_k$, $N^{\mathrm{A}}_k$, and
$N^{\mathrm{B}}_k$ in (\ref{F4.9}), respectively, we obtain a $-2$LRT statistic
$\Theta_k$. If the maximum
\begin{align*}
\Theta=\max_{k\in\{1,\ldots,M-1\}}\Theta_k
\end{align*}
exceeds a critical value, an alarm will sound for a change-point. Accordingly, the
change-point $\tau$ may be estimated as
\begin{align}
\hat{\tau}=\mathop{\arg\max}_{k\in\{1,\ldots,M-1\}}\Theta_k.\label{F4.10}
\end{align}

The above change-point detection approach is based on the two-sample test method
without considering directional shift information. In reality, however, practical
shifts tend to appear in only a few effects, leading to deviations of only a few
coefficient subvectors. In change-point detection, such knowledge about the most
likely shift directions should be exploited sufficiently. To this end, we combine
the $-2$LRT statistic in Equation (\ref{F4.8}) for testing hypothesis (\ref{F4.7})
with the foregoing binary segmentation method. This leads to the change-point
detection test statistic
\[
\Lambda=\max_{i\in\mathbb{I}_q}\Lambda_i=\max_{i\in\mathbb{I}_q}\max_{k\in\{1,\ldots,M-1\}}\Lambda_{i,k},
\]
where $\Lambda_{i,k}$ is obtained in a similar fashion to $\lambda_i$ in Equation
(\ref{F4.8}) by using ${\bf n}^{\mathrm{A}}_k$, ${\bf n}^{\mathrm{B}}_k$,
$N^{\mathrm{A}}_k$, and $N^{\mathrm{B}}_k$ in (\ref{F4.9}).

There remains an issue about the choice of $\mathbb{I}_q$. With the same reasons as
the choice of $\mathbb{I}_q$ for the LLD chart, we recommend considering about only
main effects and two-factor interaction effects and let $\mathbb{I}_q$ be
$\mathbb{I}_2$. If the shift indeed occurs in a main effect or a two-factor
interaction effect, the GLRT for testing hypothesis (\ref{F4.7}) with $\mathbb{I}_2$
is definitely powerful. Further, even if a shift appears in a three-factor or
higher-order interaction effect, the effects of the first two orders will be
affected to a fairly large extent, so it would still be detected powerfully.
Therefore, the test statistic for directional change-point detection can be
finalized as
\begin{align}
\Lambda=\max_{i\in\mathbb{I}_2}\Lambda_i
=\max_{i\in\mathbb{I}_2}\max_{k\in\{1,\ldots,M-1\}}\Lambda_{i,k}
=\max_{k\in\{1,\ldots,M-1\}}\max_{i\in\mathbb{I}_2}\Lambda_{i,k}.\label{F4.11}
\end{align}

Similar to the change-point estimation in (\ref{F4.10}), a relevant diagnostic
approach could also be developed based on the directional detection method, which
aims at identifying both the change-point and the shift direction. However, unlike
the detection method which employs an index subset $\mathbb{I}_2$ and can only tell
whether there is a change-point or not, the diagnosis needs to recognize the shift
direction, which may indeed be in an effect of an order higher than two. We still
refer to the choice of the candidate diagnostic set upon a signal of the LLD chart
in Section \ref{sec3.4}. Therefore, the candidate diagnostic subset of shift
directions here is suggested as $\mathbb{I}_3$. Accordingly, the change-point $\tau$
can be estimated as
\begin{align}
\hat{\tau}=\mathop{\arg\max}_{k\in\{1,\ldots,M-1\}}\Big(\max_{i\in\mathbb{I}_3}\Lambda_{i,k}\Big).\label{F4.12}
\end{align}
Based on the estimated change-point $\hat{\tau}$, the shift direction $\zeta$ can
further be identified as
\begin{align}
\hat{\zeta}=\arg\max_{i\in\mathbb{I}_3}\Lambda_{i,\hat{\tau}}.\label{F4.13}
\end{align}



\section{Approximation of Significance Levels}\label{sec4.4}

The derived $-2$LRT statistics $\Theta$ and $\Lambda$ for change-point detection
need to be compared with their critical values. However, since the exact
distributions of these statistics seem impossible to find, in this section we
suggest a suitable approach to approximate their critical values or equivalently
significance levels.

Since $\Lambda=\max_{i\in\mathbb{I}_2}\Lambda_i$ in Equation (\ref{F4.11}), the
significance level of $\Lambda$ can be approximated based on that of $\Lambda_i$.
The significance levels of $\Theta$ and $\Lambda_i$ are in fact approximated in the
same way. Here $\Theta=\max_k\Theta_k$ is the maximal LRT statistic without
integrating directional information, and $\Theta_k$ is the $-2$LRT statistic for
testing hypothesis (\ref{F4.2}) where $\widetilde{\bm{\beta}}^{(\mathrm{b})}$
differs from $\widetilde{\bm{\beta}}^{(\mathrm{a})}$ in $h-1$ independent
parameters. Similarly, $\Lambda_i=\max_k\Lambda_{i,k}$ is the maximal LRT statistic
with a given shift direction, and $\Lambda_{i,k}$ is the $-2$LRT statistic for
testing hypothesis (\ref{F4.5}) where $\widetilde{\bm{\beta}}^{(\mathrm{b})}$
deviates from $\widetilde{\bm{\beta}}^{(\mathrm{a})}$ in only one independent
parameter. So the difference between $\Theta$ and $\Lambda_i$ lies mainly in the
number of changed parameters. Let
\begin{align*}
& A(x)=(2\ln x)^{\frac{1}{2}}\quad\mathrm{and}\\
& D_d(x)=2\ln x+\frac{d}{2}\ln\ln x-\ln\Gamma\Big(\frac{d}{2}\Big),
\end{align*}
where $\Gamma(x)$ is the Gamma function. According to Theorem 1.3.1 of
Cs\"{o}rg\H{o} and Horv\'{a}th (1997), we have the following proposition for the
asymptotic null distribution of a maximal LRT statistic $Z$ such as $\Theta$ or
$\Lambda_i$ in change-point problems.
\begin{pro}
If the process is statistically IC, we have
\[
\lim_{M\to\infty}\mathrm{Pr}\Big(A(\ln M)Z^{\frac{1}{2}}\leq x+D_d(\ln
M)\Big)=\exp(-2e^{-x})
\]
with $d=h-1$ for $Z=\Theta$ and $d=1$ for $Z=\Lambda_i$ $(i\in\mathbb{I}_2)$.
\end{pro}

We can use Proposition 4.1 to obtain the asymptotic $p$-values of $\Theta$ and
$\Lambda_i$. However, it is believed that their rates of convergence are very slow,
and we may need a very large number $M$ of samples if we want to employ Proposition
4.1 for change-point detection. From the discussions in Section 1.3 of
Cs\"{o}rg\H{o} and Horv\'{a}th (1997), usually, if $M$ is not large enough, using
Proposition 4.1 results in a much larger critical value and therefore a rejection
region that is too conservative. Fortunately, based on its Theorem 1.3.2,
Cs\"{o}rg\H{o} and Horv\'{a}th (1997) also recommended another more accurate form
for approximating the $p$-value of a maximal LRT statistic $Z$ such as $\Theta$ or
$\Lambda_i$. This form is
\begin{align}
\mathrm{Pr}\big(Z^{\frac{1}{2}}>x\big)=\frac{x^d\exp(-x^2/2)}{2^{d/2}\Gamma(d/2)}
\Big(\ln s-\frac{d}{x^2}\ln s+\frac{4}{x^2}\Big)\label{F4.14}
\end{align}
with $d=h-1$ for $Z=\Theta$ and $d=1$ for $Z=\Lambda_i$ $(i\in\mathbb{I}_2)$, where
$s=(1-b_1)(1-b_2)/(b_1b_2)$ and $b_1=b_2=(\ln M)^{\frac{3}{2}}/M$.

Based on the approximate $p$-value of $\Lambda_i$, let us now turn to the
significance level of $\Lambda=\max_{i\in\mathbb{I}_2}\Lambda_i$. Generally, these
$\Lambda_i$s are correlated, and therefore the analytical form of
$\mathrm{Pr}(\Lambda>c)$ seems rather difficult to obtain. One simple and natural
idea is to use the classical Bonferroni procedure in terms of rejecting $\Lambda\geq
c$ with a desired type I error $\alpha$ if any
$\mathrm{Pr}(\Lambda_i>c)\leq\alpha/K$ ($i\in\mathbb{I}_2$), where $K$ is the
cardinality of the subset $\mathbb{I}_2$. If $K$ is small, the Bonferroni procedure
would have a false alarm rate close to $\alpha$. However, it is rather conservative
if $K$ is large and the $\Lambda_i$s are highly correlated.

Hence, we recommend using the famous modified Bonferroni procedure (Simes (1986)),
which is less conservative than the classical one but still simple to apply. In
addition, the modified procedure is superior to the classical one when the test
statistics involved are highly correlated. The modified Bonferroni procedure for
these $\Lambda_i$ ($i\in\mathbb{I}_2$) are summarized as follows:
\begin{description}
\item[Step 1.] Calculate $\Lambda_i$ for each
$i\in\mathbb{I}_2$ based on $\Lambda_{i,k}$ ($k=1,\ldots,M-1$);
\item[Step 2.] Use Equation (\ref{F4.14}) with $d=1$ to compute the
approximate $p$-values $\hat{p}_i$ for each $\Lambda_i$, $i\in\mathbb{I}_2$;
\item[Step 3.] Given a desired type I error $\alpha$, reject
the null hypothesis if $\hat{p}_{(i)}\leq i\alpha/K$ for at least one $i$, where
$\hat{p}_{(1)}\leq\ldots\leq\hat{p}_{(K)}$ are the ordered values of
$\hat{p}_{1},\ldots,\hat{p}_{K}$.
\end{description}

%Table 1
\begin{table}[!ht]
\tabcolsep 8.5pt \vspace{-0.1cm} \centering \caption{Performance of the modified
Bonferroni procedure in approximating the significance levels of $\Lambda$}
\vspace{0.3cm}
\renewcommand{\arraystretch}{1.25}
\begin{tabular}{c|c|ccc}\hline
$\alpha$ & $M$ & $i\in\mathbb{I}_1$, $K=5$ & $i\in\mathbb{I}_2$, $K=14$ &
$i\in\mathbb{I}_3$, $K=21$
\\\hline
& 40 & 0.045 & 0.036 & 0.035\\
0.05 & 80 & 0.046 & 0.044 & 0.042\\
& 120 & 0.051 & 0.046 & 0.042
\\\hline
& 40 & 0.089 & 0.077 & 0.075\\
0.10 & 80 & 0.096 & 0.084 & 0.079\\
& 120 & 0.101 & 0.089 & 0.082
\\\hline
\end{tabular}
\end{table}

Table 4.1 lists the approximate false alarm rate $\hat{\alpha}$ of $\Lambda$ in
various cases, given a desired type I error $\alpha$ and simulated by applying the
modified Bonferroni procedure and Equation (\ref{F4.14}). Here we consider four
factors with 2, 2, 2, and 3 levels. In addition, for simulation purpose, we also
consider shift directions with indexes in the subsets $\mathbb{I}_1$,
$\mathbb{I}_2$, and $\mathbb{I}_3$ with their cardinalities $K=$ 5, 14, and 21,
respectively. The results are based on 5,000 replicated simulations. Clearly, the
false alarm rate $\hat{\alpha}$ increases as the cardinality $K$ of the selected
index subset decreases and the number $M$ of samples increases. These simulated
$\hat{\alpha}$s are close to their desired value $\alpha$. Even for the worst case
of $K=21$ and $M=40$, the method still produces a good approximation.



\section{Performance Assessment}\label{sec4.5}

In this section, we compare the powers of the two change-point detection methods,
which are based on the two-sample test with and without integrating directional
shift information. Then we investigate the corresponding diagnostic performance of
the two approaches in identifying the change-point $\tau$ and the shift direction
$\zeta$. In addition, by revisiting the AEC example introduced previously in Section
1.1.2, we illustrate the implementation of the proposed directional change-point
detection and diagnostic approaches.

Assume that during a production process there are four quality characteristics with
the first three assessed as conforming or nonconforming and the last one evaluated
as excellent, acceptable, or unacceptable. This leads to a multivariate multinomial
process with four factors $C_1$, $C_2$, $C_3$, and $C_4$, which can be arranged into
a four-way contingency table of size $2\times2\times2\times3$. Also assume that
before the change-point $\tau$, the log-linear model has the coefficient vector
\begin{eqnarray}
\begin{array}{rrrrrrrrrl}
\widetilde{\bm{\beta}}^{(\mathrm{a})}=[ & \beta_0 & 0.86 & 0.89
& 0.82 & 0.72 & 0.08 & \\
& 0.10 & 0.12 & 0.12 & -0.13 & 0.10 & -0.06 & \\
& 0.07 & 0.16 & -0.14 & 0.13 & -0.10 & -0.08 & \\
& -0.04 & -0.07 & -0.11 & -0.05 & 0 & 0 & ]^T,
\end{array}
\nonumber
\end{eqnarray}
where $\beta_0$ is the intercept accommodating the constraint $\bm{1}^T{\bf p}=1$.
Based on $\widetilde{\bm{\beta}}^{(\mathrm{a})}$, the before-change probability
vector ${\bf p}_{\mathrm{a}}$ can be further calculated. Assume that throughout the
simulations there are $M=80$ samples of size $N=1200$, and the change-point $\tau$
is set as 30.


\subsection{Power Comparison}

%Table 2
\begin{table}[!ht]
\tabcolsep 7.5pt \vspace{-0.1cm} \centering \caption{Power comparison for
one-coefficient shifts of the first two orders} \vspace{0.3cm}
\renewcommand{\arraystretch}{1.25}
\begin{tabular}{c|cc|cc|cc}\hline
$\delta$ & With & Without & With & Without & With & Without
\\\hline
& \multicolumn{2}{c|}{$\beta_{(2)}$} & \multicolumn{2}{c|}{$\beta_{(4_1)}$} &
\multicolumn{2}{c}{$\beta_{(1,3)}$}
\\\hline
0.02 & 0.121 & 0.081 & 0.210 & 0.107 & 0.171 & 0.100 \\
0.03 & 0.332 & 0.151 & 0.555 & 0.239 & 0.486 & 0.221 \\
0.04 & 0.633 & 0.314 & 0.876 & 0.503 & 0.847 & 0.488 \\
0.05 & 0.888 & 0.553 & 0.985 & 0.809 & 0.981 & 0.785 \\
0.06 & 0.985 & 0.804 & 1.000 & 0.967 & 0.999 & 0.950 \\
0.07 & 0.998 & 0.944 & 1.000 & 0.999 & 1.000 & 0.994
\\\hline
& \multicolumn{2}{c|}{$\beta_{(1,4_2)}$} & \multicolumn{2}{c|}{$\beta_{(2,3)}$} &
\multicolumn{2}{c}{$\beta_{(3,4_1)}$}
\\\hline
0.02 & 0.130 & 0.085 & 0.192 & 0.102 & 0.214 & 0.110 \\
0.03 & 0.339 & 0.154 & 0.532 & 0.247 & 0.607 & 0.273 \\
0.04 & 0.673 & 0.327 & 0.875 & 0.532 & 0.910 & 0.574 \\
0.05 & 0.908 & 0.562 & 0.988 & 0.832 & 0.993 & 0.870 \\
0.06 & 0.986 & 0.825 & 0.999 & 0.969 & 1.000 & 0.984 \\
0.07 & 0.999 & 0.953 & 1.000 & 0.998 & 1.000 & 0.999
\\\hline
\multicolumn{7}{l}{$\hat{\alpha}=0.044$ for ``With'' and 0.048 for ``Without''.}
\end{tabular}
\end{table}

The desired type I error $\alpha$ is set as 0.05 for both detection methods. The
reported results are based on 5,000 replicated simulations. As mentioned previously,
shifts in the marginal distribution of one factor are represented by deviations of
its main effect or the corresponding coefficient subvector, and shifts in the
dependence among multiple factors are reflected by deviations of their interaction
effect or the corresponding coefficient subvector. So let us consider first that at
the change-point $\tau$, only one coefficient, say $\beta_i$ ($i\in\mathbb{I}_2$),
in $\bm{\beta}^{(\mathrm{a})}$ adds with an increment $\delta_i$. In other words,
this one-coefficient shift occurs in a main effect or a two-factor interaction
effect. Table 4.2 tabulates the powers of the two change-point detection approaches
``with'' and ``without'' the knowledge of shift directions. Hereafter, the two
methods are represented by ``With'' and ``Without'' for short in the tables. Note
that the approximate false alarm rate $\hat{\alpha}$ of the directional method is
0.044, a little smaller than the $\hat{\alpha}$ 0.048 of the undirectional one. In
spite of this, in the case of one-coefficient shifts such as deviations of
$\beta_{(2)}$, $\beta_{(4_1)}$, $\beta_{(1,3)}$, $\beta_{(1,4_2)}$, $\beta_{(2,3)}$,
and $\beta_{(3,4_1)}$, the power of the directional method is uniformly greater than
that of the undirectional approach. This is expected, since the directional method
takes advantage of the information about one-coefficient shifts in the effects of
the first two orders, whereas the undirectional approach has to consider all kinds
of shifts.

%Table 3
\begin{table}[!ht]
\tabcolsep 7.5pt \vspace{-0.1cm} \centering \caption{Power comparison for
one-coefficient shifts of the third order} \vspace{0.3cm}
\renewcommand{\arraystretch}{1.25}
\begin{tabular}{c|cc|cc|cc}\hline
$\delta$ & With & Without & With & Without & With & Without
\\\hline
& \multicolumn{2}{c|}{$\beta_{(1,2,3)}$} & \multicolumn{2}{c|}{$\beta_{(1,3,4_2)}$}
& \multicolumn{2}{c}{$\beta_{(2,3,4_1)}$}
\\\hline
0.02 & 0.157 & 0.116 & 0.102 & 0.075 & 0.197 & 0.125 \\
0.03 & 0.406 & 0.289 & 0.229 & 0.151 & 0.490 & 0.315 \\
0.04 & 0.733 & 0.623 & 0.461 & 0.307 & 0.817 & 0.644 \\
0.05 & 0.939 & 0.900 & 0.721 & 0.571 & 0.975 & 0.920 \\
0.06 & 0.992 & 0.989 & 0.907 & 0.819 & 0.999 & 0.992 \\
0.07 & 1.000 & 1.000 & 0.979 & 0.959 & 1.000 & 1.000
\\\hline
\multicolumn{7}{l}{$\hat{\alpha}=0.044$ for ``With'' and 0.048 for ``Without''.}
\end{tabular}
\end{table}

Let us now turn to the performance of the two methods if at the change-point $\tau$
the only shift occurs in a three-factor interaction effect. The results are given in
Table 4.3, where shifts in $\beta_{(1,2,3)}$, $\beta_{(1,3,4_2)}$, and
$\beta_{(2,3,4_1)}$ are considered. Since shifts in a third-order effect can be
reflected to a large extent by changes in the effects of the first two orders, Table
4.3 shows that the directional method still uniformly outperforms the undirectional
one.

We come next to the case of two-coefficient shifts to verify the robustness of the
directional method, which specializes in detecting shifts in only one coefficient.
Here two coefficients increase by $\delta_1$ and $\delta_2$, e.g.,
$\beta_{(1)}+\delta_1$ and $\beta_{(4_2)}+\delta_2$ simultaneously at the
change-point $\tau$. Table 4.4 demonstrates the powers of the two detection
approaches under different combinations of coefficients of the first three orders.
We can see that the directional method is still more powerful than the undirectional
one, reflecting a robustness to shifts in more than one coefficient.

%Table 4
\begin{table}[!ht]
\tabcolsep 5.5pt \vspace{-0.1cm} \centering \caption{Power comparison for
two-coefficient shifts of the first three orders} \vspace{0.3cm}
\renewcommand{\arraystretch}{1.25}
\begin{tabular}{cc|cc|cc|cc|cc}\hline
\ $\delta_1$\  & \ $\delta_2$\  & With & Without & With & Without & With & Without &
With & Without
\\\hline
& & $\beta_{(1)}$ & $\beta_{(4_2)}$ & $\beta_{(2)}$ & $\beta_{(1,4_1)}$ &
$\beta_{(3)}$ & $\beta_{(2,3)}$ & $\beta_{(1,2)}$ & $\beta_{(1,3)}$
\\\hline
0.02 & 0.02 & 0.180 & 0.115 & 0.285 & 0.162 & 0.599 & 0.290 & 0.589 & 0.303 \\
0.02 & 0.04 & 0.651 & 0.358 & 0.915 & 0.653 & 0.987 & 0.839 & 0.973 & 0.804 \\
0.04 & 0.02 & 0.554 & 0.335 & 0.707 & 0.450 & 0.957 & 0.715 & 0.983 & 0.833 \\
0.04 & 0.04 & 0.799 & 0.634 & 0.960 & 0.867 & 1.000 & 0.986 & 1.000 & 0.990
\\\hline
& & $\beta_{(2,3)}$ & $\beta_{(3,4_1)}$ & $\beta_{(2,4_1)}$ & $\beta_{(2,4_2)}$ &
$\beta_{(1,2,4_1)}$ & $\beta_{(1,2,4_2)}$ & $\beta_{(1,3,4_1)}$ &
$\beta_{(2,3,4_1)}$
\\\hline
0.02 & 0.02 & 0.455 & 0.256 & 0.532 & 0.255 & 0.446 & 0.263 & 0.783 & 0.504 \\
0.02 & 0.04 & 0.959 & 0.787 & 0.914 & 0.647 & 0.852 & 0.691 & 0.996 & 0.970 \\
0.04 & 0.02 & 0.935 & 0.753 & 0.986 & 0.832 & 0.952 & 0.857 & 0.996 & 0.966 \\
0.04 & 0.04 & 0.997 & 0.974 & 0.999 & 0.977 & 0.995 & 0.984 & 1.000 & 0.999
\\\hline
\multicolumn{10}{l}{$\hat{\alpha}=0.044$ for ``With'' and 0.048 for ``Without''.}
\end{tabular}
\end{table}

\subsection{Diagnostic Performance}

The diagnostic performance is investigated in this subsection, under the same
parameter settings as in last section. The change-point $\tau$ is estimated based on
Equation (\ref{F4.10}) for the method without directional shift information and on
Equation (\ref{F4.12}) for the one with such information. These two schemes are
compared in terms of the averages (Avg) and the standard deviations (Std) of the
change-point estimates $\hat{\tau}$, and the probabilities
$\mathrm{Pr}(|\hat{\tau}-\tau|\leq 1)$ and $\mathrm{Pr}(|\hat{\tau}-\tau|\leq 2)$
(denoted by $\mathrm{Pr}_1$ and $\mathrm{Pr}_2$, respectively), which quantify the
consistency of change-point identification. On the other hand, the shift direction
estimate $\hat{\zeta}$ is derived using Equation (\ref{F4.13}) only for the
directional approach, and to the best of our knowledge, there is currently no method
for comparison in recognizing shift directions. Here the matching probability that
the estimated index $\hat{\zeta}$ of the only deviated coefficient $\beta_{\zeta}$
is indeed the true one $\zeta$, $\mathrm{Pr}(\hat{\zeta}=\zeta)$ denoted by
$\mathrm{Pr}_{\zeta}$, is selected for measuring the accuracy of identifying shift
directions. Note that for diagnosis the directional method employs the
one-coefficient shift direction index subset $\mathbb{I}_3$ rather than
$\mathbb{I}_2$, including coefficients in the main, two-factor interaction, and
three-factor interaction effects as the candidate diagnostic shift directions.

%Table 5
\begin{table}[htp]
\tabcolsep 4.0pt \vspace{-0.1cm} \centering \caption{Diagnostic comparison for
one-coefficient shifts of the first three orders} \vspace{0.3cm}
\renewcommand{\arraystretch}{1.25}
\begin{tabular}{c|c|cc|cc|cc|cc|c}\hline
$\beta$ & $\delta$ & With & Without & With & Without & With & Without & With &
Without & With
\\\hline
& & \multicolumn{2}{c|}{Avg} & \multicolumn{2}{c|}{Std} &
\multicolumn{2}{c|}{$\mathrm{Pr}_1$} & \multicolumn{2}{c|}{$\mathrm{Pr}_2$} &
$\mathrm{Pr}_{\zeta}$
\\\hline
& 0.03 & 34.3 & 36.1 & 19.5 & 24.1 & 0.188 & 0.105 & 0.267 & 0.157 & 0.522 \\
& 0.04 & 31.7 & 34.3 & 13.2 & 19.9 & 0.354 & 0.197 & 0.461 & 0.271 & 0.765 \\
$\beta_{(2)}$ & 0.05 & 30.5 & 32.1 & 7.58 & 14.5 & 0.516 & 0.320 & 0.643 & 0.421 & 0.917 \\
& 0.06 & 30.2 & 31.0 & 4.37 & 9.74 & 0.633 & 0.453 & 0.761 & 0.582 & 0.981 \\
& 0.08 & 30.0 & 30.1 & 1.87 & 3.53 & 0.787 & 0.685 & 0.882 & 0.803 & 0.999 \\
& 0.10 & 30.0 & 30.0 & 1.17 & 1.63 & 0.888 & 0.830 & 0.952 & 0.914 & 1.000
\\\hline
& 0.03 & 32.9 & 35.8 & 16.2 & 22.4 & 0.269 & 0.135 & 0.361 & 0.186 & 0.633 \\
& 0.04 & 31.1 & 32.7 & 9.06 & 16.0 & 0.469 & 0.280 & 0.595 & 0.376 & 0.874 \\
$\beta_{(1,3)}$ & 0.05 & 30.1 & 31.0 & 4.50 & 10.2 & 0.622 & 0.437 & 0.751 & 0.564 & 0.968 \\
& 0.06 & 30.0 & 30.2 & 2.75 & 5.47 & 0.725 & 0.593 & 0.835 & 0.712 & 0.992 \\
& 0.08 & 30.0 & 30.1 & 1.32 & 1.99 & 0.857 & 0.801 & 0.937 & 0.893 & 1.000 \\
& 0.10 & 30.0 & 30.0 & 0.809 & 1.01 & 0.937 & 0.913 & 0.976 & 0.965 & 1.000
\\\hline
& 0.03 & 32.1 & 34.7 & 14.1 & 20.6 & 0.309 & 0.184 & 0.418 & 0.254 & 0.565 \\
& 0.04 & 30.6 & 32.1 & 7.17 & 14.4 & 0.516 & 0.343 & 0.639 & 0.438 & 0.797 \\
$\beta_{(3,4_1)}$ & 0.05 & 30.1 & 30.8 & 3.78 & 8.10 & 0.679 & 0.531 & 0.791 & 0.648 & 0.924 \\
& 0.06 & 30.0 & 30.2 & 2.09 & 4.00 & 0.775 & 0.670 & 0.880 & 0.785 & 0.972 \\
& 0.08 & 30.0 & 30.0 & 1.04 & 1.37 & 0.899 & 0.854 & 0.962 & 0.933 & 0.997 \\
& 0.10 & 30.0 & 30.0 & 0.634 & 0.747 & 0.960 & 0.945 & 0.988 & 0.981 & 1.000
\\\hline
& 0.03 & 34.3 & 36.8 & 19.7 & 24.4 & 0.198 & 0.098 & 0.269 & 0.143 & 0.412 \\
& 0.04 & 31.4 & 34.3 & 12.6 & 20.0 & 0.357 & 0.195 & 0.463 & 0.269 & 0.646 \\
$\beta_{(1,3,4_2)}$ & 0.05 & 30.7 & 32.2 & 7.74 & 14.2 & 0.514 & 0.332 & 0.636 & 0.433 & 0.827 \\
& 0.06 & 30.1 & 30.9 & 4.23 & 9.37 & 0.648 & 0.474 & 0.769 & 0.595 & 0.921 \\
& 0.08 & 30.0 & 30.1 & 1.82 & 3.05 & 0.804 & 0.716 & 0.896 & 0.829 & 0.988 \\
& 0.10 & 30.0 & 30.0 & 1.09 & 1.48 & 0.891 & 0.849 & 0.953 & 0.925 & 0.999
\\\hline
\end{tabular}
\end{table}

Table 4.5 presents the results for some one-coefficient shifts in the effects of the
first three orders, such as $\beta_{(2)}$, $\beta_{(1,3)}$, $\beta_{(3,4_1)}$, and
$\beta_{(1,3,4_2)}$ based on 5,000 replicated simulations. It can be seen that the
estimator (\ref{F4.12}) for the directional method is always more consistent than
(\ref{F4.10}) in recognizing the change-point, giving smaller biases, smaller
standard deviations, and higher probabilities $\mathrm{Pr}_1$ and $\mathrm{Pr}_2$.
Again, this follows because the estimator (\ref{F4.12}) exploits fully the
directional information. As for the shift direction, the estimator (\ref{F4.13})
derived from the directional approach performs fairly well in that it has a high
matching probability of $\hat{\zeta}=\zeta$, and certainly this probability
increases as the shift magnitude $\delta$ increases. In addition, according to the
practical definition of shifts, estimating the shift direction may provide some
insights into multivariate categorical processes. For example, if the shift
direction index is estimated as $\hat{\zeta}=7$, by the relation
$\beta_7=\beta_{(1,3)}$, we may conclude that the interaction of factors $C_1$ and
$C_3$ deviates at the change-point.

\subsection{Revisiting the AEC Example}

Here we turn to the AEC example introduced previously to show how to implement the
proposed directional detection and diagnostic methodology in practice. The AEC
example has $p=3$ factors: leakage current (LC), dissipation factor (DF), and
capacity (CAP), and each has two levels of conforming and nonconforming. The
cross-classification counts with all factor level combinations are stored in a
three-way contingency table with $2^3=8$ cells and can be modeled by a log-linear
model.

There is a reference dataset of multivariate sampling observations, which involves
$M=120$ samples of size $N=500$. Usually, the sample size can be chosen as dozens of
the number of cells. The observation vector ${\bf n}_k$ ($k=1,\ldots,120$) is of
size $8\times 1$, for instance, ${\bf n}_1=[483,14,2,0,1,0,0,0]^T$. The elements of
${\bf n}_k$ jointly follow a multinomial distribution $\mbox{MN}(500;{\bf p})$,
where ${\bf p}$ is the cell probability vector and may shift at a change-point.

In Phase I, the main task is to check whether there are unusual patterns in the
reference dataset. Let us now employ the proposed directional detection method to
see if a change-point exists. We select the index subset $\mathbb{I}_2$, because
shifts tend to occur in main effects and two-factor interaction effects. In
addition, the desired type I error $\alpha$ is set as 0.05. By first fixing each
$i\in\mathbb{I}_2$, for each $k=1,\ldots,119$, based on the grouped data
$\sum_{j=1}^k{\bf n}_j$ and $\sum_{j=k+1}^{120}{\bf n}_j$, we use the method given
in the appendix to calculate the $-2$LRT statistic $\Lambda_{i,k}$ for testing
hypothesis (\ref{F4.5}). Then for each $i\in\mathbb{I}_2$, we obtain
$\Lambda_i=\max_k\Lambda_{i,k}$, which are 2.013, 7.099, 17.33, 5.081, 20.17, and
11.06. By using Equation (\ref{F4.14}), we calculate the approximate $p$-values of
these $\Lambda_i$ as 0.900, 0.140, 0.001, 0.323, 0.000, and 0.024, respectively.
Arranging these $p$-values from the minimum to the maximum as
$\hat{p}_{(1)},\ldots,\hat{p}_{(6)}$ ($\mathbb{I}_2$ has a cardinality $K=6$), we
find that $\hat{p}_{(i)}\leq i\alpha/6$ holds for $i=1,2$. Therefore, there is a
change-point in the reference dataset.

Since a change-point exists, the next step is to identify its location and the shift
direction, with the assumption that only one coefficient deviates in the effects of
the first three orders. Here the proposed directional diagnostic approach adopts the
candidate index subset $\mathbb{I}_3$ for the only potential shift direction. Still,
for each $i\in\mathbb{I}_3$ and $k=1,\ldots,119$, we calculate the $-2$statistic
$\Lambda_{i,k}$ for testing hypothesis (\ref{F4.5}). For each $k$, by calculating
$\max_{i\in\mathbb{I}_3}\Lambda_{i,k}$, we see that
$\max_{i\in\mathbb{I}_3}\Lambda_{i,48}$ is the maximum among all
$\max_{i\in\mathbb{I}_3}\Lambda_{i,k}$ ($k=1,\ldots,119$). According to the
estimator (\ref{F4.12}), the change-point is identified as $\hat{\tau}=48$. For
$\Lambda_{i,48}$ ($i\in\mathbb{I}_3$), it turns out that the maximum among them is
$\Lambda_{5,48}$. Hence the shift direction index is recognized as $\hat{\zeta}=5$
based on the estimator (\ref{F4.13}), and the shifted coefficient is
$\beta_5=\beta_{(1,3)}$, which represents the interaction effect of factors $C_1$
and $C_3$. In the settings of the AEC example, this corresponds to a shift in the
dependence between the characteristics LC and CAP.


\section{Summary}\label{sec4.6}

This chapter has developed a new log-linear directional change-point detection
methodology for Phase I analysis in multivariate categorical processes. The method
integrates the knowledge of potential shift directions, which are formulated as
deviations of coefficient subvectors in log-linear modeling. In addition, a
post-signal diagnostic scheme for recognizing the change-point location and the
shift direction has been proposed. Both the detection and diagnostic approaches work
well in multivariate categorical processes that follow multivariate binomial or
multivariate multinomial distributions. Monte Carlo simulations have shown high
detection power and good diagnostic consistency.

It should be pointed out that the proposed approach can be readily extended for
detecting multiple change-points by employing the binary segmentation method
recursively (Yao (1988)). Moreover, the presence of outliers has seriously adverse
effects on the modeling and monitoring of multivariate categorical processes.
Therefore, outlier detection procedures, which aim at identifying any abnormal
samples from a dataset, are quite important and deserve future research.



\section{Appendix:
Derivation of the $-2$LRT Statistic $\lambda_i$ for Testing Hypothesis
(\ref{F4.5})}\label{sec4.7}

Under $H_1$ of (\ref{F4.5}), $\widetilde{\bm\beta}^{(\mathrm{a})}$ and
$\widetilde{\bm\beta}^{(\mathrm{b})}$ have a relationship as described in
(\ref{F4.4}). Because of the constraint $\bm 1^T{\bf p}=1$, the following equations
hold
\[
\bm{1}^T\exp\big(\bm{1}\beta^{(\mathrm{a})}_0 +{\bf
X}\bm\beta^{(\mathrm{a})}\big)=1,
\]
\[
\bm{1}^T\exp\big(\bm{1}\beta^{(\mathrm{a})}_0 +{\bf
X}\bm\beta^{(\mathrm{a})}+\bm{1}\alpha_i+{\bf x}_i\delta_i\big)=1,
\]
where ${\bf x}_i$ is the $i$th column of $\bf X$. So $\beta^{(\mathrm{a})}_0$ and
$\alpha_i$ can be expressed as
\[
\beta^{(\mathrm{a})}_0=-\ln\big(\bm{1}^T\exp\big({\bf
X}\bm\beta^{(\mathrm{a})}\big)\big),
\]
\[
\alpha_i=-\ln\big(\bm{1}^T\exp\big(\bm{1}\beta^{(\mathrm{a})}_0 +{\bf
X}\bm\beta^{(\mathrm{a})}+{\bf x}_i\delta_i\big)\big).
\]
Again, assume that ${\bf n}_{\mathrm{a}}$ and ${\bf n}_{\mathrm{b}}$ are two
independent random vectors subject to $\mbox{MN}(N_{\mathrm{a}};{\bf
p}_{\mathrm{a}})$ and $\mbox{MN}(N_{\mathrm{b}};{\bf p}_{\mathrm{b}})$,
respectively. The log-likelihood function of ${\bf p}_{\mathrm{a}}$ and ${\bf
p}_{\mathrm{b}}$ can be written as
\begin{align*}
L&={\bf n}_{\mathrm{a}}^T\ln{\bf p}_{\mathrm{a}}+\ln N_{\mathrm{a}}!-\bm 1^T\ln{\bf
n}_{\mathrm{a}}!+{\bf
n}_{\mathrm{b}}^T\ln{\bf p}_{\mathrm{b}}+\ln N_{\mathrm{b}}!-\bm 1^T\ln{\bf n}_{\mathrm{b}}!\\
&={\bf n}_{\mathrm{a}}^T\widetilde{\bf X}\widetilde{\bm\beta}^{(\mathrm{a})}+\ln
N_{\mathrm{a}}!-\bm 1^T\ln{\bf n}_{\mathrm{a}}!+{\bf n}_{\mathrm{b}}^T\widetilde{\bf
X}\widetilde{\bm\beta}^{(\mathrm{b})}+\ln N_{\mathrm{b}}!-\bm 1^T\ln{\bf n}_{\mathrm{b}}!\\
&=(N_{\mathrm{a}}+N_{\mathrm{b}})\beta^{(\mathrm{a})}_0+({\bf n}_{\mathrm{a}}+{\bf
n}_{\mathrm{b}})^T{\bf X}\bm\beta^{(\mathrm{a})}+N_{\mathrm{b}}\alpha_i+{\bf
n}_{\mathrm{b}}^T{\bf
x}_i\delta_i\\
&\ \ \;\;+\ln N_{\mathrm{a}}!-\bm 1^T\ln{\bf n}_{\mathrm{a}}!+\ln
N_{\mathrm{b}}!-\bm 1^T\ln{\bf n}_{\mathrm{b}}!.
\end{align*}

To determine the MLEs of ${\bf p}_{\mathrm{a}}$ and ${\bf p}_{\mathrm{b}}$, the
first-order partial derivatives of $L$ with respect to both
$\bm\beta^{(\mathrm{a})}$ and $\delta_i$ are required. After some simplifications,
we have
\[
\frac{\partial L}{\partial\bm\beta^{(\mathrm{a})}}={\bf X}^T({\bf
n}_{\mathrm{a}}+{\bf n}_{\mathrm{b}}-N_{\mathrm{a}}{\bf
p}_{\mathrm{a}}-N_{\mathrm{b}}{\bf p}_{\mathrm{b}}),
\]
\begin{align*}
\frac{\partial L}{\partial\delta_i}=s(\delta_i)=&\;{\bf x}_i^T{\bf
n}_{\mathrm{b}}-N_{\mathrm{b}}{\bf x}_i^T\exp\big[\bm{1}\beta^{(\mathrm{a})}_0+{\bf X}\bm\beta^{(\mathrm{a})}+{\bf x}_i\delta_i\\
&\;-\bm{1}\ln\big(\bm{1}^T\exp\big(\bm{1}\beta^{(\mathrm{a})}_0 +{\bf
X}\bm\beta^{(\mathrm{a})}+{\bf x}_i\delta_i\big)\big)\big].
\end{align*}
Let
\[
{\bf k}(\delta_i)=\exp\big[\bm{1}\beta^{(\mathrm{a})}_0+{\bf
X}\bm\beta^{(\mathrm{a})}+{\bf
x}_i\delta_i-\bm{1}\ln\big(\bm{1}^T\exp\big(\bm{1}\beta^{(\mathrm{a})}_0 +{\bf
X}\bm\beta^{(\mathrm{a})}+{\bf x}_i\delta_i\big)\big)\big].
\]
The second-order partial derivative $L$ with respect to $\delta_i$ can be formulated
as
\begin{align*}
s^{\prime}(\delta_i)=&\;\frac{\partial^2 L}{\partial\delta^2_i}
=\;-N_{\mathrm{b}}{\bf x}_i^T\di\big({\bf k}(\delta_i)\big){\bf x}_i +
N_{\mathrm{b}}{\bf x}_i^T{\bf k}(\delta_i){\bf k}^T(\delta_i){\bf x}_i,
\end{align*}
where $\di({\bf a})$ is the diagonal square matrix with the column vector ${\bf a}$
as its diagonal elements. It is obvious that
\begin{align*}
&s(0)={\bf x}_i^T{\bf n}_{\mathrm{b}}-N_{\mathrm{b}}{\bf x}_i^T
\exp\big(\bm{1}\beta^{(\mathrm{a})}_0+{\bf X}\bm\beta^{(\mathrm{a})}\big)={\bf
x}_i^T({\bf
n}_{\mathrm{b}}-N_{\mathrm{b}}{\bf p}_{\mathrm{a}}),\\
&s^{\prime}(0)=-N_{\mathrm{b}}{\bf x}_i^T\di\big({\bf p}_{\mathrm{a}}\big){\bf x}_i+
N_{\mathrm{b}}{\bf x}_i^T{\bf p}_{\mathrm{a}}{\bf p}_{\mathrm{a}}^T{\bf
x}_i=-N_{\mathrm{b}}{\bf x}_i^T\bm{\Sigma}_{\mathrm{a}}{\bf x}_i,
\end{align*}
where $\bm\Sigma_{\mathrm{a}}=\di\big({\bf p}_{\mathrm{a}}\big)-{\bf
p}_{\mathrm{a}}{\bf p}_{\mathrm{a}}^T$. By performing the first-order Taylor
expansion of $s(\delta_i)$ at $\delta_i=0$, we have
\[
s(\delta_i)\approx s(0)+s^{\prime}(0)\delta_i.
\]
To obtain the MLE of $\delta_i$, let $s(\delta_i)$ be 0. Therefore, the MLE of
$\delta_i$ should approximately satisfy
\begin{align}
\delta_i\approx-\frac{s(0)}{s^{\prime}(0)}=\frac{1}{N_{\mathrm{b}}}\big({\bf
x}_i^T\bm{\Sigma}_{\mathrm{a}}{\bf x}_i\big)^{-1}{\bf x}_i^T\big({\bf
n}_{\mathrm{b}}-N_{\mathrm{b}}{\bf p}_{\mathrm{a}}\big).\label{F4.15}
\end{align}
On the other hand, by performing the first-order Taylor expansion of ${\bf
p}_{\mathrm{b}}$ at ${\bf p}_{\mathrm{a}}$ (i.e., $\delta_i=0$), we have
\begin{align}
{\bf p}_{\mathrm{b}}=&\;\exp\big(\bm{1}\beta^{(\mathrm{a})}_0+{\bf
X}\bm\beta^{(\mathrm{a})}+\bm{1}\alpha_i+{\bf
x}_i\delta_i\big)\nonumber\\
=&\;\exp\big(\bm{1}\beta^{(\mathrm{a})}_0+{\bf X}\bm\beta^{(\mathrm{a})}-
\bm{1}\ln\big(\bm{1}^T\exp\big(\bm{1}\beta^{(\mathrm{a})}_0 +{\bf
X}\bm\beta^{(\mathrm{a})}+{\bf x}_i\delta_i\big)\big)+{\bf
x}_i\delta_i\big)\nonumber\\
\approx &\;\exp\big(\bm{1}\beta^{(\mathrm{a})}_0+{\bf
X}\bm\beta^{(\mathrm{a})}\big)+\big(\di\big({\bf p}_{\mathrm{a}}\big){\bf
x}_i-{\bf p}_{\mathrm{a}}{\bf p}_{\mathrm{a}}^T{\bf x}_i\big)\delta_i\nonumber\\
=&\;{\bf p}_{\mathrm{a}}+\bm{\Sigma}_{\mathrm{a}}{\bf x}_i\delta_i.\label{F4.16}
\end{align}
Therefore, similar to $\delta_i$, the MLE of $\bm\beta^{(\mathrm{a})}$ should
satisfy
\begin{align}
\frac{\partial L}{\partial\bm\beta^{(\mathrm{a})}}\approx{\bf X}^T({\bf
n}_{\mathrm{a}}+{\bf n}_{\mathrm{b}}-N_{\mathrm{a}}{\bf
p}_{\mathrm{a}}-N_{\mathrm{b}}{\bf
p}_{\mathrm{a}}-N_{\mathrm{b}}\bm{\Sigma}_{\mathrm{a}}{\bf x}_i\delta_i)=\bm
0.\label{F4.17}
\end{align}
By substituting $\delta_i$ in (\ref{F4.15}) into (\ref{F4.17}), we have
\begin{align}
{\bf X}^T\big[{\bf n}_{\mathrm{a}}+{\bf n}_{\mathrm{b}}-N_{\mathrm{a}}{\bf
p}_{\mathrm{a}}-N_{\mathrm{b}}{\bf p}_{\mathrm{a}}-\bm{\Sigma}_{\mathrm{a}}{\bf
x}_i\big({\bf x}_i^T\bm{\Sigma}_{\mathrm{a}}{\bf x}_i\big)^{-1}{\bf x}_i^T\big({\bf
n}_{\mathrm{b}}-N_{\mathrm{b}}{\bf p}_{\mathrm{a}}\big)\big]=\bm 0.\label{F4.18}
\end{align}
After some calculations, (\ref{F4.18}) plus the constraint $\bm 1^T{\bf
p}_{\mathrm{a}}=1$ leads to
\begin{align}
{\bf p}_{\mathrm{a}}=\left[\begin{array}{c} \bm 1^T\\
{\bf X}^T(N_{\mathrm{a}}+N_{\mathrm{b}})-{\bf X}^T\bm{\Sigma}_{\mathrm{a}}{\bf
A}_iN_{\mathrm{b}}
\end{array}\right]^{-1}\left[\begin{array}{c}1\\
{\bf X}^T({\bf n}_{\mathrm{a}}+{\bf n}_{\mathrm{b}})-{\bf
X}^T\bm{\Sigma}_{\mathrm{a}}{\bf A}_i{\bf n}_{\mathrm{b}}
\end{array}\right],\label{F4.19}
\end{align}
where ${\bf A}_i={\bf x}_i\big({\bf x}_i^T\bm{\Sigma}_{\mathrm{a}}{\bf
x}_i\big)^{-1}{\bf x}_i^T$. Note that (\ref{F4.19}) may be regarded as ${\bf
p}_{\mathrm{a}}={\bf f}({\bf p}_{\mathrm{a}})$.

By performing numerical iterations based on (\ref{F4.19}) until some stopping
criterion is reached, we obtain the MLE $\widehat{\bf p}^{H_1}_{\mathrm{a}}$ of
${\bf p}_{\mathrm{a}}$ under $H_1$ and hence
$\widehat{\bm\Sigma}^{H_1}_{\mathrm{a}}$. Generally, the initial value of ${\bf
p}_{\mathrm{a}}$ in the above iteration can be set as $({\bf n}_{\mathrm{a}}+{\bf
n}_{\mathrm{b}})/(N_{\mathrm{a}}+N_{\mathrm{b}})$, and our simulations show that
this iteration will stop after only a few steps. The MLE $\hat{\delta}^{H_1}_i$ of
$\delta_i$ can be calculated from (\ref{F4.15}). Based on $\widehat{\bf
p}^{H_1}_{\mathrm{a}}$ and $\hat{\delta}^{H_1}_i$, and according to (\ref{F4.16}),
the MLE $\widehat{\bf p}^{H_1}_{\mathrm{b}}$ of ${\bf p}_{\mathrm{b}}$ under $H_1$
can also be derived. Under $H_0$ of (\ref{F4.5}), it is known that the MLEs of ${\bf
p}_{\mathrm{a}}$ and ${\bf p}_{\mathrm{B}}$ are $\widehat{\bf
p}^{H_0}_{\mathrm{a}}=\widehat{\bf p}^{H_0}_{\mathrm{b}}=({\bf n}_{\mathrm{a}}+{\bf
n}_{\mathrm{b}})/(N_{\mathrm{a}}+N_{\mathrm{b}})$. Based on these, the $-2$LRT
statistic $\lambda_i$ for testing (\ref{F4.5}) can finally be formulated as
\begin{align}\label{F4.20}
\lambda_i&=2{\bf n}_{\mathrm{a}}^T\ln\widehat{\bf p}^{H_1}_{\mathrm{a}}+2{\bf
n}_{\mathrm{b}}^T\ln\widehat{\bf p}^{H_1}_{\mathrm{b}}-2({\bf n}_{\mathrm{a}}+{\bf
n}_{\mathrm{b}})^T\ln\Big(\frac{{\bf n}_{\mathrm{a}}+{\bf
n}_{\mathrm{b}}}{N_{\mathrm{a}}+N_{\mathrm{b}}}\Big).
\end{align}

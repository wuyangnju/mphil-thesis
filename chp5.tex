\chapter{Conclusion}\label{chp5}

This thesis provides a systematic framework of methodology that deals with
statistical process control for multivariate categorical processes, including
multivariate binomial processes and multivariate multinomial processes. These
developed techniques consider both marginal factor distributions and dependence or
cross-classifications among multiple factors by employing log-linear models, which
integrate main effects and interaction effects.

In particular, the log-linear multivariate categorical (LMC) control chart shown in
Chapter 2 and the log-linear directional (LLD) control chart illustrated in Chapter
3 are for online monitoring in Phase II. The LLD chart exploits more information
than the LMC chart, in that the former considers directional shifts in the form of
deviations in coefficient vectors of a log-linear model. The directional method for
change-point detection proposed in Chapter 4 can be employed in Phase I analysis for
multivariate categorical processes, which utilizes similar information to the LLD
chart.

Albeit a relatively complete methodology, there still remain many research topics in
SPC for multivariate categorical processes. The first issue is about the sparsity
phenomenon in a multi-way contingency table mentioned in Chapter 2. In fact, it is
somewhat similar to that in high-quality processes (e.g., Nelson (1994); McCool and
Joyner-Motley (1998)), where the proportion of nonconforming products is extremely
small, say, in the level of parts-per-million. Consequently, the $p$-chart will be
of little use, since the estimated nonconforming fraction is mostly zero. Therefore,
it is important to use a dataset large enough to achieve a reasonably accurate
estimate of parameters. We refer to Yang et al. (2002) for a discussion on the
effect of the dataset size in Phase I on estimating the control limits of a
geometric chart (Kaminsky et al. (1992)), which is developed particularly for
monitoring high-quality processes. Based on log-linear models, the LMC chart should
also be conveniently modified to monitor multivariate categorical high-quality
processes (Niaki and Abbasi (2007)).

We also find that for a factor with three or more attribute levels, there is usually
an order between them, such as good, marginal, and bad. However, traditionally a
multi-level factor is treated as nominal data, and therefore the ordinal information
between attribute levels is neglected and not exploited. Our ongoing research is
utilizing the midrank of each level (Agresti (2010)) and developing an ordinal
control chart for detecting location shifts in univariate ordinal categorical
processes. We believe that it can be extended to a multivariate version.

In addition, the considered log-linear model is related to the so-called
context-tree model, which describes dependent categorical data with finite attribute
levels in terms of context dependency (Ben-Gal et al. (2003); Brice and Jiang
(2009)). Such dependency means that the statistical distribution of a new sample is
conditional on a set of the most recent samples that precedes it in a data stream
(Brice et al. (2012)). We know that the multivariate binomial or multivariate
multinomial data in this thesis are assumed to be temporally independent, but the
context dependency should also apply to a multi-way contingency table in that at a
time point a cell count may depend spatially on the counts in its neighborhood
cells. This may be another point of view that characterizes a multi-way contingency
table and deserves the development of a control chart based on it.

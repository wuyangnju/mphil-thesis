\bibliographystyle{abbrv}
\begin{thebibliography}{99}

\bibitem{ras-seq-parallel} J. LUO, L. Jeff HONG, Barry L. Nelson, Y. WU. Fully Sequential Procedures for Large-Scale Ranking-and-Selection Problems in Parallel Computing Environments. Operation Research, 2013.

\bibitem{potwsc11ras} J. LUO, L. Jeff HONG. Large-scale ranking and selection using cloud computing. Proceedings of the 2011 Winter Simulation Conference, 4051–4061.

\bibitem{ras-seq-jeff} L. J Hong. Fully sequential indifference-zone selection procedures with variance-dependent sampling. Naval Research Logistics, 53:464-476, 2006.

\bibitem{ras-recent-advances} Seong-Hee Kim, Barry L. Nelson. Recent Advances in Ranking and Selection. Winter Simulation Conference, 2007.

\bibitem{ehiorams06ras} Kim, S. H. and B. L. Nelson. Elsevier Handbooks in Operations Research and Management Science: Simulation, Chapter 17. Selecting the best system, pp. 501–534. Elsevier.

\bibitem{ms05ras} Jurgen Branke, Stephen E. Chick, Christian Schmidt. Selecting a Selection Procedure. Management Science, 2005.

\bibitem{cistam1978rinott} Rinott, Y. On two-stage selection procedures and related probability-inequalities. Communications in Statistics - Theory and Methods A7, 799–811.

\bibitem{rinott-constant} Wilcox, Rand R. A Table for Rinott's Selection Procedure. Journal of Quality Technology, April 1984, pp. 97-100.

\bibitem{tomacs01kn} Kim, S. H. and B. L. Nelson. A fully sequential procedure for indifference-zone selection in simulation. ACM Transactions on Modeling and Computer Simulation 11, 251.273.

\bibitem{or01nsgs} Nelson, B. L., J. Swann, D. Goldsman, and W. Song. Simple procedures for selecting the best simulated system when the number of alternatives is large. Operations Research 49, 950–963.

\bibitem{potwsc09ovs} Hong, L. J. and B. L. Nelson. A brief introduction to optimization via simulation. Proceedings of the 2009 Winter Simulation Conference, 75–85.

\bibitem{potwsc05ras} Chen, E. J. Using parallel and distributed computing to increase the capability of selection procedures. In Proceedings of the 2005 Winter Simulation Conference, pp. 723–731.

\bibitem{toams1954iz} Bechhofer, R. E. A single-sample multiple decision procedure for ranking means of normal populations with known variances. The Annals of Mathematical Statistics 25, 16–39.

\bibitem{nyjws95iz} Bechhofer, R. E., T. J. Santner, and D. M. Goldsman. Design and Analysis of Experiments for Statistical Selection, Screening, and Multiple Comparisons. New York: John Wiley \& Sons.

\bibitem{google-map-reduce} D. Jeffrey, G. Sanjay. MapReduce: Simplified Data Processing on Large Clusters. Google Inc., 2004.

\bibitem{amdahl} Amdahl, Gene M. Validity of the single processor approach to achieving large scale computing capabilities. Proceedings of the April 18–20, 1967, spring joint computer conference: 483–485.

\bibitem{gustafson} Gustafson, John L. Reevaluating Amdahl's law. Communications of the ACM 31 (5): 532–533.

\bibitem{pca97} Singh, David Culler; J.P. Parallel computer architecture ([Nachdr.] ed.). San Francisco: Morgan Kaufmann Publ.

\bibitem{cotacm90fuji} Fujimoto, R. M. Parallel discrete event simulation. Communications of the ACM 33, 30–53.

\bibitem{scsmasm10fuji} Fujimoto, R. M., A. W. Malik, and A. J. Park (2010). Parallel and distributed simulation in the cloud. SCS Modeling and Simulation Magazine 1.

\bibitem{cissac1985ss} Koenig, L. W., and A. M. Law. A procedure for selecting a subset of size m containing the l best of k independent normal populations, with applications to simulation. Communications in Statistics: Simulation and Computation 1985 14:719–734.

\bibitem{smoms93threestage} Buzacott, J. A., and J. G. Shanthikumar. Stochastic Models of Manufacturing Systems. 1993.

\bibitem{moore} Moore, Gordon E. Cramming more components onto integrated circuits. Electronics Magazine.

\end{thebibliography}
